
\documentclass[11pt]{article}
\usepackage{fullpage}
\usepackage{siunitx}
\usepackage{hyperref,graphicx,booktabs,dcolumn}
\usepackage{stata}
\usepackage[x11names]{xcolor}
\usepackage{natbib}
\usepackage{chngcntr}
\usepackage{pgfplotstable}
\usepackage{pdflscape}

\usepackage{multirow}
\usepackage{booktabs}

\newcommand{\specialcell}[2][c]{%
  \begin{tabular}[#1]{@{}l@{}}#2\end{tabular}}
\newcommand{\specialcelll}[3][c]{%
  \begin{tabular}[#1]{@{}l@{}l@{}}#2\end{tabular}}
\newcommand{\specialcellll}[4][c]{%
  \begin{tabular}[#1]{@{}l@{}l@{}l@{}}#2\end{tabular}}
  
  
\newcommand{\thedate}{\today}

\counterwithin{figure}{section}
\counterwithin{table}{section}

\bibliographystyle{unsrt}

\begin{document}

\begin{titlepage}
    \begin{flushright}
        \Huge
        \textbf{Lipid-lowering strategies for primary prevention of coronary heart disease 
in the United Kingdom: A cost-effectiveness analysis}
\color{black}
\rule{16cm}{2mm} \\
\Large
\color{black}
Protocol \\
\thedate \\
\color{blue}
\url{https://github.com/jimb0w/LDL} \\
\color{black}
       \vfill
    \end{flushright}
        \Large

\noindent
Jedidiah Morton \\
Research Fellow \\
\color{blue}
\href{mailto:Jedidiah.Morton@Monash.edu}{Jedidiah.Morton@monash.edu} \\ 
\color{black}
\\
Monash University, Melbourne, Australia \\\
\\
\end{titlepage}

\begin{abstract}


This is the protocol for the paper \emph{Lipid-lowering strategies for primary prevention of coronary heart disease 
in the United Kingdom: A cost-effectiveness analysis}. In this protocol, the construction of 
a microsimulation model (based on the results of Mendelian randomisation analyses)
that ages people from the UK Biobank study from 30-85 years, tracking coronary heart disease (CHD),
is outlined. Using this model, the efficacy of different lipid-lowering strategies (LLS) for the primary
prevention of coronary heart disease (CHD) are simulated. These strategies include: low/moderate intensity statins;
high intensity statins; low/moderate intensity statins and ezetimibe; and inclisiran. 
All lead to varying to degress of low-density lipoprotein-cholesterol (LDL-C) reduction, come at a vast range of costs, 
and may have different levels of adherence, which makes comparison of cost-effectiveness
of these agents interesting. These lipid lowering strategies are also tested across a range of ages at intervention, because
the risk of CHD is proportional to the cumulative exposure to LDL-C, and, thus, earlier lowering of LDL-C will lead to greater
benefit in the longer-term; however, even if it is more effective to lower LDL-C at a younger age, at what age it is most
cost-effective to intervene is unclear. Therefore, we use the cumulative causal effect of LDL-C on CHD risk in this model
to estimate the effectiveness and cost-effectiveness of each LLS at ages 30, 40, 50, and 60, from the UK healthcare perspective. 
These analyses mostly utilise data from the UK Biobank study, and for estimates not directly estimable from the UK Biobank, 
other UK sources are used (for utilities and healthcare costs). The analyses use the UK willingness-to-pay 
threshold range of \textsterling 20,000-30,000. 

The results show that LLS initiated earlier in life prevent more MIs and are more cost-effective than LLS initiated later in life. 
Moreover, because absolute risk is higher in males and people with higher LDL-C, 
cost-effectiveness improves by targeting these interventions to these clinical sub-groups. 
Indeed, some of the statin-based LLS were cost-saving in people with high LDL-C, 
although they were also cost-effective at all ages for most sub-groups. 
Inclisiran was not cost-effective in any subgroup or any simulation. 
These results demonstrate that statin-based LLS are a highly cost-effective method of 
reducing the lifetime risk of CHD when initiated from as young as 30 years of
age and support a shift in the approach to primary prevention of CHD away from
short-term absolute risk estimates to early and sustained lowering of LDL-C.

\end{abstract}

\pagebreak
\tableofcontents
\pagebreak
\listoftables
\pagebreak
\listoffigures
\pagebreak

\pagebreak
\Large
\noindent
\textbf{Preface}
\normalsize

This work was completed with funding from the National Health and Medical Research Council of Australia, Ideas grant, 
Application ID: 2012582, which was obtained by Zanfina Ademi as the Chief Investigator, who supervised this work. Contributors 
to this project include: Clara Marquina, Melanie Lloyd, Gerald Watts, Sophia Zoungas, Danny Liew, and Zanfina Ademi. 

To generate this document, the Stata package texdoc \cite{Jann2016Stata} was used, which is 
available from: \color{blue} \url{http://repec.sowi.unibe.ch/stata/texdoc/} \color{black} (accessed 14 November 2022). 
The final Stata do file and this pdf are available at: \color{blue} \url{https://github.com/jimb0w/LDL} \color{black} -- 
The code in this pdf is 99.9999\% complete, except for certain unicode characters used in some tables and figures, 
so for complete reproducibility use the do file. 
For most plots throughout, the colour schemes used are \emph{inferno} and \emph{viridis} from the
\emph{viridis} package \cite{GarnierR2021}. \\

\pagebreak
\Large
\noindent
\textbf{Abbreviations}
\normalsize

\begin{itemize}
\item CHD: Coronary heart disease
\item LDL-C: Low-density lipoprotein-cholesterol
\item LLS: Lipid lowering strategy
\item MI: Myocardial infarction
\item MR: Mendelian randomisation
\item NHS: National Health Service (of the United Kingdom)
\item OSA: One-way sensitivity analysis
\item PSA: Probabilistic sensitivity analysis
\item QALY: Quality-adjusted life year
\item RR: Relative risk
\item SE: Standard error
\item UK: United Kingdom
\item YLL: Years of life lived
\end{itemize}

\pagebreak
\section{Introduction}

Coronary heart disease (CHD) remains a leading cause of morbidity and mortality worldwide \cite{RothJACC2020}.
An important causal determinant of CHD are low-density lipoproteins (LDL) \cite{FerenceEHJ2017}, 
whereby exposure to high levels of LDL over time exerts a cumulative effect on the risk for CHD 
(i.e., risk is proportional to both magnitude and duration of exposure) \cite{FerenceJAMA2019}.
However, robust estimates of the cost-effectiveness of strategies 
for early and intensive pharmacological lowering of LDL-cholesterol (LDL-C) on the lifetime 
risk of CHD based on this causal effect are lacking \cite{AdemiPE2022}. 

These estimates would ideally be derived from a randomized clinical trial; 
however, such a trial is unlikely to ever be undertaken given that it would 
take multiple decades and would be prohibitively expensive. 
Therefore, in this analysis, the causal effect of LDL-C on CHD derived from 
Mendelian randomisation (MR) analyses is used to develop a microsimulation model that integrates
the cumulative, causal effect of lowering low-density lipoprotein-cholesterol (LDL-C) on risk of CHD. 
The rationale for this, and the proposed approach have been reviewed in detail previously
\cite{AdemiPE2022}. The primary data source will be the UK Biobank 
\cite{SudlowPLOSMED2015}, and where data from the UK Biobank is not directly available (utilities and
healthcare costs), the best available sources are selected. 

Specifically, here, I examine the effect of the following lipid-lowering strategies (LLS) for primary
prevention of CHD: low/moderate intensity statins; high intensity statins; low/moderate intensity statins 
and ezetimibe; and inclisiran. Indeed, because these strategies
lead to varying to degress of low-density lipoprotein-cholesterol (LDL-C) reduction, come at a vast range of costs, 
and may have different levels of adherence, comparison of cost-effectiveness
of these agents is of considerable interest, and may help select the best agents to implement for primary
prevention of CHD at scale. 

Let us first outline why we would consider so many agents. Statins are undoubtedly the standard for LDL-C reduction,
as they are efficacious and cheap. However, side effects occur, and statin intolerance occurs in a non-trivial
number of people \cite{BytyciEHJ2022}, and the risk of intolerance increases with increasing statin dose \cite{BytyciEHJ2022}. 
To combat these issues, other agents can be used in combination with, or instead of statins. Ezetimibe
offers an option for intensifying LDL-C reduction when used in combination with low/moderate intensity statins, 
which leads to greater efficacy (in terms of LDL-C reduction) and less intolerance-related drug discontinuation 
\cite{KimLancet2022,AmbeAth2014}. 
Alternatively, Inclisiran is injected twice a year, and achieves large and sustained LDL-C reductions with just this twice-yearly dosing
scheme; Inclisiran may offer an advantage over the other therapies, as the others are all orally taken, and some have suggested 
the twice-yearly dosing scheme will improve adherence \cite{GencerEHJ2022}. (Note: improved adherence to injectables is only speculation, 
and has not yet been demonstrated; it is also possible people are less likely to agree to use an injectable when a 
\emph{much} cheaper pill is available.) This also explains why we wouldn't consider Monoclonal Antibodies to PCSK9 --
they are injected much more frequently, come at a far higher cost, and could be very difficult to manufacture at scale, making them
poor options for primary prevention of CHD. 

So, with all this information, the question arises: Which of these LLS are cost-effective, and at what 
age should they be implemented for primary prevention of CHD? The answer to this question will of course vary for different
groups in the population, especially based on absolute risk for CHD over the lifetime. Thus, 
in this analysis, the answer to this question will be sought for the overall UK Biobank population, 
and stratified by sub-groups (male and female, and by baseline LDL-C). 


\pagebreak
\section{Model structure}
\label{modelstructure}

To orient ourselves before even touching the data, let's first look at the structure of the model 
that I will be building here (figure~\ref{Schematic}). All individuals will start in the 
``No CHD'' health state at age 30 years (the risk of CHD before age 30 is
negligible). The model then simulates the cohort up to age 85 years, ageing individuals in 0.1-year increments.
In each cycle, individuals in the ``No CHD'' health state can move out of the ``No CHD'' health state and 
into the ``CHD'' health state by experiencing a non-fatal MI, or move into the ``Death'' health state
by dying from Non-CHD causes or from from a fatal MI or CHD death. ``Death'' is the only absorbing
state in this model. The other transition is from ``CHD'' to ``Death'', which occurs when people with 
CHD experience death from any cause. 

The focus of this model is for \emph{primary prevention} of CHD. This has two consequences for the model. First, 
repeat events are not considered -- once an individual experiences a non-fatal MI, they are only at risk
of all-cause mortality (the disutilities and healthcare costs of repeat MIs are assumed to be captured in the
chronic utilities and costs of this health state). Second, the risk of all-cause mortality in people with 
existing CHD is assumed to be unrelated to their LDL-C or cumulative LDL-C. 

The UK Biobank will be used as the population of the model and 
to estimate all these transition probabilities, the methods of which are 
detailed in section ~\ref{TPs}. Briefly, the transitions out of the ``No CHD'' health state are 
estimated using age-period-cohort models \cite{CarstensenSTATMED2007}, and the transition from 
``CHD'' to ``Death'' using a similar model. The incidence of non-fatal MI and rate of Fatal MI/CHD death
are assumed to be proportional to mean cumulative LDL-C; thus, these rates are adjusted as proportional
to the mean cumulative LDL-C over the lifetime for each UK Biobank participant (calculated in section
~\ref{LDLtrajsection}). These are the transition probabilities on which the interventions to lower LDL-C
operate (through altering mean cumulative LDL-C). 

This has now oriented us to what we hope to achieve with the data, so let's proceed to data cleaning.

\begin{figure}
    \centering
    \includegraphics[width=0.8\textwidth]{Model schematic.pdf}
    \caption{Model structure. Dashed lines are transition probabilities influenced by mean cumulative LDL-C; solid lines are transition probabilities not influenced by LDL-C.}
    \label{Schematic}
\end{figure}

\clearpage
\pagebreak
\section{Data cleaning}

Even though the UK Biobank is curated and probably very sound, it is worth conducting data 
cleaning/sanity checks to minimise the probability of major mistakes, in addition to creating a 
usable working dataset. 

First let's load the dataset, one variable at a time (for speed).
The variables of interest for this study are:
\begin{itemize}
\item Participant ID (Unique Data Identifier (UDI): eid))
\item Sex (UDI: 31-0.0)
\item Date of assessment (UDI: 53-0.0--53-3.0)
\item UK Biobank assessment centre (UDI: 54-0.0--54-3.0)
\item Year of birth (UDI: 34-0.0)
\item Month of birth (UDI: 52-0.0)
\item Date and source of first myocardial infarction (MI) (UDI: 42000-0.0--42001-0.0)
\item Date and causes of death (UDI: 40000-0.0--40002-1.14)
\item LDL-C measurement and date (UDI: 30780-0.0--30781-1.0)
\item Pre-coded medication use (UDI: 6153-0.0--6153-3.3 (females) \& 6177-0.0--6177-3.2 (males))
\end{itemize}

Then I have to drop all the people who have withdrawn consent since the download of this 
data extract. 

\color{Blue4}
\begin{stlog}\input{log/1.log.tex}\end{stlog}
\begin{stlog}\input{log/2.log.tex}\end{stlog}
\color{black}
Date of assessment is the first variable of interest.
\color{Blue4}
\begin{figure}
    \centering
    \includegraphics[width=0.8\textwidth]{log/3.pdf}
    \caption{Initial study visit date for UK Biobank participants}
    \label{DAhist}
\end{figure}
\begin{stlog}\input{log/3.log.tex}\end{stlog}
\color{black}

In figure~\ref{DAhist} the initial pilot study is evident before the main recruitment phase (that stops around Christmas evidently). 
All looks fine for this. 

Now place of assessment. It's also worth placing people in England/Scotland/Wales as this will become important later when using 
hospital and mortality data.

\color{Blue4}
\begin{stlog}\input{log/4.log.tex}\end{stlog}
\color{black}

This extract doesn't have date of birth, so it must be created from month and year of birth. 

\color{Blue4}
\begin{stlog}\input{log/5.log.tex}\end{stlog}
\begin{stlog}\input{log/6.log.tex}\end{stlog}
\color{black}

Only a small number of people (3) have no available year or month of birth, who will be excluded from analyses. 

\color{Blue4}
\begin{figure}
    \centering
    \includegraphics[width=0.8\textwidth]{log/7.pdf}
    \caption{Date of birth for UK Biobank participants}
    \label{DOBhist}
\end{figure}
\begin{stlog}\input{log/7.log.tex}\end{stlog}
\color{black}

Overall, date of birth looks sensible (figure~\ref{DOBhist}).
Next, MIs.
Note for MIsource the values mean the following:
\begin{itemize}
\item 0 Self-report only
\item 11 Hospital primary
\item 12 Death primary
\item 21 Hospital secondary
\item 22 Death secondary
\end{itemize}

\color{Blue4}
\begin{figure}
    \centering
    \includegraphics[width=0.8\textwidth]{log/8.pdf}
    \caption{Date of first MI for UK Biobank participants}
    \label{MIhist}
\end{figure}
\begin{stlog}\input{log/8.log.tex}\end{stlog}
\color{black}

So there are about 15885 incident MIs (I use about because dates of follow-up haven't been defined yet), 
which should give us some power. What is a bit concerning is that 39 \% are from a secondary diagnosis on an admission, 
because you would expect that an MI would be the primary reason for a hospital admission and be coded as such, 
whereas a secondary diagnosis might indicate it is a historical MI (meaning the "date", and therefore the age, of the incident MI might be off).
It's probably reasonable to assume that having an MI as any diagnosis is sufficient to indicate that this individual 
has had an MI.
Also note that there are 183 missing dates for MI, but because the interest in this study is only incident MI and
they're all self-reported, it's okay to assume these are just prevalent MI at first visit. 

Date of death is next. Not sure why there are two fields \ldots

\color{Blue4}
\begin{stlog}\input{log/9.log.tex}\end{stlog}
\color{black}

No one has two dates of death, so that's good. 

\color{Blue4}
\begin{figure}
    \centering
    \includegraphics[width=0.8\textwidth]{log/10.pdf}
    \caption{Date of death for UK Biobank participants}
    \label{dodhist}
\end{figure}
\begin{stlog}\input{log/10.log.tex}\end{stlog}
\color{black}

So this makes sense, and the deaths that coincide with MIs seem reasonable, which is good. 


In terms of cause of death, the only deaths of interest in this study are deaths from CHD,
so it can be defined easily here.
The definition of CHD death used will be the same as Ference et al. \cite{FerenceJAMA2019}
because that's where the estimate of the effect of LDL-C on MI and coronary death will come from. 
However, it appears Ference et al. used all contributing causes of death to define coronary death, which was fine for 
their purposes. But for this study, it only makes sense to include coronary death if
it is the underlying cause of death, as if someone dies of cancer/dementia or anything
else with CVD, it's not really fair to assume that LDL reduction would have prevented
that death. 

\color{Blue4}
\begin{stlog}\input{log/11.log.tex}\end{stlog}
\color{black}

Now LDL-C. There are two values for LDL-C, of varying completeness:

\color{Blue4}
\begin{stlog}\input{log/12.log.tex}\end{stlog}
\color{black}

This will be collapsed into a single value for LDL-C, as most people have only one. 
While LDL-C will vary over time for some people, it is likely LDL-C at one 
point in time is a good proxy for LDL-C over the lifetime as most people don't 
have substantial variation in their LDL-C over time 
(unless they start a lipid-lowering therapy) \cite{DuncanJAHA2019}. 

\color{Blue4}
\begin{figure}
    \centering
    \includegraphics[width=0.8\textwidth]{log/13.pdf}
    \caption{LDL-C for UK Biobank participants}
    \label{LDLhist}
\end{figure}
\begin{stlog}\input{log/13.log.tex}\end{stlog}
\color{black}

Again, still looking fine (figure~\ref{LDLhist}), as expected for UK Biobank. 
Not everyone has a value, so again these individuals will need to be 
dropped for certain analyses. 

It's also worth stratifying LDL-C by lipid-lowering therapy (LLT) use
to ensure that field makes sense:

\color{Blue4}
\begin{figure}
    \centering
    \includegraphics[width=0.8\textwidth]{log/14.pdf}
    \caption{LDL-C for UK Biobank participants by LLT status}
    \label{LDLhist2}
\end{figure}
\begin{stlog}\input{log/14.log.tex}\end{stlog}
\color{black}

Finally, strip the dataset to keep only what is needed.

\color{Blue4}
\begin{stlog}\input{log/15.log.tex}\end{stlog}
\color{black}

\clearpage
\pagebreak
\section{Transition probabilities}
\label{TPs}

In this section, I outline the methods used to estimate the transition probabilities 
for the model (at this stage, unadjusted for LDL-C). 
The first step in this is converting the UK Biobank data to survival-time data. 
For that a censoring date is needed. According to UK Biobank 
(\color{blue}
\url{https://biobank.ctsu.ox.ac.uk/showcase/exinfo.cgi?src=Data_providers_and_dates}
\color{black}
; accessed 14 November 2022), 
follow-up is complete up to:
\begin{itemize}
\item 30 September 2021 for mortality data from England and Wales
\item 31 October 2021 for mortality data from Scotland
\item 30 September 2021 for hospital data from England
\item 31 July 2021 for hospital data from Scotland
\item 28 February 2018 for hospital data from Wales
\end{itemize}

So it would make sense to censor at 28 February 2018 for Wales, 31 July 2021 for Scotland, and 30 September 2021 for England. 

With this the survival time data can be assembled, dropping those with no available date of birth and with prevalent MI at
their first assessment. Thus, follow-up starts from date of first assessment and continues until the first of a non-fatal MI, 
death, or end of follow-up. Also, any MI where death occurred within two weeks will be treated as fatal 
(and dealt with later as a fatal MI).

\subsection{Non-fatal MI} 

\color{Blue4}
\begin{stlog}\input{log/16.log.tex}\end{stlog}
\begin{stlog}\input{log/17.log.tex}\end{stlog}
\color{black}

This makes sense as a rate, and there is a reasonable amount of power.
Now by age and sex: 

\color{Blue4}
\begin{stlog}\input{log/18.log.tex}\end{stlog}
\color{black}

\begin{table}[h!]
  \begin{center}
    \caption{Crude non-fatal MI counts}
    \label{MItable}
    \pgfplotstabletypeset[
      multicolumn names,
      col sep=comma,
      string type,
      display columns/0/.style={column name=$Age$, column type={l}, text indicator="},
      display columns/1/.style={column name=$MIs$, column type={r}, column type/.add={|}{}},
      display columns/2/.style={column name=$Person-years$, column type={r}},
      display columns/3/.style={column name=$Incidence$, column type={r}},
      display columns/4/.style={column name=$MIs$, column type={r}, column type/.add={|}{}},
      display columns/5/.style={column name=$Person-years$, column type={r}},
      display columns/6/.style={column name=$Incidence$, column type={r}},
      every head row/.style={
        before row={\toprule
					& \multicolumn{3}{c}{Females} & \multicolumn{3}{c}{Males}\\
					},
        after row={ &  &  & (per 1000py) &  &  & (per 1000py) \\
        \midrule}
            },
        every last row/.style={after row=\bottomrule},
    ]{CSV/MINtable.csv}
  \end{center}
\end{table}

Okay, so there is reasonable power from ages 40-80 (table~\ref{MItable}), 
but not enough follow-up to do anything over age 80 with any confidence. 

Now the age-specific incidence of non-fatal MI can be modelled.
The method used will be the age-period-cohort (APC) model \cite{CarstensenSTATMED2007}.
What will be done here is as follows:
The data will be tabulated into 0.5-year intervals by age and year (i.e. date of follow-up).
Each unit contains the number of events and person-years of follow-up. The model is then fit on
this tabulated data, using the midpoint of each interval to represent the value of age and year
in the model. The model is a Poisson model, with spline effects of age, year, and cohort (year minus
age), using the log of person-time as the offset. Knot locations are those suggested by Frank Harrel \cite{Harrell2001Springer}.
Males and females are analysed in separate models. 
These models are then used to predict the incidence of non-fatal MI at each age (in 0.1-year increments,
for use in the model later, which will have 0.1-year increments), with the prediction year set at 2016. 

\color{Blue4}
\begin{stlog}\input{log/19.log.tex}\end{stlog}
\begin{figure}
    \centering
    \includegraphics[width=0.8\textwidth]{log/20.pdf}
    \caption{Age- and sex-specific incidence of non-fatal MI among UK Biobank participants}
    \label{MIinc}
\end{figure}
\begin{stlog}\input{log/20.log.tex}\end{stlog}
\color{black}

Realistically, it would have been good to limit prediction of rates to between 40 and 80, 
because those are the years for which there is reasonable data. However, I want to
model the lifetime risk of MI up to age 85 years, as 80 is a bit too
young to call a "lifetime risk". Admittedly, the general rule for extrapolating outside the range of your data is: don't; 
but the benefits probably outweigh the drawbacks here. It certainly won't be a problem under 40 as the very 
low rates won't really affect the model.

\subsection{Fatal MI and mortality in people without MI}

To complete the transition probabilities for the model 
age-specific mortality rates for people with and without MI will need to be estimated. 
Moreover, for people without MI, deaths will be split into MI/coronary (CHD) and non-CHD, 
as the model will assume CHD deaths are influenced by cumulative LDL, and non-CHD deaths are not. 

As a reminder, the censoring dates are:
\begin{itemize}
\item 30 September 2021 for mortality data from England and Wales
\item 31 October 2021 for mortality data from Scotland
\item 30 September 2021 for hospital data from England
\item 31 July 2021 for hospital data from Scotland
\item 28 February 2018 for hospital data from Wales
\end{itemize}

Even though there is more follow-up time for mortality data, to be sure of MI status, 
the hospital follow-up time still needs to be factored in.
So the censoring dates will be the same as those for non-fatal MI. 
This time, it's a bit more difficult for people with MI, because they need
to be censored at development of MI, and then followed up with MI to estimate
the mortality rate. 
Note that I will still exclude people with pre-existing MI at baseline, because to include them in this 
calculation would introduce selection bias (because they survived from their MI to inclusion
in UK Biobank, biasing the mortality estimates). 

First will be people without MI (again, fatal MIs are defined
by mortality within 14 days). People are followed from date of first assessment
until the first of a non-fatal MI, death (from MI/CHD or other causes), or end of follow-up: 

\color{Blue4}
\begin{stlog}\input{log/21.log.tex}\end{stlog}
\begin{stlog}\input{log/22.log.tex}\end{stlog}
\color{black}

The crude rates make sense for CHD and non-CHD, now by age and sex:

\color{Blue4}
\begin{stlog}\input{log/23.log.tex}\end{stlog}
\color{black}

\begin{table}[h!]
  \begin{center}
    \caption{Crude death counts for people without MI}
    \label{D1table}
    \pgfplotstabletypeset[
      multicolumn names,
      col sep=comma,
      header=false,
      string type,
	  display columns/0/.style={column name=$Sex$,
		assign cell content/.code={
\pgfkeyssetvalue{/pgfplots/table/@cell content}
{\multirow{11}{*}{##1}}}},
      display columns/1/.style={column name=$Age$, column type={l}, text indicator="},
      display columns/2/.style={column name=$Person-years$, column type={r}, column type/.add={|}{|}},
      display columns/3/.style={column name=$Deaths$, column type={r}},
      display columns/4/.style={column name=$Mortality$, column type={r}, column type/.add={}{|}},
      display columns/5/.style={column name=$Deaths$, column type={r}},
      display columns/6/.style={column name=$Mortality$, column type={r}, column type/.add={}{|}},
      every head row/.style={
        before row={\toprule
					& & & \multicolumn{2}{c}{CHD} & \multicolumn{2}{c}{Non-CHD} \\
					},
        after row={ & & & & (per 1000py) & & (per 1000py) \\
        \midrule}
            },
        every nth row={11}{before row=\midrule},
        every last row/.style={after row=\bottomrule},
    ]{CSV/nCVDdeathNtable.csv}
  \end{center}
\end{table}

Again, APC models are used (methods exactly the same as for non-fatal MI above), fitting a separate model for each outcome and sex.  

\color{Blue4}
\begin{stlog}\input{log/24.log.tex}\end{stlog}
\begin{figure}
    \centering
    \includegraphics[width=0.8\textwidth]{log/25.pdf}
    \caption{Age-, sex-, and cause-specific mortality among UK Biobank participants without CVD}
    \label{NOCVDmort}
\end{figure}
\begin{stlog}\input{log/25.log.tex}\end{stlog}
\color{black}

\subsection{Mortality following MI}

Now for mortality in the other health state, ``CHD'' (i.e., pre-existing MI). 
Here, people are followed from date of first MI (plus 14 days as to not introduce 
immortal-time bias, given how fatal MI is defined) until death or end of follow-up:

\color{Blue4}
\begin{stlog}\input{log/26.log.tex}\end{stlog}
\begin{stlog}\input{log/27.log.tex}\end{stlog}
\begin{stlog}\input{log/28.log.tex}\end{stlog}
\color{black}

\begin{table}[h!]
  \begin{center}
    \caption{Crude death counts for people with MI, by age and sex}
    \label{D2table}
    \pgfplotstabletypeset[
      multicolumn names,
      col sep=comma,
      header=false,
      string type,
	  display columns/0/.style={column name=$Cohort$,
		assign cell content/.code={
\pgfkeyssetvalue{/pgfplots/table/@cell content}
{\multirow{10}{*}{##1}}}},
      display columns/1/.style={column name=$Age$, column type={l}, text indicator="},
      display columns/2/.style={column name=$Deaths$, column type={r}, column type/.add={|}{}},
      display columns/3/.style={column name=$Person-years$, column type={r}},
      display columns/4/.style={column name=$Mortality$, column type={r}, column type/.add={}{|}},
      display columns/5/.style={column name=$Deaths$, column type={r}},
      display columns/6/.style={column name=$Person-years$, column type={r}},
      display columns/7/.style={column name=$Mortality$, column type={r}, column type/.add={}{|}},
      display columns/8/.style={column name=$Deaths$, column type={r}},
      display columns/9/.style={column name=$Person-years$, column type={r}},
      display columns/10/.style={column name=$Mortality$, column type={r}, column type/.add={}{|}},
      every head row/.style={
        before row={\toprule
					& & \multicolumn{3}{c}{Females} & \multicolumn{3}{c}{Males}\\
					},
        after row={ & & & & (per 1000py) & & & (per 1000py) \\
        \midrule}
            },
        every nth row={10}{before row=\midrule},
        every last row/.style={after row=\bottomrule},
    ]{CSV/PCVdeathtable.csv}
  \end{center}
\end{table}

\begin{table}[h!]
  \begin{center}
    \caption{Crude death counts for people with MI, by time since event and sex}
    \label{D3table}
     \fontsize{9pt}{11pt}\selectfont\pgfplotstabletypeset[
      multicolumn names,
      col sep=comma,
      header=false,
      string type,
	  display columns/0/.style={column name=$Cohort$,
		assign cell content/.code={
\pgfkeyssetvalue{/pgfplots/table/@cell content}
{\multirow{6}{*}{##1}}}},
      display columns/1/.style={column name=$Time-since-event$, column type={l}, text indicator="},
      display columns/2/.style={column name=$Deaths$, column type={r}, column type/.add={|}{}},
      display columns/3/.style={column name=$Person-years$, column type={r}},
      display columns/4/.style={column name=$Mortality$, column type={r}, column type/.add={}{|}},
      display columns/5/.style={column name=$Deaths$, column type={r}},
      display columns/6/.style={column name=$Person-years$, column type={r}},
      display columns/7/.style={column name=$Mortality$, column type={r}, column type/.add={}{|}},
      display columns/8/.style={column name=$Deaths$, column type={r}},
      display columns/9/.style={column name=$Person-years$, column type={r}},
      display columns/10/.style={column name=$Mortality$, column type={r}, column type/.add={}{|}},
      every head row/.style={
        before row={\toprule
					& & \multicolumn{3}{c}{Females} & \multicolumn{3}{c}{Males}\\
					},
        after row={ & & & & (per 1000py) & & & (per 1000py) \\
        \midrule}
            },
        every nth row={6}{before row=\midrule},
        every last row/.style={after row=\bottomrule},
    ]{CSV/PCVdeathtabledur.csv}
  \end{center}
\end{table}

\clearpage

The mortality rates are much higher than for people without MI, as expected.
Mortality is also very high immediately after the event (i.e., is influenced by
time since the event) (table~\ref{D3table}). This can be factored in by including a time-since-event term in the Poisson model
used to estimate mortality rates post-MI. 
Again, the methods will be drawn from Bendix Carstensen \cite{CarstensenBMJO2020}. The methods are similar 
to those described above. In this analysis, data are tabulated into 0.5-year intervals by age, year (i.e. date of follow-up),
and duration (i.e. time since the MI). Each unit contains the number of events and person-years of follow-up. 
The model is then fit on this tabulated data, using the midpoint of each interval to represent the value of age, year, and duration
in the model. The model is a Poisson model, with spline effects of age, duration, and age at MI (age minus duration), a log-linear effect
of time (year), using the log of person-time as the offset. 
Knot locations are those suggested by Frank Harrel \cite{Harrell2001Springer}, except for duration, because
the majority of deaths occur very early on, so I had to specify knots specific to this data to avoid duplicated knot locations.
Again, males and females are analysed in separate models. 
These models are then used to predict the incidence of non-fatal MI at each age (in 0.1-year increments,
for use in the model later, which will have 0.1-year increments) and time-since MI (i.e., a much larger prediction matrix than before)
, with the prediction year again set at 2016. 

\color{Blue4}
\begin{stlog}\input{log/29.log.tex}\end{stlog}
\begin{figure}
    \centering
    \includegraphics[width=0.8\textwidth]{log/30.pdf}
    \caption{Age-, sex-, and time-since-MI-specific mortality among UK Biobank participants with MI}
    \label{PMImort}
\end{figure}
\begin{stlog}\input{log/30.log.tex}\end{stlog}
\color{black}

As expected, uncertainty around these mortality rates is very high (figure~\ref{PMImort}), 
especially a few years after the event (for which there is little data).
This is probably not going to be a major issue, given the interest in this study is primary prevention. 

Finally, note these are just examples in figure~\ref{PMImort} --
a mortality rate has been predicted for every sex/age/time-since-event possible. 

\clearpage
\pagebreak
\section{LDL-C trajectories}
\label{LDLtrajsection}

\subsection{Mean LDL-C in UK Biobank}

At this point, the relevant transition probabilities for the model have been estimated. 
However, they are currently independent of LDL-C, so this will need to be incorporated into the non-fatal MI
and fatal MI/CHD death transition probabilities. 
For this study, the effect estimates of LDL-C on CHD will be derived from MR, 
as this gives a causal estimate of the effect. Specifically, the results of 
Ference et al. \cite{FerenceJAMA2019} will be used, 
which showed that for every 1 mmol/L reduction in LDL-C over the lifetime, 
the odds ratio for CHD was 0.46 (to be converted into a relative risk later on, see section~\ref{LDLAMI}). 
This number applies to the LDL-C over the lifetime, not instantaneous LDL-C at a given age. 
Thus, the model will require a projection of the lifetime LDL-C for everyone in the sample, which can be used
to estimate the cumulative LDL-C at any given age or each individual, 
which in turn is compared to the mean mean [two means intentional] cumulative 
LDL-C of the entire sample (i.e., first calculate the mean cumulative LDL-C for each individual at a given age, and
then estimate the mean of this value, for the ``mean mean cumulative LDL-C'') to calculate the reduction in MI risk. 
However, UK Biobank only has LDL-C at a single point in time (for the majority), and only assessed medication use at 
enrolment, so assumptions will need to be made to estimate cumulative LDL-C. These assumptions are as follows:

\begin{itemize}
\item LDL-C is constant from age 40 onwards. This is supported by some literature 
for people who don't take LLT \cite{DuncanJAHA2019}. 
\item Mean LDL-C is 0.75 mmol/L  at birth \cite{DescampsAth2004}, 
increases linearly to a mean of 2 mmol/L by age 5
(assumption based on \cite{KitJAMA2012}), and after this increases linearly to whatever value
the individual has recorded in UK Biobank by age 40. 
\item Where an individual sits on the LDL-C distribution is constant throughout life (i.e., someone in the 
5th percentile of LDL-C will be in that percentile for life). 
\item People receiving LLT at baseline initiated therapy 5 years before their date of first assessment. 
Given how low LLT persistence is, \cite{TothLHD2019,TalicCDT2021,OforiJOG2017} 
this is probably a reasonably conservative assumption.
\item People persist on LLT forever once they start LLT. Now this is really unrealistic, 
but it's also really conservative, so it is probably suitable. 
\item People not on LLT at baseline initiate LLT at an average rate of approximately 10 people
per 1,000 person-years. This estimate is taken from O'Keefe et al. \cite{OKeefeCLINEPI2016}, which also showed that LLT initiation
is highly dependent on age, so the LLT initiation rate will be:
\begin{itemize}
\item 1 per 1,000 person-years for people aged 40-49
\item 15 per 1,000 person-years for people aged 50-59
\item 35 per 1,000 person-years for people aged 60 and above
\end{itemize}
These numbers are all interpreted from Figure 1C in the O'Keefe et al. paper. Moreover, it is reasonable to assume that
LDL-C would affect probability of LLT initation, as would sex. The O'Keefe paper suggests that males are approximately 10-20\%
more likely to initiate LLT; thus, males will be 10\% more likely to initiate LLT than females. 
As for LDL-C, it's very unlikely robust data linking LDL-C, age, and sex to LLT initiation exists; 
thus, I will simply assume that for every standard deviation above the mean LDL-C that an individual is, 
they become 3 times more likely to initiate LLT (this may or may not reflect real clinical practice, 
which uses LDL-C cut-offs and risk calculators but is a very conserative approach in the absence of 
more data). 
\item LLT lowers LDL-C by 30\%. This is an assumption based on real-world studies of statin 
effectiveness \cite{FangLHD2021,BacquerEJPC2020}.
\end{itemize}

Note that these assumptions are as generous as possible, to lower the LDL-C as much as possible for 
the sample (and thus, the control condition, which will come into play later). In other words, 
I have taken a very conservative approach so that the effect of the interventions is minimised. 

Once LDL-C trajectories have been calculated for each person, these can then be used to
model age- and sex-specific mean LDL-C, and in turn use those values to calculate LDL-C adjusted MI incidence (in the next section). 
Interventions will be modelled later on. 

First, LDL-C trajectories for the UK Biobank cohort are estimated from age 0 to end of follow-up. 
To visualize the process, just 20 observations are kept at first, and 
I will then calculate trajectories for everyone. 

\color{Blue4}
\begin{figure}
    \centering
    \includegraphics[width=0.8\textwidth]{log/31.pdf}
    \caption{Uncorrected LDL-C trajectories in 20 UK Biobank participants}
    \label{LDL201}
\end{figure}
\begin{stlog}\input{log/31.log.tex}\end{stlog}
\color{black}

At this point (figure~\ref{LDL201}), LDL-C is just constant over the lifetime until LLT initiation
, but it needs to be adjusted at younger ages.

\color{Blue4}
\begin{figure}
    \centering
    \includegraphics[width=0.8\textwidth]{log/32.pdf}
    \caption{LDL-C trajectories in 20 UK Biobank participants}
    \label{LDLage20}
\end{figure}
\begin{stlog}\input{log/32.log.tex}\end{stlog}
\color{black}

This is very clunky (figure~\ref{LDLage20}), but better than assuming a constant lifetime LDL-C.
Moreover, the model ultimately only uses mean cumulative LDL-C, so that will smooth the 
results. 

\color{Blue4}
\begin{figure}
    \centering
    \includegraphics[width=0.8\textwidth]{log/33.pdf}
    \caption{Cumulative LDL-C trajectories in 20 UK Biobank participants}
    \label{cumLDLage20}
\end{figure}
\begin{figure}
    \centering
    \includegraphics[width=0.8\textwidth]{log/33_1.pdf}
    \caption{Mean cumulative LDL-C trajectories in 20 UK Biobank participants}
    \label{aveLDLage20}
\end{figure}
\begin{stlog}\input{log/33.log.tex}\end{stlog}
\color{black}

These look okay (figures~\ref{cumLDLage20} \& \ref{aveLDLage20}).
Now this process can be repeated for the entire sample. 

\color{Blue4}
\begin{stlog}\input{log/34.log.tex}\end{stlog}
\begin{figure}
    \centering
    \includegraphics[width=0.8\textwidth]{log/35.pdf}
    \caption{Mean cumulative LDL-C by sex}
    \label{cummeanLDL}
\end{figure}
\begin{stlog}\input{log/35.log.tex}\end{stlog}
\color{black}

This is pretty nasty after age ~80 (figure~\ref{cummeanLDL}), so to fix that:

\color{Blue4}
\begin{figure}
    \centering
    \includegraphics[width=0.8\textwidth]{log/36.pdf}
    \caption{Mean cumulative LDL-C by sex, modelled}
    \label{cummeanLDLmod}
\end{figure}
\begin{stlog}\input{log/36.log.tex}\end{stlog}
\color{black}

Much nicer (figure~\ref{cummeanLDLmod}). To re-iterate, what I have calculated so far is the mean mean cumulative LDL-C estimates for the
UK Biobank population, which is not the same as the standard of care/control scenario. That is done below. 
Before that, it is worth using this process to work out the proportion of people in the 
primary prevention population on LLT in the control scenario 
at ages 40, 50, 60, 70, and 80 (Table~\ref{statinproptab}).

\begin{table}[h!]
  \begin{center}
    \caption{Proportion of primary prevention population in the control scenario on LLT 
at or before a given age, stratified by sex and LDL-C.}
    \label{statinproptab}
     \selectfont\pgfplotstabletypeset[
      multicolumn names,
      col sep=colon,
      header=false,
      string type,
	  display columns/0/.style={column name=Sex,
		assign cell content/.code={
\pgfkeyssetvalue{/pgfplots/table/@cell content}
{\multirow{4}{*}{##1}}}},
      display columns/1/.style={column name=Age, column type={l}, text indicator=", column type/.add={}{|}},
      display columns/2/.style={column name=Overall, column type={r}},
      display columns/3/.style={column name=$\geq$3.0, column type={r}},
      display columns/4/.style={column name=$\geq$4.0, column type={r}},
      display columns/5/.style={column name=$\geq$5.0, column type={r}},
      every head row/.style={
        before row={\toprule
					& & \multicolumn{4}{c}{LDL-C (mmol/L)}\\
					},
        after row={\midrule}
            },
        every nth row={5}{before row=\midrule},
        every last row/.style={after row=\bottomrule},
    ]{CSV/statinprop.csv}
  \end{center}
\end{table}

\color{Blue4}
\begin{stlog}\input{log/37.log.tex}\end{stlog}
\color{black}

\subsection{Interventions}

Indeed, I now need to simulate LDL-C trajectories for our full cohort over the lifetime, and under several conditions (i.e., drugs).
The conditions, and their respective LDL-C reductions, are as follows:
\begin{enumerate}
\setcounter{enumi}{-1}
\item Standard of care/control. This is similar to that described above. In short, this arm is current standard of care, 
where: people receiving LLT at baseline are assumed to have initiated therapy 5 years before their date of first assessment;
people persist on LLT forever once they start LLT; and
people not on LLT at baseline initiate LLT at an average rate of approximately 10 people per 1,000 person-years,
and this rate is affected by age, sex, and LDL-C. Again, all these assumptions are very generous to take the 
most conservative approach and minimise the effectiveness of the other interventions. 
The only difference between this control scenario with the estimation of average LDL-C during follow-up described above 
is that LLT will lower LDL-C by 45\% (a generous (and therefore conservative) assumption based on the effects of low/moderate intensity statins, 
high intensity statins, and low/moderate intensity statins, outlined in the following dot points), not 30\% as above.
\item Low/moderate intensity statins. The most common statin in this category (as defined by the ACC/AHA \cite{StoneJACC2014}) 
in the UK in June 2021 was Atorvastatin (20mg and 10mg; see below), and the best available evidence for the effect of atorvastatin on 
LDL-C comes from a systematic review, which estimated a 42.3\% (95\%CI: 42.0, 42.6) reduction at 20mg, and 37.1\% (36.9, 37.3) for 10mg
\cite{AdamsCDSR2015}. 
The next most common was Simvastatin (40mg and 20mg), which reduced LDL-C by 35\% over follow up in the  LDL-C 
Scandinavian Simvastatin Survival Study \cite{4SLancet1994}. Thus, a reasonable estimate for LDL-C reduction on low/moderate intensity statins
would be 40\%, with a 95\% CI of 39 to 41 (which happens to be the exact result from the 
Collaborative Atorvastatin Diabetes Study (CARDS) trial, the major primary prevention trial with atorvastatin \cite{ColhounLancet2004}). 
\item High intensity statins. Again, the most common in this category is Atorvastatin (40mg and 80mg), and the aforementioned
systematic review puts the LDL-C reduction with 40mg of Atorvastatin at 47.4\% (46.9, 48.0), and 51.7\% (51.2, 52.2) for 80mg 
\cite{AdamsCDSR2015}. 
Thus, 50\% (49, 51) would be a reasonable estimate for LDL-C reduction with high-intensity statins. 
\item Low/moderate intensity statins and ezetimibe. The best evidence for the effect of ezetimibe added to statins is 
taken from a systematic review \cite{AmbeAth2014}, which estimates that adding ezetimibe to ongoing statin reduced LDL-C by 26.0\% 
(25.2, 26.8), while switching to high-intensity Rosuvastatin 10mg reduced LDL-C by 19.7\% (17.7, 21.7). This suggests
that a reasonable estimate of the effect of low/moderate intensity statins on LDL-C (compared to nothing at all) would be 55\% 
(54, 56). 
\item Inclisiran. The LDL-C reduction for people taking Inclisiran is 51.5\% (95\% CI, 49.0 to 53.9), 
based on a weighted average of the results (time-weighted average reduction in LDL-C) 
of the ORION-10 and ORION-11 trials \cite{KausikNEJM2020},
which were selected as the most relevant evidence (phase III trials in a population where ~99\% of the people did
not have familial hypercholesterolaemia). 
\end{enumerate}

For all interventions (i.e., everything above except the control), they are implemented at ages 30, 40, 50, and 60 years
(i.e., 4 different strategies per intervention), and it is assumed that LLT initiation is as in the control arm until 
the age of the intervention, at which everyone gets the intervention. 

This is how the current Statin use in the UK was assessed: 


\color{Blue4}
\begin{stlog}\input{log/38.log.tex}\end{stlog}
\begin{stlog}\input{log/39.log.tex}\end{stlog}
\begin{stlog}\input{log/40.log.tex}\end{stlog}
\color{black}

Back to calculation of LDL-C trajectories. 
For these calculations, mortality is not taken into account yet, as that will be part of the model. 
So, even if an individual lacks follow-up, I can still estimate an LDL-C trajectory across their
entire lifespan.

Just to visualise it, the interventions will just be applied to two people at first
, one with a very low LDL-C, the other very high.

\color{Blue4}
\begin{figure}
    \centering
    \includegraphics[width=0.8\textwidth]{log/41.pdf}
    \caption{LDL-C with various interventions for 2 individuals}
    \label{LDLwithint}
\end{figure}
\begin{stlog}\input{log/41.log.tex}\end{stlog}
\color{black}
Some things to note here (figure~\ref{LDLwithint}):
There's a large difference in 
absolute LDL-C reduction by baseline LDL-C (as expected/by definition). 
Also, the control condition can result in a lower LDL-C at the end of life than 
low/moderate intensity statins. This is good -- this analysis is seeking
to answer the question as to whether it's more cost-effective to intervene to 
lower LDL-C earlier in life, 
so comparing a strategy that usually results in intense LDL-C lowering later in life
(the control) with a less intense strategy earlier in life is interesting. 
\color{Blue4}
\begin{figure}
    \centering
    \includegraphics[width=0.8\textwidth]{log/42.pdf}
    \caption{Cumulative LDL-C with various interventions for 2 individuals}
    \label{cumLDLwithint}
\end{figure}
\begin{figure}
    \centering
    \includegraphics[width=0.8\textwidth]{log/42_1.pdf}
    \caption{Mean cumulative LDL-C with various interventions for 2 individuals}
    \label{aveLDLwithint}
\end{figure}
\begin{stlog}\input{log/42.log.tex}\end{stlog}
\color{black}
Now it can be repeated for everyone under all 17 scenarios (16 interventions and 1 control):
\color{Blue4}
\begin{stlog}\input{log/43.log.tex}\end{stlog}
\color{black}

\clearpage
\pagebreak
\section{LDL-C--adjusted MI incidence}
\label{LDLAMI}

Now that LDL-C trajectories over the lifetime for everyone in the model population under all scenarios 
have been estimated, the incidence of MI at a given age, sex, and mean cumulative LDL-C can be estimated.

As mentioned above, the effect of LDL-C on CHD is summarised with the
odds ratio for CHD of 0.46 per mmol/L reduction in (lifetime) LDL-C \cite{FerenceJAMA2019}.
This needs to be converted into a relative risk before use in the model. 
This is done using the formula: 

\begin{quote}
\begin{math} 
RR = \frac{OR}{(1-P_0)+(P_0 \times OR)}
\end{math}
\end{quote}

where \begin{math} OR \end{math} is the odds ratio, 
\begin{math} RR \end{math} the relative risk, and 
\begin{math} P_0 \end{math} the risk of the outcome
in the unexposed group \cite{ZhangJAMA1998}. 

This yields a RR of:

\begin{quote}
\begin{math} 
\frac{0.46}{(1-0.064)+(0.064 \times 0.46)} = 0.48
\end{math}
\end{quote}

Which is the number that will be applied to estimate the relative risk for MI for the cohort using the following equation:

\begin{quote}
\begin{math} 
R_a = R \times 0.48^{(LDL_\mu-LDL_\tau)}
\end{math}
\end{quote}

where \begin{math} R_a \end{math} is the adjusted age-specific rate, 
\begin{math} R \end{math} the original age-specific rate (estimated in section~\ref{TPs}),
\begin{math} LDL_\mu \end{math} the mean cumulative LDL-C for UK Biobank 
sample at that given age (estimated in section~\ref{LDLtrajsection})
, and \begin{math} LDL_\tau \end{math} the mean cumulative LDL-C for 
the specific UK Biobank participant at that given age (estimated in section~\ref{LDLtrajsection}).

Note how this is the primary way disease biology is incorporated into the model--
the model uses cumulative LDL-C, not instantaneous LDL-C, to estimate the incidence of MI. 

To visualise the process, again just two people are used -- one with a very low LDL-C, the other very high
-- and just the high-intensity statin arm is displayed.

\color{Blue4}
\begin{figure}
    \centering
    \includegraphics[width=0.8\textwidth]{log/44.pdf}
    \caption{Incidence of non-fatal MI for 2 individuals by age of intervention}
    \label{nfMIincLDL1}
\end{figure}
\begin{figure}
    \centering
    \includegraphics[width=0.8\textwidth]{log/44_1.pdf}
    \caption{Incidence of fatal MI for 2 individuals by age of intervention}
    \label{fMIincLDL1}
\end{figure}
\begin{stlog}\input{log/44.log.tex}\end{stlog}
\color{black}
A couple of things to note. First, LDL-C influences risk of MI (hardly groundbreaking science); 
second, the benefit of LDL-C reduction depends heavily on timing of the intervention and baseline LDL-C. 
What about the difference between interventions? For this, let's just use the person with high LDL-C and
look at Low/moderate intensity vs. high-intensity statins.
\color{Blue4}
\begin{figure}
    \centering
    \includegraphics[width=0.8\textwidth]{log/45.pdf}
    \caption{Incidence of non-fatal MI for 1 individual by intervention}
    \label{nfMIincLDL2}
\end{figure}
\begin{figure}
    \centering
    \includegraphics[width=0.8\textwidth]{log/45_1.pdf}
    \caption{Incidence of fatal MI for 1 individual by intervention}
    \label{fMIincLDL2}
\end{figure}
\begin{stlog}\input{log/45.log.tex}\end{stlog}
\color{black}
Repeat the process for everyone: 
\color{Blue4}
\begin{stlog}\input{log/46.log.tex}\end{stlog}
\color{black}

\clearpage
\pagebreak
\section{Microsimulation model}

At this point, everything required to construct the model has been estimated.

Recall the model structure (figure~\ref{Schematic1}).

\begin{figure}
    \centering
    \includegraphics[width=0.8\textwidth]{Model schematic.pdf}
    \caption{Model structure. Dashed lines are transition probabilities influenced by mean cumulative LDL-C; solid lines are transition probabilities not influenced by LDL-C.}
    \label{Schematic1}
\end{figure}

So far, the following have been estimated:
\begin{itemize}
\item Age- and sex-specific incidence of MI (fatal and non-fatal), 
adjusted for an individual's LDL-C trajectory over their lifetime
under different scenarios.
\item Age- and sex- non-CHD mortality for people without MI. 
\item Age-, sex-, and time-since-MI-specific mortality for people with MI. 
\end{itemize}

These can be used to simulate the UK Biobank cohort 
over their lifetime in a microsimulation model. 
The model starts at age 30 and runs to age 85, ageing in 0.1-year increments. 
All individuals start free of MI, and in each cycle are at risk for non-fatal MI, 
fatal MI/CHD death, and non-CHD death. Once a non-fatal MI has occurred, individuals are at risk for death. 
\emph{Repeat events are not tracked.}
For health economic analyses, the costs of repeat events will be assumed to be 
captured in the ongoing cost of managing MI. 

\subsection{First cycle in detail}

So, here I will first build the overall structure of the model.
Health economic outcomes will be added later on. 
The first cycle is explained in detail:

\color{Blue4}
\begin{stlog}\input{log/47.log.tex}\end{stlog}
\begin{stlog}\input{log/48.log.tex}\end{stlog}
\begin{stlog}\input{log/49.log.tex}\end{stlog}
\color{black}

This is the starting structure of the model. (Note 
the \emph{eid} hasn't been listed for privacy reasons, but it is there,
and will be used for individual LDL-C trajectories/MI risk). Next
transition probabilities are merged in (well, rates are, then converted to 
transition probabilities):

\color{Blue4}
\begin{stlog}\input{log/50.log.tex}\end{stlog}
\begin{stlog}\input{log/51.log.tex}\end{stlog}
\color{black}

Note how the rate of MI is different for each person, 
but the non-CHD mortality is the same by sex.
Next, people are transitioned between states:

\color{Blue4}
\begin{stlog}\input{log/52.log.tex}\end{stlog}
\color{black}

As expected, for people aged 30 the probability of 
an event or death is extremely low, so we don't see 
many transitions here. Just to show the mechanics of the model, the 
transition probabilities can be made higher:

\color{Blue4}
\begin{stlog}\input{log/53.log.tex}\end{stlog}
\color{black}

Also note the variable \emph{MI} is used to track
both non-fatal MI and coronary death. 
After events occur, the cohort is aged one cycle:

\color{Blue4}
\begin{stlog}\input{log/54.log.tex}\end{stlog}
\begin{stlog}\input{log/55.log.tex}\end{stlog}
\color{black}

This can be repeated to age 85.
The only difference between these and the first cycle 
is that the population with MI is also aged,
but the principles are the same.
The populations are also saved at ages 40, 50, and 60 years
for use in the interventions later. 

\subsection{Full model}

Notice how seeds are set, this ensures the conditions for each simulation are 
exactly the same and thus, people serve directly as their own control for each intervention. 

Additionally, merging on numeric variables with decimals is not a good idea, because
they cannot be represented in binary, and so aren't always stored the same across datasets. 
Thus, I will first convert all ages/durations into integers to run the model.

\color{Blue4}
\begin{stlog}\input{log/56.log.tex}\end{stlog}
\begin{stlog}\input{log/57.log.tex}\end{stlog}
\begin{stlog}\input{log/58.log.tex}\end{stlog}
\color{black}

Once run, it is also seen that all the information required
is saved at the end of the run (so events don't need to be tracked throughout)
-- because the model stops cycling people at death, the final dataset indicates 
the age people died, when and if they had an MI, and at what age. 
I.e., all the things necessary to track utilities and costs. 

\subsection{LDL-C adjustment check}

Before going further, it would be prudent to check that the method of adjusting 
CHD risk via LDL-C has not biased our results dramatically. This can be done
by comparing the results of the control simulation to one using the original
MI rates (i.e., those unadjusted for LDL-C):

\color{Blue4}
\begin{stlog}\input{log/59.log.tex}\end{stlog}
\begin{stlog}\input{log/60.log.tex}\end{stlog}
\begin{figure}
    \centering
    \includegraphics[width=0.8\textwidth]{log/61.pdf}
    \caption{Age of MI or coronary death by method of estimating rates}
    \label{ageMIcheck}
\end{figure}
\begin{figure}
    \centering
    \includegraphics[width=0.8\textwidth]{log/61_1.pdf}
    \caption{Cumulative incidence of MI or coronary death by method of estimating rates}
    \label{cumMIcheck}
\end{figure}
\begin{stlog}\input{log/61.log.tex}\end{stlog}
\color{black}

So the adjustment makes little difference, 
which means there is reassurance that the method
of adjusting rates is not \emph{very} wrong. 
It's also worth pointing out at this point that the lifetime risk 
of MI/coronary death under the control scenario is 
relatively consistent with the available literature,
although estimates of lifetime risk vary considerably depending on definitions 
(many were all CHD, not just MI and coronary death)
and country, but the estimate in the present study is higher than some 
\cite{RapsomanikiLancet2014,TurinCirc2010}
considerably lower than others \cite{BerryNEJM2012,LeeningBMJ2014},
and similar to one \cite{StenlingAth2020}. This is a bit surprising given
that UK Biobank is known to have an important "health volunteer" bias \cite{FryAJE2017}.
Thus, it is reassuring to know that the lifetime risk of MI 
does not appear to be massively under or over-estimated.  

\subsection{Interventions}

Now to simulate the interventions:

\color{Blue4}
\begin{stlog}\input{log/62.log.tex}\end{stlog}
\color{black}

\subsection{Lifetime risk of MI}

And to plot the lifetime risk of MI under all conditions:

\color{Blue4}
\begin{stlog}\input{log/63.log.tex}\end{stlog}
\begin{figure}
    \centering
    \includegraphics[width=0.8\textwidth]{log/64.pdf}
    \caption{Cumulative incidence of MI or coronary death by intervention}
    \label{cumMIint}
\end{figure}
\begin{stlog}\input{log/64.log.tex}\end{stlog}
\color{black}

It can be seen from figure~\ref{cumMIint} that earlier and more aggressive lowering of LDL-C is
clearly a better strategy for the prevention of MI and coronary death. 

\subsection{Utilities and costs}

The question now is whether the extra decades of treatment are justified
from a cost-effectiveness perspective. To assess this, the
incremental cost-effectiveness ratio (ICER; from the UK healthcare perspective) will be calculated, 
which requires calculating the incremental quality-adjusted life years (QALYs)
and incremental healthcare costs. To do this, the following inputs are used:

\begin{itemize}
\item Utility values for people without MI in the UK, set using the following equation: 
\begin{math} 0.9454933 + 0.0256466*male - 0.0002213*age - 0.0000294*age^2 \end{math}
, derived from Ara and Brazier \cite{ARAVIH2010}, 
which appears to be a standard for health economic analyses in the UK. 
\item Utility values for people with MI, set at 0.79 (95\%CI 0.73, 0.85), 
derived from a systematic review of utility values in CVD \cite{BettsHQLO2020}. 
People with MI also incurred an acute disutility associated with the initial MI, which was set at 0.12 \cite{LewisJACCHF2014}
and applied for 3 months, for a final acute disutility of 0.03 per year in the year the event occured in. 
\item Cost of acute MI in the UK, set at \textsterling 2047.31. This is derived directly from the National Health
Service Cost Schedule for 2020/21 (\cite{NHSCOST202021}). 
See Table~\ref{ACUTEMICOST} for details. 
\item Excess chronic costs of managing MI (including subsequent events) in the UK. This is set at 
\textsterling 4705.45 (Standard error (SE): 112.71) for the first 6 months, and \textsterling 1015.21 
(SE: 171.23) per year thereafter (values adjusted from 2014 \textsterling to 2021 \textsterling using the NHS cost inflation index 
(from: \cite{NHSinflation2021}, which was a mean of 1.38\% from 2014-2021; 
thus the conversion formula was \begin{math} cost_{2014} \times 1.0138^7 \end{math}). 
These values were derived from a cohort study using
the Clinical Practice Research Datalink in the UK \cite{DaneseeBMJO2016} 
(the original unadjusted values: 4275.41 (SE: 102.41) and \textsterling 922.43 (SE: 155.58)). 
\item The average annual cost of statins in the UK, for the control arm. This is set at \textsterling 19.00, derived directly from 
the NHS Drug Tariff data \cite{NHSDrugTariff}, with the distribution of statin use derived from the English Prescribing Dataset in
June 2021 \cite{NHSEPD} (see below for details on how this was calculated).
\item The cost of low/moderate intensity statins. As above, the most common in this class was Atorvastatin 20mg, which also
happens to be the most expensive in the NHS drug tariff (out of Atorvastatin 10mg and 20mg, and Simvastatin 20mg and 40mg),
and is thus the most conservative, at a price of \textsterling 1.41 per 28 tablets, which leads to an annual cost of 
\begin{math} \frac{1.41}{28} \times 365.25 = \end{math} \textsterling 18.39
\item The cost of high intensity statins. As above, the most common in this class was Atorvastatin 40mg, but
to be conservative, it's best to use the price of Atorvastatin 80mg, which has a packet cost of \textsterling 2.10, leading
to an annual cost of \begin{math} \frac{2.10}{28} \times 365.25 = \end{math} \textsterling 27.39
\item The cost of low/moderate intensity statins and ezetimibe. The cost of a packet of 28 ezetimibe (10mg) tablets
is \textsterling 2.37, leading to an annual cost of 
\begin{math} \frac{2.37}{28} \times 365.25 = \end{math} \textsterling 30.92, which we can add to the annual cost of
low-moderate intensity statins used above for an annual cost of 
\begin{math} 18.39+30.92 = \end{math} \textsterling 49.31
\item The cost of Inclisiran. This wasn't available on the drug tariff in June 2021, but is available under a special agreement 
at a price of \textsterling 1987.36 per injection \cite{NHSDMDInclisiran2022}, leading to an annual cost of 
\begin{math} 1987.36 \times 2 = \end{math}  \textsterling 3,974.72.
This is very high, probably because Inclisiran is currently
only indicated for very high-risk people. Thus, the price at which it is cost-effective for primary prevention is probably going
to be considerably lower than in these high-risk populations. As such, later on, I will run threshold analyses later to determine
the maximum cost at which Inclisiran is cost-effective in the primary prevention setting (with considerable confidence in my
assumption that it won't be cost-effective at this price). 
\end{itemize}

The ICERs calculated here will be compared to the The U.K. National Institute for Health and Care Excellence (NICE) 
willingness-to-pay threshold, which is a range -- \textsterling 20,000 to  \textsterling 30,000
per QALY \cite{NICEHTA2013}. 

These cost-effectiveness analyses will also require the following assumptions:
\begin{itemize}
\item 18\% of all fatal MI's occur in hospital (and therefore accrue 18\% of the cost
of a non-fatal MI; see below for details).
\item The interventions continue after an MI, and add to the ongoing costs of management.
\item Similarly, statin use is initiated after an MI for everyone in the control arm, adding to the ongoing cost of management. 
\item Discounting starts from the age of intervention (meaning the control scenario
QALYs and costs will vary depending on the intervention). The discounting rate in the primary analysis
will be 3.5\%, as recommended by NICE \cite{NICEHTA2013}.
\end{itemize}

\begin{table}[h!]
  \begin{center}
    \caption{Calculation of acute MI costs}
    \label{ACUTEMICOST}
     \fontsize{7pt}{9pt}\selectfont\pgfplotstabletypeset[
      col sep=comma,
      header=false,
      string type,
      display columns/0/.style={column name=Code, column type={l}, text indicator="},
      display columns/1/.style={column name=Description, column type={l}},
      display columns/2/.style={column name=Number, column type={r}},
      display columns/3/.style={column name=Unit cost (\textsterling), column type={r}},
	  display columns/4/.style={column name=Weighted mean (\textsterling),
		assign cell content/.code={
\pgfkeyssetvalue{/pgfplots/table/@cell content}
{\multirow{5}{*}{##1}}}},
      every head row/.style={
        before row={\toprule},
        after row={\midrule}
            },
        every last row/.style={after row=\bottomrule},
    ]{ACUTEMICOST.csv}
  \end{center}
\end{table}

How annual statin cost was arrived at (the source of the 18\% follows this):

\color{Blue4}
\begin{stlog}\input{log/65.log.tex}\end{stlog}
\begin{stlog}\input{log/66.log.tex}\end{stlog}
\begin{stlog}\input{log/67.log.tex}\end{stlog}
\begin{stlog}\input{log/68.log.tex}\end{stlog}
\color{black}

The source of the 18\%:

\color{Blue4}
\begin{stlog}\input{log/69.log.tex}\end{stlog}
\begin{stlog}\input{log/70.log.tex}\end{stlog}
\color{black}

With all this, the matrices/datasets for utilities and costs can be created:

\color{Blue4}
\begin{stlog}\input{log/71.log.tex}\end{stlog}
\begin{stlog}\input{log/72.log.tex}\end{stlog}
\color{black}

This is a matrix that has cumulative YLL a a given age, 
depending on the age the intervention is started,
ready to merge into the microsimulation model results. 
It can already be seen how impactful discounting will be
on our results -- someone only accrues 24.7 YLL
if they survive from 30 to 85. 

Let's include utilities to make these QALYs:

\color{Blue4}
\begin{figure}
    \centering
    \includegraphics[width=0.8\textwidth]{log/73.pdf}
    \caption{Age and sex-specific utility for people without MI}
    \label{AgeUT}
\end{figure}
\begin{stlog}\input{log/73.log.tex}\end{stlog}
\begin{stlog}\input{log/74.log.tex}\end{stlog}
\color{black}

This is a matrix that has cumulative QALYs for 
any given age before development of MI (i.e., cumulative QALYs
for the alive without MI health state). The same is done for time spent with MI
, assuming no effect of MI duration on utilities, just age:

\color{Blue4}
\begin{stlog}\input{log/75.log.tex}\end{stlog}
\begin{stlog}\input{log/76.log.tex}\end{stlog}
\color{black}

Next is a cost matrix for acute MI, with the only variation in 
cost coming from discounting reflecting the age at MI (and thus time since 
intervention). 

\color{Blue4}
\begin{stlog}\input{log/77.log.tex}\end{stlog}
\color{black}

And also chronic costs of MI:

\color{Blue4}
\begin{stlog}\input{log/78.log.tex}\end{stlog}
\begin{stlog}\input{log/79.log.tex}\end{stlog}
\color{black}

And finally, annual drug costs. 
This is straightforward for the interventions, but is trickier for the control, 
as everyone starts therapy at a different time (for primary prevention), and then 
everyone initiates LLT following an MI. 
The way to do this is have two matrices; one for primary prevention statin costs 
(that can be recoded to 0 for people without LLT later on) and post-MI statin costs 
(applied to everyone with MI in the control arm). 

\color{Blue4}
\begin{stlog}\input{log/80.log.tex}\end{stlog}
\begin{stlog}\input{log/81.log.tex}\end{stlog}
\begin{stlog}\input{log/82.log.tex}\end{stlog}
\begin{stlog}\input{log/83.log.tex}\end{stlog}
\begin{stlog}\input{log/84.log.tex}\end{stlog}
\begin{stlog}\input{log/85.log.tex}\end{stlog}
\color{black}

\subsection{Health economic analysis}

With all the utility and cost matrices, all the relevant results from each simulation
can be calculated. Note that values are only counted from intervention start, despite the simulations
all starting from age 30.

\color{Blue4}
\begin{stlog}\input{log/86.log.tex}\end{stlog}
\color{black}

\begin{table}[h!]
  \begin{center}
    \caption{Microsimulation results -- Low/moderate intensity statins}
    \label{Microsim1}
     \fontsize{6pt}{8pt}\selectfont\pgfplotstabletypeset[
      multicolumn names,
      col sep=colon,
      header=false,
      string type,
	  display columns/0/.style={column name=,
		assign cell content/.code={
\pgfkeyssetvalue{/pgfplots/table/@cell content}
{\multirow{10}{*}{##1}}}},
      display columns/1/.style={column name=, column type={l}, text indicator="},
      display columns/2/.style={column name=, column type={r}, column type/.add={|}{|}},
      display columns/3/.style={column name=Absolute value, column type={r}},
      display columns/4/.style={column name=Difference to control, column type={r}, column type/.add={}{|}},
      every head row/.style={
        before row={\toprule
					\multicolumn{1}{c}{Age of intervention} & \multicolumn{1}{c}{Outcome} & \multicolumn{1}{c}{Control} & \multicolumn{2}{c}{Intervention}\\
					},
        after row={\midrule}
            },
        every nth row={10}{before row=\midrule},
        every last row/.style={after row=\bottomrule},
    ]{CSV/Res_HOF_1.csv}
  \end{center}
\end{table}

\begin{table}[h!]
  \begin{center}
    \caption{Microsimulation results -- High intensity statins}
    \label{Microsim2}
     \fontsize{6pt}{8pt}\selectfont\pgfplotstabletypeset[
      multicolumn names,
      col sep=colon,
      header=false,
      string type,
	  display columns/0/.style={column name=,
		assign cell content/.code={
\pgfkeyssetvalue{/pgfplots/table/@cell content}
{\multirow{10}{*}{##1}}}},
      display columns/1/.style={column name=, column type={l}, text indicator="},
      display columns/2/.style={column name=, column type={r}, column type/.add={|}{|}},
      display columns/3/.style={column name=Absolute value, column type={r}},
      display columns/4/.style={column name=Difference to control, column type={r}, column type/.add={}{|}},
      every head row/.style={
        before row={\toprule
					\multicolumn{1}{c}{Age of intervention} & \multicolumn{1}{c}{Outcome} & \multicolumn{1}{c}{Control} & \multicolumn{2}{c}{Intervention}\\
					},
        after row={\midrule}
            },
        every nth row={10}{before row=\midrule},
        every last row/.style={after row=\bottomrule},
    ]{CSV/Res_HOF_2.csv}
  \end{center}
\end{table}


\begin{table}[h!]
  \begin{center}
    \caption{Microsimulation results -- Low/moderate intensity statins and ezetimibe}
    \label{Microsim3}
     \fontsize{6pt}{8pt}\selectfont\pgfplotstabletypeset[
      multicolumn names,
      col sep=colon,
      header=false,
      string type,
	  display columns/0/.style={column name=,
		assign cell content/.code={
\pgfkeyssetvalue{/pgfplots/table/@cell content}
{\multirow{10}{*}{##1}}}},
      display columns/1/.style={column name=, column type={l}, text indicator="},
      display columns/2/.style={column name=, column type={r}, column type/.add={|}{|}},
      display columns/3/.style={column name=Absolute value, column type={r}},
      display columns/4/.style={column name=Difference to control, column type={r}, column type/.add={}{|}},
      every head row/.style={
        before row={\toprule
					\multicolumn{1}{c}{Age of intervention} & \multicolumn{1}{c}{Outcome} & \multicolumn{1}{c}{Control} & \multicolumn{2}{c}{Intervention}\\
					},
        after row={\midrule}
            },
        every nth row={10}{before row=\midrule},
        every last row/.style={after row=\bottomrule},
    ]{CSV/Res_HOF_3.csv}
  \end{center}
\end{table}

\begin{table}[h!]
  \begin{center}
    \caption{Microsimulation results -- Inclisiran}
    \label{Microsim4}
     \fontsize{6pt}{8pt}\selectfont\pgfplotstabletypeset[
      multicolumn names,
      col sep=colon,
      header=false,
      string type,
	  display columns/0/.style={column name=,
		assign cell content/.code={
\pgfkeyssetvalue{/pgfplots/table/@cell content}
{\multirow{10}{*}{##1}}}},
      display columns/1/.style={column name=, column type={l}, text indicator="},
      display columns/2/.style={column name=, column type={r}, column type/.add={|}{|}},
      display columns/3/.style={column name=Absolute value, column type={r}},
      display columns/4/.style={column name=Difference to control, column type={r}, column type/.add={}{|}},
      every head row/.style={
        before row={\toprule
					\multicolumn{1}{c}{Age of intervention} & \multicolumn{1}{c}{Outcome} & \multicolumn{1}{c}{Control} & \multicolumn{2}{c}{Intervention}\\
					},
        after row={\midrule}
            },
        every nth row={10}{before row=\midrule},
        every last row/.style={after row=\bottomrule},
    ]{CSV/Res_HOF_4.csv}
  \end{center}
\end{table}

\clearpage

Or, a simpler table to summarise the results: 

\begin{table}[h!]
  \begin{center}
    \caption{Microsimulation results -- Summary of all interventions. All results shown are the difference between the intervention and control.}
    \label{Microsim5}
     \fontsize{6pt}{8pt}\selectfont\pgfplotstabletypeset[
      multicolumn names,
      col sep=colon,
      header=false,
      string type,
	  display columns/0/.style={column name=Age of intervention,
		assign cell content/.code={
\pgfkeyssetvalue{/pgfplots/table/@cell content}
{\multirow{4}{*}{##1}}}},
      display columns/1/.style={column name=Outcome, column type={l}, text indicator=", column type/.add={}{|}},
      display columns/2/.style={column name= \specialcell{\noindent Low/moderate \\ intensity statins}, column type={r}},
      display columns/3/.style={column name=High intensity statins, column type={r}, column type/.add={}{}},
      display columns/4/.style={column name=\specialcell{\noindent Low/moderate intensity \\ statins and ezetimibe}, column type={r}},
      display columns/5/.style={column name=Inclisiran, column type={r}, column type/.add={}{}},
      every head row/.style={
        before row={\toprule
					},
        after row={\midrule}
            },
        every nth row={4}{before row=\midrule},
        every last row/.style={after row=\bottomrule},
    ]{CSV/Res_HOF.csv}
  \end{center}
\end{table}

\clearpage

So, it appears that all statin/ezetimibe interventions are cost-effective, but
Inclisiran is not even close at current prices. 
Nevertheless, all interventions come at a major cost -- even if statins are very cheap, 
giving them to the entire population across most of their lifespan makes this a very
expensive public health intervention. Moreover, 
it's likely that the cost-effectiveness will be significnatly
different among different population groups (most notably by sex and LDL-C). 
Therefore, it's probably wise to target this intervention to people most 
likely to benefit -- from here, I will stratify the results by sex 
and LDL-C ($\geq$3.0, $\geq$4.0, and $\geq$5.0 mmol/L).

\subsection{Stratified by sex and LDL-C}

First, I'll present the mean and median (IQR) values of LDL-C in these groups by 
sex to match the more common way of presenting model populations 
(Table~\ref{mldltab}). 



\begin{table}[h!]
  \begin{center}
    \caption{Mean; median (IQR) LDL-C by stratification group.}
    \label{mldltab}
     \selectfont\pgfplotstabletypeset[
      multicolumn names,
      col sep=colon,
      header=false,
      string type,
	  display columns/0/.style={column name=Sex,
		assign cell content/.code={
\pgfkeyssetvalue{/pgfplots/table/@cell content}
{\multirow{4}{*}{##1}}}},
      display columns/1/.style={column name=LDL-C (mmol/L), column type={l}, text indicator=", column type/.add={}{|}},
      display columns/2/.style={column name= Mean; median (IQR), column type={r}},
      every head row/.style={
        before row={\toprule
					},
        after row={\midrule}
            },
        every nth row={4}{before row=\midrule},
        every last row/.style={after row=\bottomrule},
    ]{CSV/mldl.csv}
  \end{center}
\end{table}


\color{Blue4}
\begin{stlog}\input{log/87.log.tex}\end{stlog}
\begin{figure}
    \centering
    \includegraphics[width=0.8\textwidth]{log/88.pdf}
    \caption{Cumulative incidence of MI or coronary death by age of intervention, sex, and LDL-C -- low/moderate intensity statins}
    \label{cumMIint00}
\end{figure}
\begin{figure}
    \centering
    \includegraphics[width=0.8\textwidth]{log/88_1.pdf}
    \caption{Cumulative incidence of MI or coronary death by age of intervention sex, and LDL-C -- high intensity statins}
    \label{cumMIint00}
\end{figure}
\begin{figure}
    \centering
    \includegraphics[width=0.8\textwidth]{log/88_2.pdf}
    \caption{Cumulative incidence of MI or coronary death by age of intervention, sex, and LDL-C -- low/moderate intensity statins and ezetimibe}
    \label{cumMIint00}
\end{figure}
\begin{figure}
    \centering
    \includegraphics[width=0.8\textwidth]{log/88_3.pdf}
    \caption{Cumulative incidence of MI or coronary death by age of intervention, sex, and LDL-C -- inclisiran}
    \label{cumMIint00}
\end{figure}
\begin{stlog}\input{log/88.log.tex}\end{stlog}
\color{black}

\clearpage

Indeed, stratification was a good idea - the relative risk reductions are pretty much identical
for each group, but the absolute risk reductions are higher in males and increase with increasing
LDL-C (\emph{``and with greater absolute risk reduction comes greater cost-effectiveness'' -- Uncle Ben, probably}). 
Let's also generate the full results table by sex and LDL-C: 

\color{Blue4}
\begin{stlog}\input{log/89.log.tex}\end{stlog}
\color{black}

So, definitely too many results, but I will display them anyway for completeness. 
Go to the last 4 tables for a summary (table~\ref{Microsimintsum1} - ~\ref{Microsimintsum4}).

\begin{table}[h!]
  \begin{center}
    \caption{Microsimulation results -- Low/moderate intensity statins. Females.}
    \label{Microsimsex0int1}
     \fontsize{6pt}{8pt}\selectfont\pgfplotstabletypeset[
      multicolumn names,
      col sep=colon,
      header=false,
      string type,
	  display columns/0/.style={column name=,
		assign cell content/.code={
\pgfkeyssetvalue{/pgfplots/table/@cell content}
{\multirow{10}{*}{##1}}}},
      display columns/1/.style={column name=, column type={l}, text indicator="},
      display columns/2/.style={column name=, column type={r}, column type/.add={|}{|}},
      display columns/3/.style={column name=Absolute value, column type={r}},
      display columns/4/.style={column name=Difference to control, column type={r}, column type/.add={}{|}},
      every head row/.style={
        before row={\toprule
					\multicolumn{1}{c}{Age of intervention} & \multicolumn{1}{c}{Outcome} & \multicolumn{1}{c}{Control} & \multicolumn{2}{c}{Intervention}\\
					},
        after row={\midrule}
            },
        every nth row={10}{before row=\midrule},
        every last row/.style={after row=\bottomrule},
    ]{CSV/Res_HOF_1_sex_0_ldl_0.csv}
  \end{center}
\end{table}

\begin{table}[h!]
  \begin{center}
    \caption{Microsimulation results -- Low/moderate intensity statins. Males.}
    \label{Microsimsex1int1}
     \fontsize{6pt}{8pt}\selectfont\pgfplotstabletypeset[
      multicolumn names,
      col sep=colon,
      header=false,
      string type,
	  display columns/0/.style={column name=,
		assign cell content/.code={
\pgfkeyssetvalue{/pgfplots/table/@cell content}
{\multirow{10}{*}{##1}}}},
      display columns/1/.style={column name=, column type={l}, text indicator="},
      display columns/2/.style={column name=, column type={r}, column type/.add={|}{|}},
      display columns/3/.style={column name=Absolute value, column type={r}},
      display columns/4/.style={column name=Difference to control, column type={r}, column type/.add={}{|}},
      every head row/.style={
        before row={\toprule
					\multicolumn{1}{c}{Age of intervention} & \multicolumn{1}{c}{Outcome} & \multicolumn{1}{c}{Control} & \multicolumn{2}{c}{Intervention}\\
					},
        after row={\midrule}
            },
        every nth row={10}{before row=\midrule},
        every last row/.style={after row=\bottomrule},
    ]{CSV/Res_HOF_1_sex_1_ldl_0.csv}
  \end{center}
\end{table}


\begin{table}[h!]
  \begin{center}
    \caption{Microsimulation results -- Low/moderate intensity statins. Females with an LDL-C $\geq$3.0 mmol/L.}
    \label{Microsimsex0ldl3int1}
     \fontsize{6pt}{8pt}\selectfont\pgfplotstabletypeset[
      multicolumn names,
      col sep=colon,
      header=false,
      string type,
	  display columns/0/.style={column name=,
		assign cell content/.code={
\pgfkeyssetvalue{/pgfplots/table/@cell content}
{\multirow{10}{*}{##1}}}},
      display columns/1/.style={column name=, column type={l}, text indicator="},
      display columns/2/.style={column name=, column type={r}, column type/.add={|}{|}},
      display columns/3/.style={column name=Absolute value, column type={r}},
      display columns/4/.style={column name=Difference to control, column type={r}, column type/.add={}{|}},
      every head row/.style={
        before row={\toprule
					\multicolumn{1}{c}{Age of intervention} & \multicolumn{1}{c}{Outcome} & \multicolumn{1}{c}{Control} & \multicolumn{2}{c}{Intervention}\\
					},
        after row={\midrule}
            },
        every nth row={10}{before row=\midrule},
        every last row/.style={after row=\bottomrule},
    ]{CSV/Res_HOF_1_sex_0_ldl_3.csv}
  \end{center}
\end{table}

\begin{table}[h!]
  \begin{center}
    \caption{Microsimulation results -- Low/moderate intensity statins. Males with an LDL-C $\geq$3.0 mmol/L.}
    \label{Microsimsex1ldl3int1}
     \fontsize{6pt}{8pt}\selectfont\pgfplotstabletypeset[
      multicolumn names,
      col sep=colon,
      header=false,
      string type,
	  display columns/0/.style={column name=,
		assign cell content/.code={
\pgfkeyssetvalue{/pgfplots/table/@cell content}
{\multirow{10}{*}{##1}}}},
      display columns/1/.style={column name=, column type={l}, text indicator="},
      display columns/2/.style={column name=, column type={r}, column type/.add={|}{|}},
      display columns/3/.style={column name=Absolute value, column type={r}},
      display columns/4/.style={column name=Difference to control, column type={r}, column type/.add={}{|}},
      every head row/.style={
        before row={\toprule
					\multicolumn{1}{c}{Age of intervention} & \multicolumn{1}{c}{Outcome} & \multicolumn{1}{c}{Control} & \multicolumn{2}{c}{Intervention}\\
					},
        after row={\midrule}
            },
        every nth row={10}{before row=\midrule},
        every last row/.style={after row=\bottomrule},
    ]{CSV/Res_HOF_1_sex_1_ldl_3.csv}
  \end{center}
\end{table}

\begin{table}[h!]
  \begin{center}
    \caption{Microsimulation results -- Low/moderate intensity statins. Females with an LDL-C $\geq$4.0 mmol/L.}
    \label{Microsimsex0ldl4int1}
     \fontsize{6pt}{8pt}\selectfont\pgfplotstabletypeset[
      multicolumn names,
      col sep=colon,
      header=false,
      string type,
	  display columns/0/.style={column name=,
		assign cell content/.code={
\pgfkeyssetvalue{/pgfplots/table/@cell content}
{\multirow{10}{*}{##1}}}},
      display columns/1/.style={column name=, column type={l}, text indicator="},
      display columns/2/.style={column name=, column type={r}, column type/.add={|}{|}},
      display columns/3/.style={column name=Absolute value, column type={r}},
      display columns/4/.style={column name=Difference to control, column type={r}, column type/.add={}{|}},
      every head row/.style={
        before row={\toprule
					\multicolumn{1}{c}{Age of intervention} & \multicolumn{1}{c}{Outcome} & \multicolumn{1}{c}{Control} & \multicolumn{2}{c}{Intervention}\\
					},
        after row={\midrule}
            },
        every nth row={10}{before row=\midrule},
        every last row/.style={after row=\bottomrule},
    ]{CSV/Res_HOF_1_sex_0_ldl_4.csv}
  \end{center}
\end{table}

\begin{table}[h!]
  \begin{center}
    \caption{Microsimulation results -- Low/moderate intensity statins. Males with an LDL-C $\geq$4.0 mmol/L.}
    \label{Microsimsex1ldl4int1}
     \fontsize{6pt}{8pt}\selectfont\pgfplotstabletypeset[
      multicolumn names,
      col sep=colon,
      header=false,
      string type,
	  display columns/0/.style={column name=,
		assign cell content/.code={
\pgfkeyssetvalue{/pgfplots/table/@cell content}
{\multirow{10}{*}{##1}}}},
      display columns/1/.style={column name=, column type={l}, text indicator="},
      display columns/2/.style={column name=, column type={r}, column type/.add={|}{|}},
      display columns/3/.style={column name=Absolute value, column type={r}},
      display columns/4/.style={column name=Difference to control, column type={r}, column type/.add={}{|}},
      every head row/.style={
        before row={\toprule
					\multicolumn{1}{c}{Age of intervention} & \multicolumn{1}{c}{Outcome} & \multicolumn{1}{c}{Control} & \multicolumn{2}{c}{Intervention}\\
					},
        after row={\midrule}
            },
        every nth row={10}{before row=\midrule},
        every last row/.style={after row=\bottomrule},
    ]{CSV/Res_HOF_1_sex_1_ldl_4.csv}
  \end{center}
\end{table}

\begin{table}[h!]
  \begin{center}
    \caption{Microsimulation results -- Low/moderate intensity statins. Females with an LDL-C $\geq$5.0 mmol/L.}
    \label{Microsimsex0ldl5int1}
     \fontsize{6pt}{8pt}\selectfont\pgfplotstabletypeset[
      multicolumn names,
      col sep=colon,
      header=false,
      string type,
	  display columns/0/.style={column name=,
		assign cell content/.code={
\pgfkeyssetvalue{/pgfplots/table/@cell content}
{\multirow{10}{*}{##1}}}},
      display columns/1/.style={column name=, column type={l}, text indicator="},
      display columns/2/.style={column name=, column type={r}, column type/.add={|}{|}},
      display columns/3/.style={column name=Absolute value, column type={r}},
      display columns/4/.style={column name=Difference to control, column type={r}, column type/.add={}{|}},
      every head row/.style={
        before row={\toprule
					\multicolumn{1}{c}{Age of intervention} & \multicolumn{1}{c}{Outcome} & \multicolumn{1}{c}{Control} & \multicolumn{2}{c}{Intervention}\\
					},
        after row={\midrule}
            },
        every nth row={10}{before row=\midrule},
        every last row/.style={after row=\bottomrule},
    ]{CSV/Res_HOF_1_sex_0_ldl_5.csv}
  \end{center}
\end{table}

\begin{table}[h!]
  \begin{center}
    \caption{Microsimulation results -- Low/moderate intensity statins. Males with an LDL-C $\geq$5.0 mmol/L.}
    \label{Microsimsex1ldl5int1}
     \fontsize{6pt}{8pt}\selectfont\pgfplotstabletypeset[
      multicolumn names,
      col sep=colon,
      header=false,
      string type,
	  display columns/0/.style={column name=,
		assign cell content/.code={
\pgfkeyssetvalue{/pgfplots/table/@cell content}
{\multirow{10}{*}{##1}}}},
      display columns/1/.style={column name=, column type={l}, text indicator="},
      display columns/2/.style={column name=, column type={r}, column type/.add={|}{|}},
      display columns/3/.style={column name=Absolute value, column type={r}},
      display columns/4/.style={column name=Difference to control, column type={r}, column type/.add={}{|}},
      every head row/.style={
        before row={\toprule
					\multicolumn{1}{c}{Age of intervention} & \multicolumn{1}{c}{Outcome} & \multicolumn{1}{c}{Control} & \multicolumn{2}{c}{Intervention}\\
					},
        after row={\midrule}
            },
        every nth row={10}{before row=\midrule},
        every last row/.style={after row=\bottomrule},
    ]{CSV/Res_HOF_1_sex_1_ldl_5.csv}
  \end{center}
\end{table}


\begin{table}[h!]
  \begin{center}
    \caption{Microsimulation results -- High intensity statins. Females.}
    \label{Microsimsex0int2}
     \fontsize{6pt}{8pt}\selectfont\pgfplotstabletypeset[
      multicolumn names,
      col sep=colon,
      header=false,
      string type,
	  display columns/0/.style={column name=,
		assign cell content/.code={
\pgfkeyssetvalue{/pgfplots/table/@cell content}
{\multirow{10}{*}{##1}}}},
      display columns/1/.style={column name=, column type={l}, text indicator="},
      display columns/2/.style={column name=, column type={r}, column type/.add={|}{|}},
      display columns/3/.style={column name=Absolute value, column type={r}},
      display columns/4/.style={column name=Difference to control, column type={r}, column type/.add={}{|}},
      every head row/.style={
        before row={\toprule
					\multicolumn{1}{c}{Age of intervention} & \multicolumn{1}{c}{Outcome} & \multicolumn{1}{c}{Control} & \multicolumn{2}{c}{Intervention}\\
					},
        after row={\midrule}
            },
        every nth row={10}{before row=\midrule},
        every last row/.style={after row=\bottomrule},
    ]{CSV/Res_HOF_2_sex_0_ldl_0.csv}
  \end{center}
\end{table}

\begin{table}[h!]
  \begin{center}
    \caption{Microsimulation results -- High intensity statins. Males.}
    \label{Microsimsex1int2}
     \fontsize{6pt}{8pt}\selectfont\pgfplotstabletypeset[
      multicolumn names,
      col sep=colon,
      header=false,
      string type,
	  display columns/0/.style={column name=,
		assign cell content/.code={
\pgfkeyssetvalue{/pgfplots/table/@cell content}
{\multirow{10}{*}{##1}}}},
      display columns/1/.style={column name=, column type={l}, text indicator="},
      display columns/2/.style={column name=, column type={r}, column type/.add={|}{|}},
      display columns/3/.style={column name=Absolute value, column type={r}},
      display columns/4/.style={column name=Difference to control, column type={r}, column type/.add={}{|}},
      every head row/.style={
        before row={\toprule
					\multicolumn{1}{c}{Age of intervention} & \multicolumn{1}{c}{Outcome} & \multicolumn{1}{c}{Control} & \multicolumn{2}{c}{Intervention}\\
					},
        after row={\midrule}
            },
        every nth row={10}{before row=\midrule},
        every last row/.style={after row=\bottomrule},
    ]{CSV/Res_HOF_2_sex_1_ldl_0.csv}
  \end{center}
\end{table}


\begin{table}[h!]
  \begin{center}
    \caption{Microsimulation results -- High intensity statins. Females with an LDL-C $\geq$3.0 mmol/L.}
    \label{Microsimsex0ldl3int2}
     \fontsize{6pt}{8pt}\selectfont\pgfplotstabletypeset[
      multicolumn names,
      col sep=colon,
      header=false,
      string type,
	  display columns/0/.style={column name=,
		assign cell content/.code={
\pgfkeyssetvalue{/pgfplots/table/@cell content}
{\multirow{10}{*}{##1}}}},
      display columns/1/.style={column name=, column type={l}, text indicator="},
      display columns/2/.style={column name=, column type={r}, column type/.add={|}{|}},
      display columns/3/.style={column name=Absolute value, column type={r}},
      display columns/4/.style={column name=Difference to control, column type={r}, column type/.add={}{|}},
      every head row/.style={
        before row={\toprule
					\multicolumn{1}{c}{Age of intervention} & \multicolumn{1}{c}{Outcome} & \multicolumn{1}{c}{Control} & \multicolumn{2}{c}{Intervention}\\
					},
        after row={\midrule}
            },
        every nth row={10}{before row=\midrule},
        every last row/.style={after row=\bottomrule},
    ]{CSV/Res_HOF_2_sex_0_ldl_3.csv}
  \end{center}
\end{table}

\begin{table}[h!]
  \begin{center}
    \caption{Microsimulation results -- High intensity statins. Males with an LDL-C $\geq$3.0 mmol/L.}
    \label{Microsimsex1ldl3int2}
     \fontsize{6pt}{8pt}\selectfont\pgfplotstabletypeset[
      multicolumn names,
      col sep=colon,
      header=false,
      string type,
	  display columns/0/.style={column name=,
		assign cell content/.code={
\pgfkeyssetvalue{/pgfplots/table/@cell content}
{\multirow{10}{*}{##1}}}},
      display columns/1/.style={column name=, column type={l}, text indicator="},
      display columns/2/.style={column name=, column type={r}, column type/.add={|}{|}},
      display columns/3/.style={column name=Absolute value, column type={r}},
      display columns/4/.style={column name=Difference to control, column type={r}, column type/.add={}{|}},
      every head row/.style={
        before row={\toprule
					\multicolumn{1}{c}{Age of intervention} & \multicolumn{1}{c}{Outcome} & \multicolumn{1}{c}{Control} & \multicolumn{2}{c}{Intervention}\\
					},
        after row={\midrule}
            },
        every nth row={10}{before row=\midrule},
        every last row/.style={after row=\bottomrule},
    ]{CSV/Res_HOF_2_sex_1_ldl_3.csv}
  \end{center}
\end{table}

\begin{table}[h!]
  \begin{center}
    \caption{Microsimulation results -- High intensity statins. Females with an LDL-C $\geq$4.0 mmol/L.}
    \label{Microsimsex0ldl4int2}
     \fontsize{6pt}{8pt}\selectfont\pgfplotstabletypeset[
      multicolumn names,
      col sep=colon,
      header=false,
      string type,
	  display columns/0/.style={column name=,
		assign cell content/.code={
\pgfkeyssetvalue{/pgfplots/table/@cell content}
{\multirow{10}{*}{##1}}}},
      display columns/1/.style={column name=, column type={l}, text indicator="},
      display columns/2/.style={column name=, column type={r}, column type/.add={|}{|}},
      display columns/3/.style={column name=Absolute value, column type={r}},
      display columns/4/.style={column name=Difference to control, column type={r}, column type/.add={}{|}},
      every head row/.style={
        before row={\toprule
					\multicolumn{1}{c}{Age of intervention} & \multicolumn{1}{c}{Outcome} & \multicolumn{1}{c}{Control} & \multicolumn{2}{c}{Intervention}\\
					},
        after row={\midrule}
            },
        every nth row={10}{before row=\midrule},
        every last row/.style={after row=\bottomrule},
    ]{CSV/Res_HOF_2_sex_0_ldl_4.csv}
  \end{center}
\end{table}

\begin{table}[h!]
  \begin{center}
    \caption{Microsimulation results -- High intensity statins. Males with an LDL-C $\geq$4.0 mmol/L.}
    \label{Microsimsex1ldl4int2}
     \fontsize{6pt}{8pt}\selectfont\pgfplotstabletypeset[
      multicolumn names,
      col sep=colon,
      header=false,
      string type,
	  display columns/0/.style={column name=,
		assign cell content/.code={
\pgfkeyssetvalue{/pgfplots/table/@cell content}
{\multirow{10}{*}{##1}}}},
      display columns/1/.style={column name=, column type={l}, text indicator="},
      display columns/2/.style={column name=, column type={r}, column type/.add={|}{|}},
      display columns/3/.style={column name=Absolute value, column type={r}},
      display columns/4/.style={column name=Difference to control, column type={r}, column type/.add={}{|}},
      every head row/.style={
        before row={\toprule
					\multicolumn{1}{c}{Age of intervention} & \multicolumn{1}{c}{Outcome} & \multicolumn{1}{c}{Control} & \multicolumn{2}{c}{Intervention}\\
					},
        after row={\midrule}
            },
        every nth row={10}{before row=\midrule},
        every last row/.style={after row=\bottomrule},
    ]{CSV/Res_HOF_2_sex_1_ldl_4.csv}
  \end{center}
\end{table}

\begin{table}[h!]
  \begin{center}
    \caption{Microsimulation results -- High intensity statins. Females with an LDL-C $\geq$5.0 mmol/L.}
    \label{Microsimsex0ldl5int2}
     \fontsize{6pt}{8pt}\selectfont\pgfplotstabletypeset[
      multicolumn names,
      col sep=colon,
      header=false,
      string type,
	  display columns/0/.style={column name=,
		assign cell content/.code={
\pgfkeyssetvalue{/pgfplots/table/@cell content}
{\multirow{10}{*}{##1}}}},
      display columns/1/.style={column name=, column type={l}, text indicator="},
      display columns/2/.style={column name=, column type={r}, column type/.add={|}{|}},
      display columns/3/.style={column name=Absolute value, column type={r}},
      display columns/4/.style={column name=Difference to control, column type={r}, column type/.add={}{|}},
      every head row/.style={
        before row={\toprule
					\multicolumn{1}{c}{Age of intervention} & \multicolumn{1}{c}{Outcome} & \multicolumn{1}{c}{Control} & \multicolumn{2}{c}{Intervention}\\
					},
        after row={\midrule}
            },
        every nth row={10}{before row=\midrule},
        every last row/.style={after row=\bottomrule},
    ]{CSV/Res_HOF_2_sex_0_ldl_5.csv}
  \end{center}
\end{table}

\begin{table}[h!]
  \begin{center}
    \caption{Microsimulation results -- High intensity statins. Males with an LDL-C $\geq$5.0 mmol/L.}
    \label{Microsimsex1ldl5int2}
     \fontsize{6pt}{8pt}\selectfont\pgfplotstabletypeset[
      multicolumn names,
      col sep=colon,
      header=false,
      string type,
	  display columns/0/.style={column name=,
		assign cell content/.code={
\pgfkeyssetvalue{/pgfplots/table/@cell content}
{\multirow{10}{*}{##1}}}},
      display columns/1/.style={column name=, column type={l}, text indicator="},
      display columns/2/.style={column name=, column type={r}, column type/.add={|}{|}},
      display columns/3/.style={column name=Absolute value, column type={r}},
      display columns/4/.style={column name=Difference to control, column type={r}, column type/.add={}{|}},
      every head row/.style={
        before row={\toprule
					\multicolumn{1}{c}{Age of intervention} & \multicolumn{1}{c}{Outcome} & \multicolumn{1}{c}{Control} & \multicolumn{2}{c}{Intervention}\\
					},
        after row={\midrule}
            },
        every nth row={10}{before row=\midrule},
        every last row/.style={after row=\bottomrule},
    ]{CSV/Res_HOF_2_sex_1_ldl_5.csv}
  \end{center}
\end{table}


\begin{table}[h!]
  \begin{center}
    \caption{Microsimulation results -- Low/moderate intensity statins and ezetimibe. Females.}
    \label{Microsimsex0int3}
     \fontsize{6pt}{8pt}\selectfont\pgfplotstabletypeset[
      multicolumn names,
      col sep=colon,
      header=false,
      string type,
	  display columns/0/.style={column name=,
		assign cell content/.code={
\pgfkeyssetvalue{/pgfplots/table/@cell content}
{\multirow{10}{*}{##1}}}},
      display columns/1/.style={column name=, column type={l}, text indicator="},
      display columns/2/.style={column name=, column type={r}, column type/.add={|}{|}},
      display columns/3/.style={column name=Absolute value, column type={r}},
      display columns/4/.style={column name=Difference to control, column type={r}, column type/.add={}{|}},
      every head row/.style={
        before row={\toprule
					\multicolumn{1}{c}{Age of intervention} & \multicolumn{1}{c}{Outcome} & \multicolumn{1}{c}{Control} & \multicolumn{2}{c}{Intervention}\\
					},
        after row={\midrule}
            },
        every nth row={10}{before row=\midrule},
        every last row/.style={after row=\bottomrule},
    ]{CSV/Res_HOF_3_sex_0_ldl_0.csv}
  \end{center}
\end{table}

\begin{table}[h!]
  \begin{center}
    \caption{Microsimulation results -- Low/moderate intensity statins and ezetimibe. Males.}
    \label{Microsimsex1int3}
     \fontsize{6pt}{8pt}\selectfont\pgfplotstabletypeset[
      multicolumn names,
      col sep=colon,
      header=false,
      string type,
	  display columns/0/.style={column name=,
		assign cell content/.code={
\pgfkeyssetvalue{/pgfplots/table/@cell content}
{\multirow{10}{*}{##1}}}},
      display columns/1/.style={column name=, column type={l}, text indicator="},
      display columns/2/.style={column name=, column type={r}, column type/.add={|}{|}},
      display columns/3/.style={column name=Absolute value, column type={r}},
      display columns/4/.style={column name=Difference to control, column type={r}, column type/.add={}{|}},
      every head row/.style={
        before row={\toprule
					\multicolumn{1}{c}{Age of intervention} & \multicolumn{1}{c}{Outcome} & \multicolumn{1}{c}{Control} & \multicolumn{2}{c}{Intervention}\\
					},
        after row={\midrule}
            },
        every nth row={10}{before row=\midrule},
        every last row/.style={after row=\bottomrule},
    ]{CSV/Res_HOF_3_sex_1_ldl_0.csv}
  \end{center}
\end{table}


\begin{table}[h!]
  \begin{center}
    \caption{Microsimulation results -- Low/moderate intensity statins and ezetimibe. Females with an LDL-C $\geq$3.0 mmol/L.}
    \label{Microsimsex0ldl3int3}
     \fontsize{6pt}{8pt}\selectfont\pgfplotstabletypeset[
      multicolumn names,
      col sep=colon,
      header=false,
      string type,
	  display columns/0/.style={column name=,
		assign cell content/.code={
\pgfkeyssetvalue{/pgfplots/table/@cell content}
{\multirow{10}{*}{##1}}}},
      display columns/1/.style={column name=, column type={l}, text indicator="},
      display columns/2/.style={column name=, column type={r}, column type/.add={|}{|}},
      display columns/3/.style={column name=Absolute value, column type={r}},
      display columns/4/.style={column name=Difference to control, column type={r}, column type/.add={}{|}},
      every head row/.style={
        before row={\toprule
					\multicolumn{1}{c}{Age of intervention} & \multicolumn{1}{c}{Outcome} & \multicolumn{1}{c}{Control} & \multicolumn{2}{c}{Intervention}\\
					},
        after row={\midrule}
            },
        every nth row={10}{before row=\midrule},
        every last row/.style={after row=\bottomrule},
    ]{CSV/Res_HOF_3_sex_0_ldl_3.csv}
  \end{center}
\end{table}

\begin{table}[h!]
  \begin{center}
    \caption{Microsimulation results -- Low/moderate intensity statins and ezetimibe. Males with an LDL-C $\geq$3.0 mmol/L.}
    \label{Microsimsex1ldl3int3}
     \fontsize{6pt}{8pt}\selectfont\pgfplotstabletypeset[
      multicolumn names,
      col sep=colon,
      header=false,
      string type,
	  display columns/0/.style={column name=,
		assign cell content/.code={
\pgfkeyssetvalue{/pgfplots/table/@cell content}
{\multirow{10}{*}{##1}}}},
      display columns/1/.style={column name=, column type={l}, text indicator="},
      display columns/2/.style={column name=, column type={r}, column type/.add={|}{|}},
      display columns/3/.style={column name=Absolute value, column type={r}},
      display columns/4/.style={column name=Difference to control, column type={r}, column type/.add={}{|}},
      every head row/.style={
        before row={\toprule
					\multicolumn{1}{c}{Age of intervention} & \multicolumn{1}{c}{Outcome} & \multicolumn{1}{c}{Control} & \multicolumn{2}{c}{Intervention}\\
					},
        after row={\midrule}
            },
        every nth row={10}{before row=\midrule},
        every last row/.style={after row=\bottomrule},
    ]{CSV/Res_HOF_3_sex_1_ldl_3.csv}
  \end{center}
\end{table}

\begin{table}[h!]
  \begin{center}
    \caption{Microsimulation results -- Low/moderate intensity statins and ezetimibe. Females with an LDL-C $\geq$4.0 mmol/L.}
    \label{Microsimsex0ldl4int3}
     \fontsize{6pt}{8pt}\selectfont\pgfplotstabletypeset[
      multicolumn names,
      col sep=colon,
      header=false,
      string type,
	  display columns/0/.style={column name=,
		assign cell content/.code={
\pgfkeyssetvalue{/pgfplots/table/@cell content}
{\multirow{10}{*}{##1}}}},
      display columns/1/.style={column name=, column type={l}, text indicator="},
      display columns/2/.style={column name=, column type={r}, column type/.add={|}{|}},
      display columns/3/.style={column name=Absolute value, column type={r}},
      display columns/4/.style={column name=Difference to control, column type={r}, column type/.add={}{|}},
      every head row/.style={
        before row={\toprule
					\multicolumn{1}{c}{Age of intervention} & \multicolumn{1}{c}{Outcome} & \multicolumn{1}{c}{Control} & \multicolumn{2}{c}{Intervention}\\
					},
        after row={\midrule}
            },
        every nth row={10}{before row=\midrule},
        every last row/.style={after row=\bottomrule},
    ]{CSV/Res_HOF_3_sex_0_ldl_4.csv}
  \end{center}
\end{table}

\begin{table}[h!]
  \begin{center}
    \caption{Microsimulation results -- Low/moderate intensity statins and ezetimibe. Males with an LDL-C $\geq$4.0 mmol/L.}
    \label{Microsimsex1ldl4int3}
     \fontsize{6pt}{8pt}\selectfont\pgfplotstabletypeset[
      multicolumn names,
      col sep=colon,
      header=false,
      string type,
	  display columns/0/.style={column name=,
		assign cell content/.code={
\pgfkeyssetvalue{/pgfplots/table/@cell content}
{\multirow{10}{*}{##1}}}},
      display columns/1/.style={column name=, column type={l}, text indicator="},
      display columns/2/.style={column name=, column type={r}, column type/.add={|}{|}},
      display columns/3/.style={column name=Absolute value, column type={r}},
      display columns/4/.style={column name=Difference to control, column type={r}, column type/.add={}{|}},
      every head row/.style={
        before row={\toprule
					\multicolumn{1}{c}{Age of intervention} & \multicolumn{1}{c}{Outcome} & \multicolumn{1}{c}{Control} & \multicolumn{2}{c}{Intervention}\\
					},
        after row={\midrule}
            },
        every nth row={10}{before row=\midrule},
        every last row/.style={after row=\bottomrule},
    ]{CSV/Res_HOF_3_sex_1_ldl_4.csv}
  \end{center}
\end{table}

\begin{table}[h!]
  \begin{center}
    \caption{Microsimulation results -- Low/moderate intensity statins and ezetimibe. Females with an LDL-C $\geq$5.0 mmol/L.}
    \label{Microsimsex0ldl5int3}
     \fontsize{6pt}{8pt}\selectfont\pgfplotstabletypeset[
      multicolumn names,
      col sep=colon,
      header=false,
      string type,
	  display columns/0/.style={column name=,
		assign cell content/.code={
\pgfkeyssetvalue{/pgfplots/table/@cell content}
{\multirow{10}{*}{##1}}}},
      display columns/1/.style={column name=, column type={l}, text indicator="},
      display columns/2/.style={column name=, column type={r}, column type/.add={|}{|}},
      display columns/3/.style={column name=Absolute value, column type={r}},
      display columns/4/.style={column name=Difference to control, column type={r}, column type/.add={}{|}},
      every head row/.style={
        before row={\toprule
					\multicolumn{1}{c}{Age of intervention} & \multicolumn{1}{c}{Outcome} & \multicolumn{1}{c}{Control} & \multicolumn{2}{c}{Intervention}\\
					},
        after row={\midrule}
            },
        every nth row={10}{before row=\midrule},
        every last row/.style={after row=\bottomrule},
    ]{CSV/Res_HOF_3_sex_0_ldl_5.csv}
  \end{center}
\end{table}

\begin{table}[h!]
  \begin{center}
    \caption{Microsimulation results -- Low/moderate intensity statins and ezetimibe. Males with an LDL-C $\geq$5.0 mmol/L.}
    \label{Microsimsex1ldl5int3}
     \fontsize{6pt}{8pt}\selectfont\pgfplotstabletypeset[
      multicolumn names,
      col sep=colon,
      header=false,
      string type,
	  display columns/0/.style={column name=,
		assign cell content/.code={
\pgfkeyssetvalue{/pgfplots/table/@cell content}
{\multirow{10}{*}{##1}}}},
      display columns/1/.style={column name=, column type={l}, text indicator="},
      display columns/2/.style={column name=, column type={r}, column type/.add={|}{|}},
      display columns/3/.style={column name=Absolute value, column type={r}},
      display columns/4/.style={column name=Difference to control, column type={r}, column type/.add={}{|}},
      every head row/.style={
        before row={\toprule
					\multicolumn{1}{c}{Age of intervention} & \multicolumn{1}{c}{Outcome} & \multicolumn{1}{c}{Control} & \multicolumn{2}{c}{Intervention}\\
					},
        after row={\midrule}
            },
        every nth row={10}{before row=\midrule},
        every last row/.style={after row=\bottomrule},
    ]{CSV/Res_HOF_3_sex_1_ldl_5.csv}
  \end{center}
\end{table}


\begin{table}[h!]
  \begin{center}
    \caption{Microsimulation results -- Inclisiran. Females.}
    \label{Microsimsex0int4}
     \fontsize{6pt}{8pt}\selectfont\pgfplotstabletypeset[
      multicolumn names,
      col sep=colon,
      header=false,
      string type,
	  display columns/0/.style={column name=,
		assign cell content/.code={
\pgfkeyssetvalue{/pgfplots/table/@cell content}
{\multirow{10}{*}{##1}}}},
      display columns/1/.style={column name=, column type={l}, text indicator="},
      display columns/2/.style={column name=, column type={r}, column type/.add={|}{|}},
      display columns/3/.style={column name=Absolute value, column type={r}},
      display columns/4/.style={column name=Difference to control, column type={r}, column type/.add={}{|}},
      every head row/.style={
        before row={\toprule
					\multicolumn{1}{c}{Age of intervention} & \multicolumn{1}{c}{Outcome} & \multicolumn{1}{c}{Control} & \multicolumn{2}{c}{Intervention}\\
					},
        after row={\midrule}
            },
        every nth row={10}{before row=\midrule},
        every last row/.style={after row=\bottomrule},
    ]{CSV/Res_HOF_4_sex_0_ldl_0.csv}
  \end{center}
\end{table}

\begin{table}[h!]
  \begin{center}
    \caption{Microsimulation results -- Inclisiran. Males.}
    \label{Microsimsex1int4}
     \fontsize{6pt}{8pt}\selectfont\pgfplotstabletypeset[
      multicolumn names,
      col sep=colon,
      header=false,
      string type,
	  display columns/0/.style={column name=,
		assign cell content/.code={
\pgfkeyssetvalue{/pgfplots/table/@cell content}
{\multirow{10}{*}{##1}}}},
      display columns/1/.style={column name=, column type={l}, text indicator="},
      display columns/2/.style={column name=, column type={r}, column type/.add={|}{|}},
      display columns/3/.style={column name=Absolute value, column type={r}},
      display columns/4/.style={column name=Difference to control, column type={r}, column type/.add={}{|}},
      every head row/.style={
        before row={\toprule
					\multicolumn{1}{c}{Age of intervention} & \multicolumn{1}{c}{Outcome} & \multicolumn{1}{c}{Control} & \multicolumn{2}{c}{Intervention}\\
					},
        after row={\midrule}
            },
        every nth row={10}{before row=\midrule},
        every last row/.style={after row=\bottomrule},
    ]{CSV/Res_HOF_4_sex_1_ldl_0.csv}
  \end{center}
\end{table}


\begin{table}[h!]
  \begin{center}
    \caption{Microsimulation results -- Inclisiran. Females with an LDL-C $\geq$3.0 mmol/L.}
    \label{Microsimsex0ldl3int4}
     \fontsize{6pt}{8pt}\selectfont\pgfplotstabletypeset[
      multicolumn names,
      col sep=colon,
      header=false,
      string type,
	  display columns/0/.style={column name=,
		assign cell content/.code={
\pgfkeyssetvalue{/pgfplots/table/@cell content}
{\multirow{10}{*}{##1}}}},
      display columns/1/.style={column name=, column type={l}, text indicator="},
      display columns/2/.style={column name=, column type={r}, column type/.add={|}{|}},
      display columns/3/.style={column name=Absolute value, column type={r}},
      display columns/4/.style={column name=Difference to control, column type={r}, column type/.add={}{|}},
      every head row/.style={
        before row={\toprule
					\multicolumn{1}{c}{Age of intervention} & \multicolumn{1}{c}{Outcome} & \multicolumn{1}{c}{Control} & \multicolumn{2}{c}{Intervention}\\
					},
        after row={\midrule}
            },
        every nth row={10}{before row=\midrule},
        every last row/.style={after row=\bottomrule},
    ]{CSV/Res_HOF_4_sex_0_ldl_3.csv}
  \end{center}
\end{table}

\begin{table}[h!]
  \begin{center}
    \caption{Microsimulation results -- Inclisiran. Males with an LDL-C $\geq$3.0 mmol/L.}
    \label{Microsimsex1ldl3int4}
     \fontsize{6pt}{8pt}\selectfont\pgfplotstabletypeset[
      multicolumn names,
      col sep=colon,
      header=false,
      string type,
	  display columns/0/.style={column name=,
		assign cell content/.code={
\pgfkeyssetvalue{/pgfplots/table/@cell content}
{\multirow{10}{*}{##1}}}},
      display columns/1/.style={column name=, column type={l}, text indicator="},
      display columns/2/.style={column name=, column type={r}, column type/.add={|}{|}},
      display columns/3/.style={column name=Absolute value, column type={r}},
      display columns/4/.style={column name=Difference to control, column type={r}, column type/.add={}{|}},
      every head row/.style={
        before row={\toprule
					\multicolumn{1}{c}{Age of intervention} & \multicolumn{1}{c}{Outcome} & \multicolumn{1}{c}{Control} & \multicolumn{2}{c}{Intervention}\\
					},
        after row={\midrule}
            },
        every nth row={10}{before row=\midrule},
        every last row/.style={after row=\bottomrule},
    ]{CSV/Res_HOF_4_sex_1_ldl_3.csv}
  \end{center}
\end{table}

\begin{table}[h!]
  \begin{center}
    \caption{Microsimulation results -- Inclisiran. Females with an LDL-C $\geq$4.0 mmol/L.}
    \label{Microsimsex0ldl4int4}
     \fontsize{6pt}{8pt}\selectfont\pgfplotstabletypeset[
      multicolumn names,
      col sep=colon,
      header=false,
      string type,
	  display columns/0/.style={column name=,
		assign cell content/.code={
\pgfkeyssetvalue{/pgfplots/table/@cell content}
{\multirow{10}{*}{##1}}}},
      display columns/1/.style={column name=, column type={l}, text indicator="},
      display columns/2/.style={column name=, column type={r}, column type/.add={|}{|}},
      display columns/3/.style={column name=Absolute value, column type={r}},
      display columns/4/.style={column name=Difference to control, column type={r}, column type/.add={}{|}},
      every head row/.style={
        before row={\toprule
					\multicolumn{1}{c}{Age of intervention} & \multicolumn{1}{c}{Outcome} & \multicolumn{1}{c}{Control} & \multicolumn{2}{c}{Intervention}\\
					},
        after row={\midrule}
            },
        every nth row={10}{before row=\midrule},
        every last row/.style={after row=\bottomrule},
    ]{CSV/Res_HOF_4_sex_0_ldl_4.csv}
  \end{center}
\end{table}

\begin{table}[h!]
  \begin{center}
    \caption{Microsimulation results -- Inclisiran. Males with an LDL-C $\geq$4.0 mmol/L.}
    \label{Microsimsex1ldl4int4}
     \fontsize{6pt}{8pt}\selectfont\pgfplotstabletypeset[
      multicolumn names,
      col sep=colon,
      header=false,
      string type,
	  display columns/0/.style={column name=,
		assign cell content/.code={
\pgfkeyssetvalue{/pgfplots/table/@cell content}
{\multirow{10}{*}{##1}}}},
      display columns/1/.style={column name=, column type={l}, text indicator="},
      display columns/2/.style={column name=, column type={r}, column type/.add={|}{|}},
      display columns/3/.style={column name=Absolute value, column type={r}},
      display columns/4/.style={column name=Difference to control, column type={r}, column type/.add={}{|}},
      every head row/.style={
        before row={\toprule
					\multicolumn{1}{c}{Age of intervention} & \multicolumn{1}{c}{Outcome} & \multicolumn{1}{c}{Control} & \multicolumn{2}{c}{Intervention}\\
					},
        after row={\midrule}
            },
        every nth row={10}{before row=\midrule},
        every last row/.style={after row=\bottomrule},
    ]{CSV/Res_HOF_4_sex_1_ldl_4.csv}
  \end{center}
\end{table}

\begin{table}[h!]
  \begin{center}
    \caption{Microsimulation results -- Inclisiran. Females with an LDL-C $\geq$5.0 mmol/L.}
    \label{Microsimsex0ldl5int4}
     \fontsize{6pt}{8pt}\selectfont\pgfplotstabletypeset[
      multicolumn names,
      col sep=colon,
      header=false,
      string type,
	  display columns/0/.style={column name=,
		assign cell content/.code={
\pgfkeyssetvalue{/pgfplots/table/@cell content}
{\multirow{10}{*}{##1}}}},
      display columns/1/.style={column name=, column type={l}, text indicator="},
      display columns/2/.style={column name=, column type={r}, column type/.add={|}{|}},
      display columns/3/.style={column name=Absolute value, column type={r}},
      display columns/4/.style={column name=Difference to control, column type={r}, column type/.add={}{|}},
      every head row/.style={
        before row={\toprule
					\multicolumn{1}{c}{Age of intervention} & \multicolumn{1}{c}{Outcome} & \multicolumn{1}{c}{Control} & \multicolumn{2}{c}{Intervention}\\
					},
        after row={\midrule}
            },
        every nth row={10}{before row=\midrule},
        every last row/.style={after row=\bottomrule},
    ]{CSV/Res_HOF_4_sex_0_ldl_5.csv}
  \end{center}
\end{table}

\begin{table}[h!]
  \begin{center}
    \caption{Microsimulation results -- Inclisiran. Males with an LDL-C $\geq$5.0 mmol/L.}
    \label{Microsimsex1ldl5int4}
     \fontsize{6pt}{8pt}\selectfont\pgfplotstabletypeset[
      multicolumn names,
      col sep=colon,
      header=false,
      string type,
	  display columns/0/.style={column name=,
		assign cell content/.code={
\pgfkeyssetvalue{/pgfplots/table/@cell content}
{\multirow{10}{*}{##1}}}},
      display columns/1/.style={column name=, column type={l}, text indicator="},
      display columns/2/.style={column name=, column type={r}, column type/.add={|}{|}},
      display columns/3/.style={column name=Absolute value, column type={r}},
      display columns/4/.style={column name=Difference to control, column type={r}, column type/.add={}{|}},
      every head row/.style={
        before row={\toprule
					\multicolumn{1}{c}{Age of intervention} & \multicolumn{1}{c}{Outcome} & \multicolumn{1}{c}{Control} & \multicolumn{2}{c}{Intervention}\\
					},
        after row={\midrule}
            },
        every nth row={10}{before row=\midrule},
        every last row/.style={after row=\bottomrule},
    ]{CSV/Res_HOF_4_sex_1_ldl_5.csv}
  \end{center}
\end{table}

\begin{table}[h!]
  \begin{center}
    \caption{Microsimulation results -- Low/moderate intensity statins. Summary.}
    \label{Microsimintsum1}
     \fontsize{6pt}{8pt}\selectfont\pgfplotstabletypeset[
      multicolumn names,
      col sep=colon,
      header=false,
      string type,
	  display columns/0/.style={column name=,
		assign cell content/.code={
\pgfkeyssetvalue{/pgfplots/table/@cell content}
{\multirow{16}{*}{##1}}}},
	  display columns/1/.style={column name=,
		assign cell content/.code={
\pgfkeyssetvalue{/pgfplots/table/@cell content}
{\multirow{4}{*}{##1}}}},
      display columns/2/.style={column name=, column type={l}, text indicator=", column type/.add={}{|}},
      display columns/3/.style={column name=Overall, column type={r}},
      display columns/4/.style={column name=$\geq$3.0 mmol/L, column type={r}},
      display columns/5/.style={column name=$\geq$4.0 mmol/L, column type={r}, column type/.add={}{}},
      display columns/6/.style={column name=$\geq$5.0 mmol/L, column type={r}, column type/.add={}{}},
      every head row/.style={
        before row={\toprule
					\multicolumn{1}{c}{Sex} & \multicolumn{1}{c}{Age of intervention} & \multicolumn{1}{c}{Outcome} & \multicolumn{4}{c}{LDL-C}\\
					},
        after row={\midrule}
            },
        every nth row={4}{before row=\cmidrule{2-7}},
        every nth row={16}{before row=\midrule},
        every last row/.style={after row=\bottomrule},
    ]{CSV/Res_HOF_sexldl_1.csv}
  \end{center}
\end{table}

\begin{table}[h!]
  \begin{center}
    \caption{Microsimulation results -- High intensity statins. Summary.}
    \label{Microsimintsum2}
     \fontsize{6pt}{8pt}\selectfont\pgfplotstabletypeset[
      multicolumn names,
      col sep=colon,
      header=false,
      string type,
	  display columns/0/.style={column name=,
		assign cell content/.code={
\pgfkeyssetvalue{/pgfplots/table/@cell content}
{\multirow{16}{*}{##1}}}},
	  display columns/1/.style={column name=,
		assign cell content/.code={
\pgfkeyssetvalue{/pgfplots/table/@cell content}
{\multirow{4}{*}{##1}}}},
      display columns/2/.style={column name=, column type={l}, text indicator=", column type/.add={}{|}},
      display columns/3/.style={column name=Overall, column type={r}},
      display columns/4/.style={column name=$\geq$3.0 mmol/L, column type={r}},
      display columns/5/.style={column name=$\geq$4.0 mmol/L, column type={r}, column type/.add={}{}},
      display columns/6/.style={column name=$\geq$5.0 mmol/L, column type={r}, column type/.add={}{}},
      every head row/.style={
        before row={\toprule
					\multicolumn{1}{c}{Sex} & \multicolumn{1}{c}{Age of intervention} & \multicolumn{1}{c}{Outcome} & \multicolumn{4}{c}{LDL-C}\\
					},
        after row={\midrule}
            },
        every nth row={4}{before row=\cmidrule{2-7}},
        every nth row={16}{before row=\midrule},
        every last row/.style={after row=\bottomrule},
    ]{CSV/Res_HOF_sexldl_2.csv}
  \end{center}
\end{table}


\begin{table}[h!]
  \begin{center}
    \caption{Microsimulation results -- Low/moderate intensity statins and ezetimibe. Summary.}
    \label{Microsimintsum3}
    \hspace*{-1.25cm}
     \fontsize{6pt}{8pt}\selectfont\pgfplotstabletypeset[
      multicolumn names,
      col sep=colon,
      header=false,
      string type,
	  display columns/0/.style={column name=,
		assign cell content/.code={
\pgfkeyssetvalue{/pgfplots/table/@cell content}
{\multirow{16}{*}{##1}}}},
	  display columns/1/.style={column name=,
		assign cell content/.code={
\pgfkeyssetvalue{/pgfplots/table/@cell content}
{\multirow{4}{*}{##1}}}},
      display columns/2/.style={column name=, column type={l}, text indicator=", column type/.add={}{|}},
      display columns/3/.style={column name=Overall, column type={r}},
      display columns/4/.style={column name=$\geq$3.0 mmol/L, column type={r}},
      display columns/5/.style={column name=$\geq$4.0 mmol/L, column type={r}, column type/.add={}{}},
      display columns/6/.style={column name=$\geq$5.0 mmol/L, column type={r}, column type/.add={}{}},
      every head row/.style={
        before row={\toprule
					\multicolumn{1}{c}{Sex} & \multicolumn{1}{c}{Age of intervention} & \multicolumn{1}{c}{Outcome} & \multicolumn{4}{c}{LDL-C}\\
					},
        after row={\midrule}
            },
        every nth row={4}{before row=\cmidrule{2-7}},
        every nth row={16}{before row=\midrule},
        every last row/.style={after row=\bottomrule},
    ]{CSV/Res_HOF_sexldl_3.csv}
  \end{center}
\end{table}

\begin{table}[h!]
  \begin{center}
    \caption{Microsimulation results -- Inclisiran. Summary.}
    \label{Microsimintsum4}
    \hspace*{-2.00cm}
     \fontsize{6pt}{8pt}\selectfont\pgfplotstabletypeset[
      multicolumn names,
      col sep=colon,
      header=false,
      string type,
	  display columns/0/.style={column name=,
		assign cell content/.code={
\pgfkeyssetvalue{/pgfplots/table/@cell content}
{\multirow{16}{*}{##1}}}},
	  display columns/1/.style={column name=,
		assign cell content/.code={
\pgfkeyssetvalue{/pgfplots/table/@cell content}
{\multirow{4}{*}{##1}}}},
      display columns/2/.style={column name=, column type={l}, text indicator=", column type/.add={}{|}},
      display columns/3/.style={column name=Overall, column type={r}},
      display columns/4/.style={column name=$\geq$3.0 mmol/L, column type={r}},
      display columns/5/.style={column name=$\geq$4.0 mmol/L, column type={r}, column type/.add={}{}},
      display columns/6/.style={column name=$\geq$5.0 mmol/L, column type={r}, column type/.add={}{}},
      every head row/.style={
        before row={\toprule
					\multicolumn{1}{c}{Sex} & \multicolumn{1}{c}{Age of intervention} & \multicolumn{1}{c}{Outcome} & \multicolumn{4}{c}{LDL-C}\\
					},
        after row={\midrule}
            },
        every nth row={4}{before row=\cmidrule{2-7}},
        every nth row={16}{before row=\midrule},
        every last row/.style={after row=\bottomrule},
    ]{CSV/Res_HOF_sexldl_4.csv}
  \end{center}
\end{table}

\clearpage




\pagebreak
\section{One-way sensitivity analyses}

To get any sort of confidence in the results just obtained, they need to subjected to sensitivity analyses. 
Table~\ref{inputtable} contains all inputs to the model, and thus all inputs that need to be varied for 
sensitivity analyses.

\begin{table}[]
    \caption{Model inputs}
    \label{inputtable}
\hspace*{-2cm}
\fontsize{6pt}{10pt}\selectfont\begin{tabular}{lllll}
\hline \\
Input & Value & Distribution & Source & \\
\hline \\
Incidence of non-fatal MI & Age and sex-specific & \specialcell{\noindent Log-normal \\ See Figure~\ref{MIinc}} & UK Biobank & \\
Incidence of fatal MI & Age and sex-specific & \specialcell{\noindent Log-normal \\ See Figure~\ref{NOCVDmort}} & UK Biobank & \\
\specialcell{\noindent Non-CHD mortality rate \\ for people without MI} & Age and sex-specific & \specialcell{\noindent Log-normal 
\\ See Figure~\ref{NOCVDmort}} & UK Biobank & \\
\specialcell{\noindent All-cause mortality rate \\ for people with MI} & \specialcell{\noindent Age-, sex-, and \\ 
time-since-MI-specific} & \specialcell{\noindent Log-normal \\ See Figure~\ref{PMImort}} & UK Biobank & \\
\specialcell{Effect of statins on LDL-C \\ (control arm only)} & 45\% (44, 46) reduction & Normal & \cite{AdamsCDSR2015} & \\
Effect of interventions on LDL-C & \specialcellll{Low/moderate intensity statins: 40\% (39, 41) reduction \\ 
High intensity statins: 50\% (49, 51) reduction \\ Low/moderate intensity statins and ezetimibe: 55\% (54, 56) reduction \\ 
Inclisiran: 51.5\% (49.0, 53.9) reduction} & Normal & \specialcellll{\cite{AdamsCDSR2015} \\ \cite{AdamsCDSR2015} \\ 
\cite{AmbeAth2014} \\ \cite{KausikNEJM2020}} & \\
\specialcell{\noindent Effect of cumulative LDL-C \\ on the incidence of MI} & RR: 0.48 (0.45, 0.50) & Log-normal & 
\cite{FerenceJAMA2019} & \\
Utility for people without MI & Age and sex-specific ($\pm$ 5\%) & \specialcell{\noindent Modified normal \\ 
See Figure~\ref{PSAFig4}} & \cite{ARAVIH2010} &  \\
Chronic utility for people with MI & 0.79 (0.73, 0.85) & Beta & \cite{BettsHQLO2020} & \\
Acute disutility for MI & -0.03 ($\pm$ 50\%) & Normal & \cite{LewisJACCHF2014} & \\
Cost of acute MI & \textsterling 2047.31 ($\pm$ 15\%) & Gamma  & \specialcell{\noindent National Health Service Cost \\ 
Schedule; See Table~\ref{ACUTEMICOST}} &  \\
\specialcell{\noindent Excess healthcare costs \\ for people with MI} & \specialcell{\noindent 
\textsterling 4705.45 (SE: 112.71) for the first 6 months \\ \textsterling 1015.21 (SE: 171.23) per year thereafter} & 
Gamma & \cite{DaneseeBMJO2016} &  \\ 
\specialcell{Annual cost of statins \\ (control arm only)} & \textsterling 19.00 & Fixed & \cite{NHSDrugTariff,NHSEPD} & \\
Annual cost of interventions & \specialcellll{Low/moderate intensity statins: \textsterling 18.39  \\ High intensity statins: 
\textsterling 27.39 \\ low/moderate intensity statins and ezetimibe: \textsterling 49.31 \\ Inclisiran: \textsterling 3974.72} & 
Fixed & \specialcellll{\cite{NHSDrugTariff} \\ \cite{NHSDrugTariff} \\ \cite{NHSDrugTariff} \\ \cite{NHSDMDInclisiran2022}} & \\
\hline
\end{tabular}
\end{table}

There are two kinds of sensitivity analyses that will be used in this study:
one-way and probabilistic sensitivity analyses. One-way sensitivity analyses 
are shown in this section, probabilistic in the next. For the one-way sensitivity
analyses, the primary outcome will be the ICER.
There are six inputs that must be varied for the microsimulation, 
and five that we vary after the fact. The first six:

\begin{enumerate}
\item Incidence of non-fatal MI
\item Incidence of fatal MI
\item Non-CHD mortality rate for people without MI
\item All-cause mortality rate for people with MI
\item Effect of therapies on LDL-C
\item Effect of LDL-C on MI risk
\end{enumerate}

And the following five:

\begin{enumerate}
\setcounter{enumi}{6}
\item Utility for people without MI
\item Chronic utility for people with MI
\item Acute disutility for MI
\item Cost of acute MI
\item Cost of chronic MI
\end{enumerate}

\subsection{Code}

\color{Blue4}
\begin{stlog}\input{log/90.log.tex}\end{stlog}
\color{black}

\subsection{Checks}

Before plotting these, it would be prudent to check there aren't any simulations 
with negative incremental QALYs.

\color{Blue4}
\begin{stlog}\input{log/91.log.tex}\end{stlog}
\color{black}

And also to check that the simulations have done what they're meant to have done:

\color{Blue4}
\begin{stlog}\input{log/92.log.tex}\end{stlog}
\color{black}

\subsection{Tornado diagrams}

With that all okay, the Tornado diagrams can be presented:

\color{Blue4}
\begin{stlog}\input{log/93.log.tex}\end{stlog}
\begin{stlog}\input{log/94.log.tex}\end{stlog}
\begin{stlog}\input{log/95.log.tex}\end{stlog}
\begin{figure}
    \centering
    \includegraphics[width=0.8\textwidth]{log/96.pdf}
    \caption{Tornado diagrams for each intervention strategy - Overall}
    \label{Tornado0}
\end{figure}
\begin{figure}
    \centering
    \includegraphics[width=0.8\textwidth]{log/96_1.pdf}
    \caption{Tornado diagrams for each intervention strategy - Females}
    \label{Tornado00}
\end{figure}
\begin{figure}
    \centering
    \includegraphics[width=0.8\textwidth]{log/96_2.pdf}
    \caption{Tornado diagrams for each intervention strategy - Females with LDL-C $\geq$3.0 mmol/L}
    \label{Tornado03}
\end{figure}
\begin{figure}
    \centering
    \includegraphics[width=0.8\textwidth]{log/96_3.pdf}
    \caption{Tornado diagrams for each intervention strategy - Females with LDL-C $\geq$4.0 mmol/L}
    \label{Tornado04}
\end{figure}
\begin{figure}
    \centering
    \includegraphics[width=0.8\textwidth]{log/96_4.pdf}
    \caption{Tornado diagrams for each intervention strategy - Females with LDL-C $\geq$5.0 mmol/L}
    \label{Tornado05}
\end{figure}
\begin{figure}
    \centering
    \includegraphics[width=0.8\textwidth]{log/96_5.pdf}
    \caption{Tornado diagrams for each intervention strategy - Males}
    \label{Tornado10}
\end{figure}
\begin{figure}
    \centering
    \includegraphics[width=0.8\textwidth]{log/96_6.pdf}
    \caption{Tornado diagrams for each intervention strategy - Males with LDL-C $\geq$3.0 mmol/L}
    \label{Tornado13}
\end{figure}
\begin{figure}
    \centering
    \includegraphics[width=0.8\textwidth]{log/96_7.pdf}
    \caption{Tornado diagrams for each intervention strategy - Males with LDL-C $\geq$4.0 mmol/L}
    \label{Tornado14}
\end{figure}
\begin{figure}
    \centering
    \includegraphics[width=0.8\textwidth]{log/96_8.pdf}
    \caption{Tornado diagrams for each intervention strategy - Males with LDL-C $\geq$5.0 mmol/L}
    \label{Tornado15}
\end{figure}
\begin{stlog}\input{log/96.log.tex}\end{stlog}
\color{black}

\clearpage
\pagebreak
\section{Probabilistic sensitivity analysis}

\subsection{Distributions}

The final sensitivity analysis is the probabilistic sensitivity analysis (PSA). 
First, the distributions for each of the model inputs must be derived. 
This is simple for: the incidence of MI and mortality rates, as they are just log-normally
distributed around the central value; the effect of the the interventions on LDL-C, which are normally distributed; 
and the effect of LDL-C on MI risk, which is log-normally distributed. 

The formula for the log-normal distributions is as follows (used above for the one-way sensitivity analyses): 

\begin{quote}
\begin{math}
a_{adj} = e^{\ln(a_\mu)+N(0,1) \sigma}
\end{math}
\end{quote}

The standard error for the rates are just derived from the regression models in section~\ref{TPs}.
The standard errors for the effect of the interventions on LDL-C: 

\begin{quote}
All interventions excluding Inclisiran:
\begin{math}
\sigma = \frac{0.02}{3.92} = 0.0051
\end{math}
\end{quote}

\begin{quote}
Inclisiran:
\begin{math}
\sigma = \frac{0.51-0.461}{3.92} = 0.0125
\end{math}
\end{quote}

And the standard error for the effect of LDL-C on MI risk: 

\begin{quote}
\begin{math}
\sigma = \frac{0.5-0.45}{3.92} = 0.0128
\end{math}
\end{quote}

As always, it's good to check these make sense (figures~\ref{PSAhist11} - ~\ref{PSAhist16}).

\color{Blue4}
\begin{figure}
    \centering
    \includegraphics[width=0.8\textwidth]{log/97.pdf}
    \caption{Histogram of effect of statins (control arm) on LDL-C}
    \label{PSAhist11}
\end{figure}
\begin{figure}
    \centering
    \includegraphics[width=0.8\textwidth]{log/97_1.pdf}
    \caption{Histogram of effect of low/moderate intensity statins on LDL-C}
    \label{PSAhist12}
\end{figure}
\begin{figure}
    \centering
    \includegraphics[width=0.8\textwidth]{log/97_2.pdf}
    \caption{Histogram of effect of high intensity statins on LDL-C}
    \label{PSAhist13}
\end{figure}
\begin{figure}
    \centering
    \includegraphics[width=0.8\textwidth]{log/97_3.pdf}
    \caption{Histogram of effect of low/moderate intensity statins and ezetimibe on LDL-C}
    \label{PSAhist14}
\end{figure}
\begin{figure}
    \centering
    \includegraphics[width=0.8\textwidth]{log/97_4.pdf}
    \caption{Histogram of effect of Inclisiran on LDL-C}
    \label{PSAhist15}
\end{figure}
\begin{figure}
    \centering
    \includegraphics[width=0.8\textwidth]{log/97_5.pdf}
    \caption{Histogram of relative risk of mean cumulative LDL-C on MI risk}
    \label{PSAhist16}
\end{figure}
\begin{stlog}\input{log/97.log.tex}\end{stlog}
\color{black}

Now, for the utility value for people without MI (which is already characterised by a function), it would not be efficient to generate
a unique beta distribution for each age and sex; instead, a modified normal distribution can be assumed for this (modified only in the 
sense that if the value falls outside the 0-1 range, it is constrained back to this range). 
Thus, it ends up looking like this (figure~\ref{PSAFig4}:

\color{Blue4}
\begin{stlog}\input{log/98.log.tex}\end{stlog}
\begin{figure}
    \centering
    \includegraphics[width=0.8\textwidth]{log/99.pdf}
    \caption{Distribution of utility values for people without MI in PSA}
    \label{PSAFig4}
\end{figure}
\begin{stlog}\input{log/99.log.tex}\end{stlog}
\color{black}

Now for the chronic utility value for people with MI (0.79 (0.73-0.85)), which has a beta distribution. 
The mean (\begin{math} \mu \end{math}) and variance (\begin{math} \sigma^2 \end{math})
of a beta distribution are given by:

\begin{quote}
\begin{math} 
\mu = \frac{\alpha}{\alpha + \beta}
\end{math}
\end{quote}

and

\begin{quote}
\begin{math} 
\sigma^2 = \frac{\alpha \beta}{(\alpha + \beta)^2 (\alpha + \beta + 1)}
\end{math}
\end{quote}

So, if solving for \begin{math} \alpha \end{math} and \begin{math} \beta \end{math}, it can be derived that:

\begin{quote}
\begin{math} 
\alpha = \mu^2 (\frac{1-\mu}{\sigma^2}-\frac{1}{\mu})
\end{math}
\end{quote}

and

\begin{quote}
\begin{math} 
\beta = \alpha (\frac{1}{\mu}-1)
\end{math}
\end{quote}

So for utility of people with MI, the variance is calculated as: 

\begin{quote}
\begin{math} 
\sigma^2 = (\frac{0.85-0.73}{3.92})^2 = 0.00094
\end{math}
\end{quote}

And this is used to derive alpha and beta values of 139.1, and 36.97, respectively (figure~\ref{PSAhist5}).

For the acute disutility, the standard error is as for a normal distribution:

\begin{quote}
\begin{math}
\sigma = \frac{0.015-0.045}{3.92} = 0.007653
\end{math}
\end{quote}

\color{Blue4}
\begin{figure}
    \centering
    \includegraphics[width=0.8\textwidth]{log/100.pdf}
    \caption{Histogram of chronic utility value for people with MI}
    \label{PSAhist5}
\end{figure}
\begin{figure}
    \centering
    \includegraphics[width=0.8\textwidth]{log/100_1.pdf}
    \caption{Histogram of acute disutility value for MI}
    \label{PSAhist15}
\end{figure}
\begin{stlog}\input{log/100.log.tex}\end{stlog}
\color{black}

Now for costs, which follow a gamma distribution. The mean (\begin{math} \mu \end{math}) 
and variance (\begin{math} \sigma^2 \end{math}) of a gamma distribution are given by:

\begin{quote}
\begin{math} 
\mu = k \theta
\end{math}
\end{quote}

and

\begin{quote}
\begin{math} 
\sigma^2 = k \theta^2
\end{math}
\end{quote}

It's easier to solve for \begin{math} k \end{math} and \begin{math} \theta \end{math} this time: 

\begin{quote}
\begin{math} 
k = \frac{\mu^2}{\sigma^2}
\end{math} \\
\\
\begin{math} 
\theta = \frac{\sigma^2}{\mu}
\end{math}
\end{quote}

So for the three costs, \begin{math} k \end{math} and \begin{math} \theta \end{math} values derived are:

\begin{quote}
Acute cost of MI (\textsterling 2047.31 (SE: 307.10)): 
\begin{math} 
k = 44.44 
\end{math}
and 
\begin{math} 
\theta = 46.06
\end{math} 
\\
Chronic cost of MI, first 6 months (\textsterling 4705.45 (SE: 112.71)): 
\begin{math} 
k = 1742.89
\end{math}
and 
\begin{math} 
\theta = 2.70
\end{math} 
\\
Chronic cost of MI, thereafter (\textsterling 1015.21 (SE: 171.23)): 
\begin{math} 
k = 35.15 
\end{math}
and 
\begin{math} 
\theta = 28.88
\end{math} 


\end{quote}

And again, check these make sense:

\color{Blue4}
\begin{figure}
    \centering
    \includegraphics[width=0.8\textwidth]{log/101.pdf}
    \caption{Histogram of Acute MI cost}
    \label{PSAhist6}
\end{figure}
\begin{figure}
    \centering
    \includegraphics[width=0.8\textwidth]{log/101_1.pdf}
    \caption{Histogram of Chronic MI cost, first 6 months}
    \label{PSAhist7}
\end{figure}
\begin{figure}
    \centering
    \includegraphics[width=0.8\textwidth]{log/101_2.pdf}
    \caption{Histogram of Chronic MI cost, after 6 months}
    \label{PSAhist8}
\end{figure}
\begin{stlog}\input{log/101.log.tex}\end{stlog}
\color{black}

\subsection{Code}

So now the PSA can be set up. In the interest of time, the PSA will be run in single year intervals.
(Note it's not good to re-set the seed so many times if you don't have to, but it's good to 
be able to stop and re-start it at points. Or even better, run simultaneously across multiple
cores.)

\color{Blue4}
\begin{stlog}\input{log/102.log.tex}\end{stlog}
\color{black}

\subsection{Checks}

Like the OSAs, before doing anything it's good to check there aren't any detectable issues with the PSA. 
First, check for negative incremental QALYs: 

\color{Blue4}
\begin{stlog}\input{log/103.log.tex}\end{stlog}
\color{black}

Just one, which is okay and won't impact the 95\% CIs. 
Also, check a few at random to make sure the results are reasonable:

\color{Blue4}
\begin{stlog}\input{log/104.log.tex}\end{stlog}
\color{black}

With those checks, there's some confidence there isn't a huge mistake with the PSA.
From here, results can be presented. There are a couple of results that would be enhanced
by the inclusion of PSA results -- the cumulative MI figures and the base-case results
tables -- as well as plotting the results of the PSAs across a common cost-effectiveness
plane. 

\subsection{Results: Cumulative incidence of MI}

Let's start with the figures.

\color{Blue4}
\begin{stlog}\input{log/105.log.tex}\end{stlog}
\begin{figure}
    \centering
    \includegraphics[width=0.8\textwidth]{log/106.pdf}
    \caption{Cumulative incidence of MI or coronary death, by intervention}
    \label{PSArfig1}
\end{figure}
\begin{figure}
    \centering
    \includegraphics[width=0.8\textwidth]{log/106_1.pdf}
    \caption{Cumulative incidence of MI or coronary death, by intervention and sex}
    \label{PSArfig2}
\end{figure}
\begin{figure}
    \centering
    \includegraphics[width=0.8\textwidth]{log/106_2.pdf}
    \caption{Cumulative incidence of MI or coronary death, by sex, LDL-C, and age of intervention -- Low/moderate intensity statins}
    \label{PSArfig3}
\end{figure}
\begin{figure}
    \centering
    \includegraphics[width=0.8\textwidth]{log/106_3.pdf}
    \caption{Cumulative incidence of MI or coronary death, by sex, LDL-C, and age of intervention -- High intensity statins}
    \label{PSArfig4}
\end{figure}
\begin{figure}
    \centering
    \includegraphics[width=0.8\textwidth]{log/106_4.pdf}
    \caption{Cumulative incidence of MI or coronary death, by sex, LDL-C, and age of intervention -- Low/moderate intensity statins and ezetimibe}
    \label{PSArfig5}
\end{figure}
\begin{figure}
    \centering
    \includegraphics[width=0.8\textwidth]{log/106_5.pdf}
    \caption{Cumulative incidence of MI or coronary death, by sex, LDL-C, and age of intervention -- Inclisiran}
    \label{PSArfig6}
\end{figure}
\begin{stlog}\input{log/106.log.tex}\end{stlog}
\color{black}

\subsection{Results: Tables}

It's also worth presenting the full results of the simulations
even if they will be very messy:

\color{Blue4}
\begin{stlog}\input{log/107.log.tex}\end{stlog}
\color{black}

This is extremely long/messy, so go to tables~\ref{Microsim5PSA} -- ~\ref{Microsim5PSA15} for the summaries. 

\begin{landscape}

\begin{table}[h!]
  \begin{center}
    \caption{PSA results -- Low/moderate intensity statins}
    \label{Microsim1PSA}
	\hspace*{-2.00cm}
     \fontsize{0.1pt}{0.15pt}\selectfont\pgfplotstabletypeset[
      multicolumn names,
      col sep=colon,
      header=false,
      string type,
	  display columns/0/.style={column name=,
		assign cell content/.code={
\pgfkeyssetvalue{/pgfplots/table/@cell content}
{\multirow{10}{*}{##1}}}},
      display columns/1/.style={column name=, column type={l}, text indicator="},
      display columns/2/.style={column name=, column type={r}, column type/.add={|}{|}},
      display columns/3/.style={column name=Absolute value, column type={r}},
      display columns/4/.style={column name=Difference to control, column type={r}, column type/.add={}{|}},
      every head row/.style={
        before row={\toprule
					\multicolumn{1}{c}{Age of intervention} & \multicolumn{1}{c}{Outcome} & \multicolumn{1}{c}{Control} & \multicolumn{2}{c}{Intervention}\\
					},
        after row={\midrule}
            },
        every nth row={10}{before row=\midrule},
        every last row/.style={after row=\bottomrule},
    ]{CSV/Res_HOF_PSA_1.csv}
  \end{center}
\end{table}

\begin{table}[h!]
  \begin{center}
    \caption{PSA results -- High intensity statins}
    \label{Microsim2PSA}
	\hspace*{-2.00cm}
     \fontsize{1pt}{1.5pt}\selectfont\pgfplotstabletypeset[
      multicolumn names,
      col sep=colon,
      header=false,
      string type,
	  display columns/0/.style={column name=,
		assign cell content/.code={
\pgfkeyssetvalue{/pgfplots/table/@cell content}
{\multirow{10}{*}{##1}}}},
      display columns/1/.style={column name=, column type={l}, text indicator="},
      display columns/2/.style={column name=, column type={r}, column type/.add={|}{|}},
      display columns/3/.style={column name=Absolute value, column type={r}},
      display columns/4/.style={column name=Difference to control, column type={r}, column type/.add={}{|}},
      every head row/.style={
        before row={\toprule
					\multicolumn{1}{c}{Age of intervention} & \multicolumn{1}{c}{Outcome} & \multicolumn{1}{c}{Control} & \multicolumn{2}{c}{Intervention}\\
					},
        after row={\midrule}
            },
        every nth row={10}{before row=\midrule},
        every last row/.style={after row=\bottomrule},
    ]{CSV/Res_HOF_PSA_2.csv}
  \end{center}
\end{table}


\begin{table}[h!]
  \begin{center}
    \caption{PSA results -- Low/moderate intensity statins and ezetimibe}
    \label{Microsim3PSA}
	\hspace*{-2.00cm}
     \fontsize{1pt}{1.5pt}\selectfont\pgfplotstabletypeset[
      multicolumn names,
      col sep=colon,
      header=false,
      string type,
	  display columns/0/.style={column name=,
		assign cell content/.code={
\pgfkeyssetvalue{/pgfplots/table/@cell content}
{\multirow{10}{*}{##1}}}},
      display columns/1/.style={column name=, column type={l}, text indicator="},
      display columns/2/.style={column name=, column type={r}, column type/.add={|}{|}},
      display columns/3/.style={column name=Absolute value, column type={r}},
      display columns/4/.style={column name=Difference to control, column type={r}, column type/.add={}{|}},
      every head row/.style={
        before row={\toprule
					\multicolumn{1}{c}{Age of intervention} & \multicolumn{1}{c}{Outcome} & \multicolumn{1}{c}{Control} & \multicolumn{2}{c}{Intervention}\\
					},
        after row={\midrule}
            },
        every nth row={10}{before row=\midrule},
        every last row/.style={after row=\bottomrule},
    ]{CSV/Res_HOF_PSA_3.csv}
  \end{center}
\end{table}

\begin{table}[h!]
  \begin{center}
    \caption{PSA results -- Inclisiran}
    \label{Microsim4PSA}
	\hspace*{-2.00cm}
     \fontsize{1pt}{1.5pt}\selectfont\pgfplotstabletypeset[
      multicolumn names,
      col sep=colon,
      header=false,
      string type,
	  display columns/0/.style={column name=,
		assign cell content/.code={
\pgfkeyssetvalue{/pgfplots/table/@cell content}
{\multirow{10}{*}{##1}}}},
      display columns/1/.style={column name=, column type={l}, text indicator="},
      display columns/2/.style={column name=, column type={r}, column type/.add={|}{|}},
      display columns/3/.style={column name=Absolute value, column type={r}},
      display columns/4/.style={column name=Difference to control, column type={r}, column type/.add={}{|}},
      every head row/.style={
        before row={\toprule
					\multicolumn{1}{c}{Age of intervention} & \multicolumn{1}{c}{Outcome} & \multicolumn{1}{c}{Control} & \multicolumn{2}{c}{Intervention}\\
					},
        after row={\midrule}
            },
        every nth row={10}{before row=\midrule},
        every last row/.style={after row=\bottomrule},
    ]{CSV/Res_HOF_PSA_4.csv}
  \end{center}
\end{table}



\begin{table}[h!]
  \begin{center}
    \caption{PSA results -- Low/moderate intensity statins -- All females}
    \label{Microsim1PSA00}
	\hspace*{-2.00cm}
     \fontsize{1pt}{1.5pt}\selectfont\pgfplotstabletypeset[
      multicolumn names,
      col sep=colon,
      header=false,
      string type,
	  display columns/0/.style={column name=,
		assign cell content/.code={
\pgfkeyssetvalue{/pgfplots/table/@cell content}
{\multirow{10}{*}{##1}}}},
      display columns/1/.style={column name=, column type={l}, text indicator="},
      display columns/2/.style={column name=, column type={r}, column type/.add={|}{|}},
      display columns/3/.style={column name=Absolute value, column type={r}},
      display columns/4/.style={column name=Difference to control, column type={r}, column type/.add={}{|}},
      every head row/.style={
        before row={\toprule
					\multicolumn{1}{c}{Age of intervention} & \multicolumn{1}{c}{Outcome} & \multicolumn{1}{c}{Control} & \multicolumn{2}{c}{Intervention}\\
					},
        after row={\midrule}
            },
        every nth row={10}{before row=\midrule},
        every last row/.style={after row=\bottomrule},
    ]{CSV/Res_HOF_PSA_1_sex_0_ldl_0.csv}
  \end{center}
\end{table}

\begin{table}[h!]
  \begin{center}
    \caption{PSA results -- High intensity statins -- All females}
    \label{Microsim2PSA00}
	\hspace*{-2.00cm}
     \fontsize{1pt}{1.5pt}\selectfont\pgfplotstabletypeset[
      multicolumn names,
      col sep=colon,
      header=false,
      string type,
	  display columns/0/.style={column name=,
		assign cell content/.code={
\pgfkeyssetvalue{/pgfplots/table/@cell content}
{\multirow{10}{*}{##1}}}},
      display columns/1/.style={column name=, column type={l}, text indicator="},
      display columns/2/.style={column name=, column type={r}, column type/.add={|}{|}},
      display columns/3/.style={column name=Absolute value, column type={r}},
      display columns/4/.style={column name=Difference to control, column type={r}, column type/.add={}{|}},
      every head row/.style={
        before row={\toprule
					\multicolumn{1}{c}{Age of intervention} & \multicolumn{1}{c}{Outcome} & \multicolumn{1}{c}{Control} & \multicolumn{2}{c}{Intervention}\\
					},
        after row={\midrule}
            },
        every nth row={10}{before row=\midrule},
        every last row/.style={after row=\bottomrule},
    ]{CSV/Res_HOF_PSA_2_sex_0_ldl_0.csv}
  \end{center}
\end{table}


\begin{table}[h!]
  \begin{center}
    \caption{PSA results -- Low/moderate intensity statins and ezetimibe -- All females}
    \label{Microsim3PSA00}
	\hspace*{-2.00cm}
     \fontsize{1pt}{1.5pt}\selectfont\pgfplotstabletypeset[
      multicolumn names,
      col sep=colon,
      header=false,
      string type,
	  display columns/0/.style={column name=,
		assign cell content/.code={
\pgfkeyssetvalue{/pgfplots/table/@cell content}
{\multirow{10}{*}{##1}}}},
      display columns/1/.style={column name=, column type={l}, text indicator="},
      display columns/2/.style={column name=, column type={r}, column type/.add={|}{|}},
      display columns/3/.style={column name=Absolute value, column type={r}},
      display columns/4/.style={column name=Difference to control, column type={r}, column type/.add={}{|}},
      every head row/.style={
        before row={\toprule
					\multicolumn{1}{c}{Age of intervention} & \multicolumn{1}{c}{Outcome} & \multicolumn{1}{c}{Control} & \multicolumn{2}{c}{Intervention}\\
					},
        after row={\midrule}
            },
        every nth row={10}{before row=\midrule},
        every last row/.style={after row=\bottomrule},
    ]{CSV/Res_HOF_PSA_3_sex_0_ldl_0.csv}
  \end{center}
\end{table}

\begin{table}[h!]
  \begin{center}
    \caption{PSA results -- Inclisiran -- All females}
    \label{Microsim4PSA00}
	\hspace*{-2.00cm}
     \fontsize{1pt}{1.5pt}\selectfont\pgfplotstabletypeset[
      multicolumn names,
      col sep=colon,
      header=false,
      string type,
	  display columns/0/.style={column name=,
		assign cell content/.code={
\pgfkeyssetvalue{/pgfplots/table/@cell content}
{\multirow{10}{*}{##1}}}},
      display columns/1/.style={column name=, column type={l}, text indicator="},
      display columns/2/.style={column name=, column type={r}, column type/.add={|}{|}},
      display columns/3/.style={column name=Absolute value, column type={r}},
      display columns/4/.style={column name=Difference to control, column type={r}, column type/.add={}{|}},
      every head row/.style={
        before row={\toprule
					\multicolumn{1}{c}{Age of intervention} & \multicolumn{1}{c}{Outcome} & \multicolumn{1}{c}{Control} & \multicolumn{2}{c}{Intervention}\\
					},
        after row={\midrule}
            },
        every nth row={10}{before row=\midrule},
        every last row/.style={after row=\bottomrule},
    ]{CSV/Res_HOF_PSA_4_sex_0_ldl_0.csv}
  \end{center}
\end{table}



\begin{table}[h!]
  \begin{center}
    \caption{PSA results -- Low/moderate intensity statins -- Females with LDL-C $\geq$3.0 mmol/L}
    \label{Microsim1PSA03}
	\hspace*{-2.00cm}
     \fontsize{1pt}{1.5pt}\selectfont\pgfplotstabletypeset[
      multicolumn names,
      col sep=colon,
      header=false,
      string type,
	  display columns/0/.style={column name=,
		assign cell content/.code={
\pgfkeyssetvalue{/pgfplots/table/@cell content}
{\multirow{10}{*}{##1}}}},
      display columns/1/.style={column name=, column type={l}, text indicator="},
      display columns/2/.style={column name=, column type={r}, column type/.add={|}{|}},
      display columns/3/.style={column name=Absolute value, column type={r}},
      display columns/4/.style={column name=Difference to control, column type={r}, column type/.add={}{|}},
      every head row/.style={
        before row={\toprule
					\multicolumn{1}{c}{Age of intervention} & \multicolumn{1}{c}{Outcome} & \multicolumn{1}{c}{Control} & \multicolumn{2}{c}{Intervention}\\
					},
        after row={\midrule}
            },
        every nth row={10}{before row=\midrule},
        every last row/.style={after row=\bottomrule},
    ]{CSV/Res_HOF_PSA_1_sex_0_ldl_3.csv}
  \end{center}
\end{table}

\begin{table}[h!]
  \begin{center}
    \caption{PSA results -- High intensity statins -- Females with LDL-C $\geq$3.0 mmol/L}
    \label{Microsim2PSA03}
	\hspace*{-2.00cm}
     \fontsize{1pt}{1.5pt}\selectfont\pgfplotstabletypeset[
      multicolumn names,
      col sep=colon,
      header=false,
      string type,
	  display columns/0/.style={column name=,
		assign cell content/.code={
\pgfkeyssetvalue{/pgfplots/table/@cell content}
{\multirow{10}{*}{##1}}}},
      display columns/1/.style={column name=, column type={l}, text indicator="},
      display columns/2/.style={column name=, column type={r}, column type/.add={|}{|}},
      display columns/3/.style={column name=Absolute value, column type={r}},
      display columns/4/.style={column name=Difference to control, column type={r}, column type/.add={}{|}},
      every head row/.style={
        before row={\toprule
					\multicolumn{1}{c}{Age of intervention} & \multicolumn{1}{c}{Outcome} & \multicolumn{1}{c}{Control} & \multicolumn{2}{c}{Intervention}\\
					},
        after row={\midrule}
            },
        every nth row={10}{before row=\midrule},
        every last row/.style={after row=\bottomrule},
    ]{CSV/Res_HOF_PSA_2_sex_0_ldl_3.csv}
  \end{center}
\end{table}


\begin{table}[h!]
  \begin{center}
    \caption{PSA results -- Low/moderate intensity statins and ezetimibe -- Females with LDL-C $\geq$3.0 mmol/L}
    \label{Microsim3PSA03}
	\hspace*{-2.00cm}
     \fontsize{1pt}{1.5pt}\selectfont\pgfplotstabletypeset[
      multicolumn names,
      col sep=colon,
      header=false,
      string type,
	  display columns/0/.style={column name=,
		assign cell content/.code={
\pgfkeyssetvalue{/pgfplots/table/@cell content}
{\multirow{10}{*}{##1}}}},
      display columns/1/.style={column name=, column type={l}, text indicator="},
      display columns/2/.style={column name=, column type={r}, column type/.add={|}{|}},
      display columns/3/.style={column name=Absolute value, column type={r}},
      display columns/4/.style={column name=Difference to control, column type={r}, column type/.add={}{|}},
      every head row/.style={
        before row={\toprule
					\multicolumn{1}{c}{Age of intervention} & \multicolumn{1}{c}{Outcome} & \multicolumn{1}{c}{Control} & \multicolumn{2}{c}{Intervention}\\
					},
        after row={\midrule}
            },
        every nth row={10}{before row=\midrule},
        every last row/.style={after row=\bottomrule},
    ]{CSV/Res_HOF_PSA_3_sex_0_ldl_3.csv}
  \end{center}
\end{table}

\begin{table}[h!]
  \begin{center}
    \caption{PSA results -- Inclisiran -- Females with LDL-C $\geq$3.0 mmol/L}
    \label{Microsim4PSA03}
	\hspace*{-2.00cm}
     \fontsize{1pt}{1.5pt}\selectfont\pgfplotstabletypeset[
      multicolumn names,
      col sep=colon,
      header=false,
      string type,
	  display columns/0/.style={column name=,
		assign cell content/.code={
\pgfkeyssetvalue{/pgfplots/table/@cell content}
{\multirow{10}{*}{##1}}}},
      display columns/1/.style={column name=, column type={l}, text indicator="},
      display columns/2/.style={column name=, column type={r}, column type/.add={|}{|}},
      display columns/3/.style={column name=Absolute value, column type={r}},
      display columns/4/.style={column name=Difference to control, column type={r}, column type/.add={}{|}},
      every head row/.style={
        before row={\toprule
					\multicolumn{1}{c}{Age of intervention} & \multicolumn{1}{c}{Outcome} & \multicolumn{1}{c}{Control} & \multicolumn{2}{c}{Intervention}\\
					},
        after row={\midrule}
            },
        every nth row={10}{before row=\midrule},
        every last row/.style={after row=\bottomrule},
    ]{CSV/Res_HOF_PSA_4_sex_0_ldl_3.csv}
  \end{center}
\end{table}


\begin{table}[h!]
  \begin{center}
    \caption{PSA results -- Low/moderate intensity statins -- Females with LDL-C $\geq$4.0 mmol/L}
    \label{Microsim1PSA04}
	\hspace*{-2.00cm}
     \fontsize{1pt}{1.5pt}\selectfont\pgfplotstabletypeset[
      multicolumn names,
      col sep=colon,
      header=false,
      string type,
	  display columns/0/.style={column name=,
		assign cell content/.code={
\pgfkeyssetvalue{/pgfplots/table/@cell content}
{\multirow{10}{*}{##1}}}},
      display columns/1/.style={column name=, column type={l}, text indicator="},
      display columns/2/.style={column name=, column type={r}, column type/.add={|}{|}},
      display columns/3/.style={column name=Absolute value, column type={r}},
      display columns/4/.style={column name=Difference to control, column type={r}, column type/.add={}{|}},
      every head row/.style={
        before row={\toprule
					\multicolumn{1}{c}{Age of intervention} & \multicolumn{1}{c}{Outcome} & \multicolumn{1}{c}{Control} & \multicolumn{2}{c}{Intervention}\\
					},
        after row={\midrule}
            },
        every nth row={10}{before row=\midrule},
        every last row/.style={after row=\bottomrule},
    ]{CSV/Res_HOF_PSA_1_sex_0_ldl_4.csv}
  \end{center}
\end{table}

\begin{table}[h!]
  \begin{center}
    \caption{PSA results -- High intensity statins -- Females with LDL-C $\geq$4.0 mmol/L}
    \label{Microsim2PSA04}
	\hspace*{-2.00cm}
     \fontsize{1pt}{1.5pt}\selectfont\pgfplotstabletypeset[
      multicolumn names,
      col sep=colon,
      header=false,
      string type,
	  display columns/0/.style={column name=,
		assign cell content/.code={
\pgfkeyssetvalue{/pgfplots/table/@cell content}
{\multirow{10}{*}{##1}}}},
      display columns/1/.style={column name=, column type={l}, text indicator="},
      display columns/2/.style={column name=, column type={r}, column type/.add={|}{|}},
      display columns/3/.style={column name=Absolute value, column type={r}},
      display columns/4/.style={column name=Difference to control, column type={r}, column type/.add={}{|}},
      every head row/.style={
        before row={\toprule
					\multicolumn{1}{c}{Age of intervention} & \multicolumn{1}{c}{Outcome} & \multicolumn{1}{c}{Control} & \multicolumn{2}{c}{Intervention}\\
					},
        after row={\midrule}
            },
        every nth row={10}{before row=\midrule},
        every last row/.style={after row=\bottomrule},
    ]{CSV/Res_HOF_PSA_2_sex_0_ldl_4.csv}
  \end{center}
\end{table}


\begin{table}[h!]
  \begin{center}
    \caption{PSA results -- Low/moderate intensity statins and ezetimibe -- Females with LDL-C $\geq$4.0 mmol/L}
    \label{Microsim3PSA04}
	\hspace*{-2.00cm}
     \fontsize{1pt}{1.5pt}\selectfont\pgfplotstabletypeset[
      multicolumn names,
      col sep=colon,
      header=false,
      string type,
	  display columns/0/.style={column name=,
		assign cell content/.code={
\pgfkeyssetvalue{/pgfplots/table/@cell content}
{\multirow{10}{*}{##1}}}},
      display columns/1/.style={column name=, column type={l}, text indicator="},
      display columns/2/.style={column name=, column type={r}, column type/.add={|}{|}},
      display columns/3/.style={column name=Absolute value, column type={r}},
      display columns/4/.style={column name=Difference to control, column type={r}, column type/.add={}{|}},
      every head row/.style={
        before row={\toprule
					\multicolumn{1}{c}{Age of intervention} & \multicolumn{1}{c}{Outcome} & \multicolumn{1}{c}{Control} & \multicolumn{2}{c}{Intervention}\\
					},
        after row={\midrule}
            },
        every nth row={10}{before row=\midrule},
        every last row/.style={after row=\bottomrule},
    ]{CSV/Res_HOF_PSA_3_sex_0_ldl_4.csv}
  \end{center}
\end{table}

\begin{table}[h!]
  \begin{center}
    \caption{PSA results -- Inclisiran -- Females with LDL-C $\geq$4.0 mmol/L}
    \label{Microsim4PSA04}
	\hspace*{-2.00cm}
     \fontsize{1pt}{1.5pt}\selectfont\pgfplotstabletypeset[
      multicolumn names,
      col sep=colon,
      header=false,
      string type,
	  display columns/0/.style={column name=,
		assign cell content/.code={
\pgfkeyssetvalue{/pgfplots/table/@cell content}
{\multirow{10}{*}{##1}}}},
      display columns/1/.style={column name=, column type={l}, text indicator="},
      display columns/2/.style={column name=, column type={r}, column type/.add={|}{|}},
      display columns/3/.style={column name=Absolute value, column type={r}},
      display columns/4/.style={column name=Difference to control, column type={r}, column type/.add={}{|}},
      every head row/.style={
        before row={\toprule
					\multicolumn{1}{c}{Age of intervention} & \multicolumn{1}{c}{Outcome} & \multicolumn{1}{c}{Control} & \multicolumn{2}{c}{Intervention}\\
					},
        after row={\midrule}
            },
        every nth row={10}{before row=\midrule},
        every last row/.style={after row=\bottomrule},
    ]{CSV/Res_HOF_PSA_4_sex_0_ldl_4.csv}
  \end{center}
\end{table}


\begin{table}[h!]
  \begin{center}
    \caption{PSA results -- Low/moderate intensity statins -- Females with LDL-C $\geq$5.0 mmol/L}
    \label{Microsim1PSA05}
	\hspace*{-2.00cm}
     \fontsize{1pt}{1.5pt}\selectfont\pgfplotstabletypeset[
      multicolumn names,
      col sep=colon,
      header=false,
      string type,
	  display columns/0/.style={column name=,
		assign cell content/.code={
\pgfkeyssetvalue{/pgfplots/table/@cell content}
{\multirow{10}{*}{##1}}}},
      display columns/1/.style={column name=, column type={l}, text indicator="},
      display columns/2/.style={column name=, column type={r}, column type/.add={|}{|}},
      display columns/3/.style={column name=Absolute value, column type={r}},
      display columns/4/.style={column name=Difference to control, column type={r}, column type/.add={}{|}},
      every head row/.style={
        before row={\toprule
					\multicolumn{1}{c}{Age of intervention} & \multicolumn{1}{c}{Outcome} & \multicolumn{1}{c}{Control} & \multicolumn{2}{c}{Intervention}\\
					},
        after row={\midrule}
            },
        every nth row={10}{before row=\midrule},
        every last row/.style={after row=\bottomrule},
    ]{CSV/Res_HOF_PSA_1_sex_0_ldl_5.csv}
  \end{center}
\end{table}

\begin{table}[h!]
  \begin{center}
    \caption{PSA results -- High intensity statins -- Females with LDL-C $\geq$5.0 mmol/L}
    \label{Microsim2PSA05}
	\hspace*{-2.00cm}
     \fontsize{1pt}{1.5pt}\selectfont\pgfplotstabletypeset[
      multicolumn names,
      col sep=colon,
      header=false,
      string type,
	  display columns/0/.style={column name=,
		assign cell content/.code={
\pgfkeyssetvalue{/pgfplots/table/@cell content}
{\multirow{10}{*}{##1}}}},
      display columns/1/.style={column name=, column type={l}, text indicator="},
      display columns/2/.style={column name=, column type={r}, column type/.add={|}{|}},
      display columns/3/.style={column name=Absolute value, column type={r}},
      display columns/4/.style={column name=Difference to control, column type={r}, column type/.add={}{|}},
      every head row/.style={
        before row={\toprule
					\multicolumn{1}{c}{Age of intervention} & \multicolumn{1}{c}{Outcome} & \multicolumn{1}{c}{Control} & \multicolumn{2}{c}{Intervention}\\
					},
        after row={\midrule}
            },
        every nth row={10}{before row=\midrule},
        every last row/.style={after row=\bottomrule},
    ]{CSV/Res_HOF_PSA_2_sex_0_ldl_5.csv}
  \end{center}
\end{table}


\begin{table}[h!]
  \begin{center}
    \caption{PSA results -- Low/moderate intensity statins and ezetimibe -- Females with LDL-C $\geq$5.0 mmol/L}
    \label{Microsim3PSA05}
	\hspace*{-2.00cm}
     \fontsize{1pt}{1.5pt}\selectfont\pgfplotstabletypeset[
      multicolumn names,
      col sep=colon,
      header=false,
      string type,
	  display columns/0/.style={column name=,
		assign cell content/.code={
\pgfkeyssetvalue{/pgfplots/table/@cell content}
{\multirow{10}{*}{##1}}}},
      display columns/1/.style={column name=, column type={l}, text indicator="},
      display columns/2/.style={column name=, column type={r}, column type/.add={|}{|}},
      display columns/3/.style={column name=Absolute value, column type={r}},
      display columns/4/.style={column name=Difference to control, column type={r}, column type/.add={}{|}},
      every head row/.style={
        before row={\toprule
					\multicolumn{1}{c}{Age of intervention} & \multicolumn{1}{c}{Outcome} & \multicolumn{1}{c}{Control} & \multicolumn{2}{c}{Intervention}\\
					},
        after row={\midrule}
            },
        every nth row={10}{before row=\midrule},
        every last row/.style={after row=\bottomrule},
    ]{CSV/Res_HOF_PSA_3_sex_0_ldl_5.csv}
  \end{center}
\end{table}

\begin{table}[h!]
  \begin{center}
    \caption{PSA results -- Inclisiran -- Females with LDL-C $\geq$5.0 mmol/L}
    \label{Microsim4PSA05}
	\hspace*{-2.00cm}
     \fontsize{1pt}{1.5pt}\selectfont\pgfplotstabletypeset[
      multicolumn names,
      col sep=colon,
      header=false,
      string type,
	  display columns/0/.style={column name=,
		assign cell content/.code={
\pgfkeyssetvalue{/pgfplots/table/@cell content}
{\multirow{10}{*}{##1}}}},
      display columns/1/.style={column name=, column type={l}, text indicator="},
      display columns/2/.style={column name=, column type={r}, column type/.add={|}{|}},
      display columns/3/.style={column name=Absolute value, column type={r}},
      display columns/4/.style={column name=Difference to control, column type={r}, column type/.add={}{|}},
      every head row/.style={
        before row={\toprule
					\multicolumn{1}{c}{Age of intervention} & \multicolumn{1}{c}{Outcome} & \multicolumn{1}{c}{Control} & \multicolumn{2}{c}{Intervention}\\
					},
        after row={\midrule}
            },
        every nth row={10}{before row=\midrule},
        every last row/.style={after row=\bottomrule},
    ]{CSV/Res_HOF_PSA_4_sex_0_ldl_5.csv}
  \end{center}
\end{table}



\begin{table}[h!]
  \begin{center}
    \caption{PSA results -- Low/moderate intensity statins -- All males}
    \label{Microsim1PSA10}
	\hspace*{-2.00cm}
     \fontsize{1pt}{1.5pt}\selectfont\pgfplotstabletypeset[
      multicolumn names,
      col sep=colon,
      header=false,
      string type,
	  display columns/0/.style={column name=,
		assign cell content/.code={
\pgfkeyssetvalue{/pgfplots/table/@cell content}
{\multirow{10}{*}{##1}}}},
      display columns/1/.style={column name=, column type={l}, text indicator="},
      display columns/2/.style={column name=, column type={r}, column type/.add={|}{|}},
      display columns/3/.style={column name=Absolute value, column type={r}},
      display columns/4/.style={column name=Difference to control, column type={r}, column type/.add={}{|}},
      every head row/.style={
        before row={\toprule
					\multicolumn{1}{c}{Age of intervention} & \multicolumn{1}{c}{Outcome} & \multicolumn{1}{c}{Control} & \multicolumn{2}{c}{Intervention}\\
					},
        after row={\midrule}
            },
        every nth row={10}{before row=\midrule},
        every last row/.style={after row=\bottomrule},
    ]{CSV/Res_HOF_PSA_1_sex_1_ldl_0.csv}
  \end{center}
\end{table}

\begin{table}[h!]
  \begin{center}
    \caption{PSA results -- High intensity statins -- All males}
    \label{Microsim2PSA10}
	\hspace*{-2.00cm}
     \fontsize{1pt}{1.5pt}\selectfont\pgfplotstabletypeset[
      multicolumn names,
      col sep=colon,
      header=false,
      string type,
	  display columns/0/.style={column name=,
		assign cell content/.code={
\pgfkeyssetvalue{/pgfplots/table/@cell content}
{\multirow{10}{*}{##1}}}},
      display columns/1/.style={column name=, column type={l}, text indicator="},
      display columns/2/.style={column name=, column type={r}, column type/.add={|}{|}},
      display columns/3/.style={column name=Absolute value, column type={r}},
      display columns/4/.style={column name=Difference to control, column type={r}, column type/.add={}{|}},
      every head row/.style={
        before row={\toprule
					\multicolumn{1}{c}{Age of intervention} & \multicolumn{1}{c}{Outcome} & \multicolumn{1}{c}{Control} & \multicolumn{2}{c}{Intervention}\\
					},
        after row={\midrule}
            },
        every nth row={10}{before row=\midrule},
        every last row/.style={after row=\bottomrule},
    ]{CSV/Res_HOF_PSA_2_sex_1_ldl_0.csv}
  \end{center}
\end{table}


\begin{table}[h!]
  \begin{center}
    \caption{PSA results -- Low/moderate intensity statins and ezetimibe -- All males}
    \label{Microsim3PSA10}
	\hspace*{-2.00cm}
     \fontsize{1pt}{1.5pt}\selectfont\pgfplotstabletypeset[
      multicolumn names,
      col sep=colon,
      header=false,
      string type,
	  display columns/0/.style={column name=,
		assign cell content/.code={
\pgfkeyssetvalue{/pgfplots/table/@cell content}
{\multirow{10}{*}{##1}}}},
      display columns/1/.style={column name=, column type={l}, text indicator="},
      display columns/2/.style={column name=, column type={r}, column type/.add={|}{|}},
      display columns/3/.style={column name=Absolute value, column type={r}},
      display columns/4/.style={column name=Difference to control, column type={r}, column type/.add={}{|}},
      every head row/.style={
        before row={\toprule
					\multicolumn{1}{c}{Age of intervention} & \multicolumn{1}{c}{Outcome} & \multicolumn{1}{c}{Control} & \multicolumn{2}{c}{Intervention}\\
					},
        after row={\midrule}
            },
        every nth row={10}{before row=\midrule},
        every last row/.style={after row=\bottomrule},
    ]{CSV/Res_HOF_PSA_3_sex_1_ldl_0.csv}
  \end{center}
\end{table}

\begin{table}[h!]
  \begin{center}
    \caption{PSA results -- Inclisiran -- All males}
    \label{Microsim4PSA10}
	\hspace*{-2.00cm}
     \fontsize{1pt}{1.5pt}\selectfont\pgfplotstabletypeset[
      multicolumn names,
      col sep=colon,
      header=false,
      string type,
	  display columns/0/.style={column name=,
		assign cell content/.code={
\pgfkeyssetvalue{/pgfplots/table/@cell content}
{\multirow{10}{*}{##1}}}},
      display columns/1/.style={column name=, column type={l}, text indicator="},
      display columns/2/.style={column name=, column type={r}, column type/.add={|}{|}},
      display columns/3/.style={column name=Absolute value, column type={r}},
      display columns/4/.style={column name=Difference to control, column type={r}, column type/.add={}{|}},
      every head row/.style={
        before row={\toprule
					\multicolumn{1}{c}{Age of intervention} & \multicolumn{1}{c}{Outcome} & \multicolumn{1}{c}{Control} & \multicolumn{2}{c}{Intervention}\\
					},
        after row={\midrule}
            },
        every nth row={10}{before row=\midrule},
        every last row/.style={after row=\bottomrule},
    ]{CSV/Res_HOF_PSA_4_sex_1_ldl_0.csv}
  \end{center}
\end{table}


\begin{table}[h!]
  \begin{center}
    \caption{PSA results -- Low/moderate intensity statins -- Males with LDL-C $\geq$3.0 mmol/L}
    \label{Microsim1PSA13}
	\hspace*{-2.00cm}
     \fontsize{1pt}{1.5pt}\selectfont\pgfplotstabletypeset[
      multicolumn names,
      col sep=colon,
      header=false,
      string type,
	  display columns/0/.style={column name=,
		assign cell content/.code={
\pgfkeyssetvalue{/pgfplots/table/@cell content}
{\multirow{10}{*}{##1}}}},
      display columns/1/.style={column name=, column type={l}, text indicator="},
      display columns/2/.style={column name=, column type={r}, column type/.add={|}{|}},
      display columns/3/.style={column name=Absolute value, column type={r}},
      display columns/4/.style={column name=Difference to control, column type={r}, column type/.add={}{|}},
      every head row/.style={
        before row={\toprule
					\multicolumn{1}{c}{Age of intervention} & \multicolumn{1}{c}{Outcome} & \multicolumn{1}{c}{Control} & \multicolumn{2}{c}{Intervention}\\
					},
        after row={\midrule}
            },
        every nth row={10}{before row=\midrule},
        every last row/.style={after row=\bottomrule},
    ]{CSV/Res_HOF_PSA_1_sex_1_ldl_3.csv}
  \end{center}
\end{table}

\begin{table}[h!]
  \begin{center}
    \caption{PSA results -- High intensity statins -- Males with LDL-C $\geq$3.0 mmol/L}
    \label{Microsim2PSA13}
	\hspace*{-2.00cm}
     \fontsize{1pt}{1.5pt}\selectfont\pgfplotstabletypeset[
      multicolumn names,
      col sep=colon,
      header=false,
      string type,
	  display columns/0/.style={column name=,
		assign cell content/.code={
\pgfkeyssetvalue{/pgfplots/table/@cell content}
{\multirow{10}{*}{##1}}}},
      display columns/1/.style={column name=, column type={l}, text indicator="},
      display columns/2/.style={column name=, column type={r}, column type/.add={|}{|}},
      display columns/3/.style={column name=Absolute value, column type={r}},
      display columns/4/.style={column name=Difference to control, column type={r}, column type/.add={}{|}},
      every head row/.style={
        before row={\toprule
					\multicolumn{1}{c}{Age of intervention} & \multicolumn{1}{c}{Outcome} & \multicolumn{1}{c}{Control} & \multicolumn{2}{c}{Intervention}\\
					},
        after row={\midrule}
            },
        every nth row={10}{before row=\midrule},
        every last row/.style={after row=\bottomrule},
    ]{CSV/Res_HOF_PSA_2_sex_1_ldl_3.csv}
  \end{center}
\end{table}


\begin{table}[h!]
  \begin{center}
    \caption{PSA results -- Low/moderate intensity statins and ezetimibe -- Males with LDL-C $\geq$3.0 mmol/L}
    \label{Microsim3PSA13}
	\hspace*{-2.00cm}
     \fontsize{1pt}{1.5pt}\selectfont\pgfplotstabletypeset[
      multicolumn names,
      col sep=colon,
      header=false,
      string type,
	  display columns/0/.style={column name=,
		assign cell content/.code={
\pgfkeyssetvalue{/pgfplots/table/@cell content}
{\multirow{10}{*}{##1}}}},
      display columns/1/.style={column name=, column type={l}, text indicator="},
      display columns/2/.style={column name=, column type={r}, column type/.add={|}{|}},
      display columns/3/.style={column name=Absolute value, column type={r}},
      display columns/4/.style={column name=Difference to control, column type={r}, column type/.add={}{|}},
      every head row/.style={
        before row={\toprule
					\multicolumn{1}{c}{Age of intervention} & \multicolumn{1}{c}{Outcome} & \multicolumn{1}{c}{Control} & \multicolumn{2}{c}{Intervention}\\
					},
        after row={\midrule}
            },
        every nth row={10}{before row=\midrule},
        every last row/.style={after row=\bottomrule},
    ]{CSV/Res_HOF_PSA_3_sex_1_ldl_3.csv}
  \end{center}
\end{table}

\begin{table}[h!]
  \begin{center}
    \caption{PSA results -- Inclisiran -- Males with LDL-C $\geq$3.0 mmol/L}
    \label{Microsim4PSA13}
	\hspace*{-2.00cm}
     \fontsize{1pt}{1.5pt}\selectfont\pgfplotstabletypeset[
      multicolumn names,
      col sep=colon,
      header=false,
      string type,
	  display columns/0/.style={column name=,
		assign cell content/.code={
\pgfkeyssetvalue{/pgfplots/table/@cell content}
{\multirow{10}{*}{##1}}}},
      display columns/1/.style={column name=, column type={l}, text indicator="},
      display columns/2/.style={column name=, column type={r}, column type/.add={|}{|}},
      display columns/3/.style={column name=Absolute value, column type={r}},
      display columns/4/.style={column name=Difference to control, column type={r}, column type/.add={}{|}},
      every head row/.style={
        before row={\toprule
					\multicolumn{1}{c}{Age of intervention} & \multicolumn{1}{c}{Outcome} & \multicolumn{1}{c}{Control} & \multicolumn{2}{c}{Intervention}\\
					},
        after row={\midrule}
            },
        every nth row={10}{before row=\midrule},
        every last row/.style={after row=\bottomrule},
    ]{CSV/Res_HOF_PSA_4_sex_1_ldl_3.csv}
  \end{center}
\end{table}


\begin{table}[h!]
  \begin{center}
    \caption{PSA results -- Low/moderate intensity statins -- Males with LDL-C $\geq$4.0 mmol/L}
    \label{Microsim1PSA14}
	\hspace*{-2.00cm}
     \fontsize{1pt}{1.5pt}\selectfont\pgfplotstabletypeset[
      multicolumn names,
      col sep=colon,
      header=false,
      string type,
	  display columns/0/.style={column name=,
		assign cell content/.code={
\pgfkeyssetvalue{/pgfplots/table/@cell content}
{\multirow{10}{*}{##1}}}},
      display columns/1/.style={column name=, column type={l}, text indicator="},
      display columns/2/.style={column name=, column type={r}, column type/.add={|}{|}},
      display columns/3/.style={column name=Absolute value, column type={r}},
      display columns/4/.style={column name=Difference to control, column type={r}, column type/.add={}{|}},
      every head row/.style={
        before row={\toprule
					\multicolumn{1}{c}{Age of intervention} & \multicolumn{1}{c}{Outcome} & \multicolumn{1}{c}{Control} & \multicolumn{2}{c}{Intervention}\\
					},
        after row={\midrule}
            },
        every nth row={10}{before row=\midrule},
        every last row/.style={after row=\bottomrule},
    ]{CSV/Res_HOF_PSA_1_sex_1_ldl_4.csv}
  \end{center}
\end{table}

\begin{table}[h!]
  \begin{center}
    \caption{PSA results -- High intensity statins -- Males with LDL-C $\geq$4.0 mmol/L}
    \label{Microsim2PSA14}
	\hspace*{-2.00cm}
     \fontsize{1pt}{1.5pt}\selectfont\pgfplotstabletypeset[
      multicolumn names,
      col sep=colon,
      header=false,
      string type,
	  display columns/0/.style={column name=,
		assign cell content/.code={
\pgfkeyssetvalue{/pgfplots/table/@cell content}
{\multirow{10}{*}{##1}}}},
      display columns/1/.style={column name=, column type={l}, text indicator="},
      display columns/2/.style={column name=, column type={r}, column type/.add={|}{|}},
      display columns/3/.style={column name=Absolute value, column type={r}},
      display columns/4/.style={column name=Difference to control, column type={r}, column type/.add={}{|}},
      every head row/.style={
        before row={\toprule
					\multicolumn{1}{c}{Age of intervention} & \multicolumn{1}{c}{Outcome} & \multicolumn{1}{c}{Control} & \multicolumn{2}{c}{Intervention}\\
					},
        after row={\midrule}
            },
        every nth row={10}{before row=\midrule},
        every last row/.style={after row=\bottomrule},
    ]{CSV/Res_HOF_PSA_2_sex_1_ldl_4.csv}
  \end{center}
\end{table}


\begin{table}[h!]
  \begin{center}
    \caption{PSA results -- Low/moderate intensity statins and ezetimibe -- Males with LDL-C $\geq$4.0 mmol/L}
    \label{Microsim3PSA14}
	\hspace*{-2.00cm}
     \fontsize{1pt}{1.5pt}\selectfont\pgfplotstabletypeset[
      multicolumn names,
      col sep=colon,
      header=false,
      string type,
	  display columns/0/.style={column name=,
		assign cell content/.code={
\pgfkeyssetvalue{/pgfplots/table/@cell content}
{\multirow{10}{*}{##1}}}},
      display columns/1/.style={column name=, column type={l}, text indicator="},
      display columns/2/.style={column name=, column type={r}, column type/.add={|}{|}},
      display columns/3/.style={column name=Absolute value, column type={r}},
      display columns/4/.style={column name=Difference to control, column type={r}, column type/.add={}{|}},
      every head row/.style={
        before row={\toprule
					\multicolumn{1}{c}{Age of intervention} & \multicolumn{1}{c}{Outcome} & \multicolumn{1}{c}{Control} & \multicolumn{2}{c}{Intervention}\\
					},
        after row={\midrule}
            },
        every nth row={10}{before row=\midrule},
        every last row/.style={after row=\bottomrule},
    ]{CSV/Res_HOF_PSA_3_sex_1_ldl_4.csv}
  \end{center}
\end{table}

\begin{table}[h!]
  \begin{center}
    \caption{PSA results -- Inclisiran -- Males with LDL-C $\geq$4.0 mmol/L}
    \label{Microsim4PSA14}
	\hspace*{-2.00cm}
     \fontsize{1pt}{1.5pt}\selectfont\pgfplotstabletypeset[
      multicolumn names,
      col sep=colon,
      header=false,
      string type,
	  display columns/0/.style={column name=,
		assign cell content/.code={
\pgfkeyssetvalue{/pgfplots/table/@cell content}
{\multirow{10}{*}{##1}}}},
      display columns/1/.style={column name=, column type={l}, text indicator="},
      display columns/2/.style={column name=, column type={r}, column type/.add={|}{|}},
      display columns/3/.style={column name=Absolute value, column type={r}},
      display columns/4/.style={column name=Difference to control, column type={r}, column type/.add={}{|}},
      every head row/.style={
        before row={\toprule
					\multicolumn{1}{c}{Age of intervention} & \multicolumn{1}{c}{Outcome} & \multicolumn{1}{c}{Control} & \multicolumn{2}{c}{Intervention}\\
					},
        after row={\midrule}
            },
        every nth row={10}{before row=\midrule},
        every last row/.style={after row=\bottomrule},
    ]{CSV/Res_HOF_PSA_4_sex_1_ldl_4.csv}
  \end{center}
\end{table}



\begin{table}[h!]
  \begin{center}
    \caption{PSA results -- Low/moderate intensity statins -- Males with LDL-C $\geq$5.0 mmol/L}
    \label{Microsim1PSA15}
	\hspace*{-2.00cm}
     \fontsize{1pt}{1.5pt}\selectfont\pgfplotstabletypeset[
      multicolumn names,
      col sep=colon,
      header=false,
      string type,
	  display columns/0/.style={column name=,
		assign cell content/.code={
\pgfkeyssetvalue{/pgfplots/table/@cell content}
{\multirow{10}{*}{##1}}}},
      display columns/1/.style={column name=, column type={l}, text indicator="},
      display columns/2/.style={column name=, column type={r}, column type/.add={|}{|}},
      display columns/3/.style={column name=Absolute value, column type={r}},
      display columns/4/.style={column name=Difference to control, column type={r}, column type/.add={}{|}},
      every head row/.style={
        before row={\toprule
					\multicolumn{1}{c}{Age of intervention} & \multicolumn{1}{c}{Outcome} & \multicolumn{1}{c}{Control} & \multicolumn{2}{c}{Intervention}\\
					},
        after row={\midrule}
            },
        every nth row={10}{before row=\midrule},
        every last row/.style={after row=\bottomrule},
    ]{CSV/Res_HOF_PSA_1_sex_1_ldl_5.csv}
  \end{center}
\end{table}

\begin{table}[h!]
  \begin{center}
    \caption{PSA results -- High intensity statins -- Males with LDL-C $\geq$5.0 mmol/L}
    \label{Microsim2PSA15}
	\hspace*{-2.00cm}
     \fontsize{1pt}{1.5pt}\selectfont\pgfplotstabletypeset[
      multicolumn names,
      col sep=colon,
      header=false,
      string type,
	  display columns/0/.style={column name=,
		assign cell content/.code={
\pgfkeyssetvalue{/pgfplots/table/@cell content}
{\multirow{10}{*}{##1}}}},
      display columns/1/.style={column name=, column type={l}, text indicator="},
      display columns/2/.style={column name=, column type={r}, column type/.add={|}{|}},
      display columns/3/.style={column name=Absolute value, column type={r}},
      display columns/4/.style={column name=Difference to control, column type={r}, column type/.add={}{|}},
      every head row/.style={
        before row={\toprule
					\multicolumn{1}{c}{Age of intervention} & \multicolumn{1}{c}{Outcome} & \multicolumn{1}{c}{Control} & \multicolumn{2}{c}{Intervention}\\
					},
        after row={\midrule}
            },
        every nth row={10}{before row=\midrule},
        every last row/.style={after row=\bottomrule},
    ]{CSV/Res_HOF_PSA_2_sex_1_ldl_5.csv}
  \end{center}
\end{table}


\begin{table}[h!]
  \begin{center}
    \caption{PSA results -- Low/moderate intensity statins and ezetimibe -- Males with LDL-C $\geq$5.0 mmol/L}
    \label{Microsim3PSA15}
	\hspace*{-2.00cm}
     \fontsize{1pt}{1.5pt}\selectfont\pgfplotstabletypeset[
      multicolumn names,
      col sep=colon,
      header=false,
      string type,
	  display columns/0/.style={column name=,
		assign cell content/.code={
\pgfkeyssetvalue{/pgfplots/table/@cell content}
{\multirow{10}{*}{##1}}}},
      display columns/1/.style={column name=, column type={l}, text indicator="},
      display columns/2/.style={column name=, column type={r}, column type/.add={|}{|}},
      display columns/3/.style={column name=Absolute value, column type={r}},
      display columns/4/.style={column name=Difference to control, column type={r}, column type/.add={}{|}},
      every head row/.style={
        before row={\toprule
					\multicolumn{1}{c}{Age of intervention} & \multicolumn{1}{c}{Outcome} & \multicolumn{1}{c}{Control} & \multicolumn{2}{c}{Intervention}\\
					},
        after row={\midrule}
            },
        every nth row={10}{before row=\midrule},
        every last row/.style={after row=\bottomrule},
    ]{CSV/Res_HOF_PSA_3_sex_1_ldl_5.csv}
  \end{center}
\end{table}

\begin{table}[h!]
  \begin{center}
    \caption{PSA results -- Inclisiran -- Males with LDL-C $\geq$5.0 mmol/L}
    \label{Microsim4PSA15}
	\hspace*{-2.00cm}
     \fontsize{1pt}{1.5pt}\selectfont\pgfplotstabletypeset[
      multicolumn names,
      col sep=colon,
      header=false,
      string type,
	  display columns/0/.style={column name=,
		assign cell content/.code={
\pgfkeyssetvalue{/pgfplots/table/@cell content}
{\multirow{10}{*}{##1}}}},
      display columns/1/.style={column name=, column type={l}, text indicator="},
      display columns/2/.style={column name=, column type={r}, column type/.add={|}{|}},
      display columns/3/.style={column name=Absolute value, column type={r}},
      display columns/4/.style={column name=Difference to control, column type={r}, column type/.add={}{|}},
      every head row/.style={
        before row={\toprule
					\multicolumn{1}{c}{Age of intervention} & \multicolumn{1}{c}{Outcome} & \multicolumn{1}{c}{Control} & \multicolumn{2}{c}{Intervention}\\
					},
        after row={\midrule}
            },
        every nth row={10}{before row=\midrule},
        every last row/.style={after row=\bottomrule},
    ]{CSV/Res_HOF_PSA_4_sex_1_ldl_5.csv}
  \end{center}
\end{table}

\begin{table}[h!]
  \begin{center}
    \caption{PSA results -- Summary of all interventions.}
    \label{Microsim5PSA}
	\hspace*{-3.00cm}
     \fontsize{1pt}{1.5pt}\selectfont\pgfplotstabletypeset[
      multicolumn names,
      col sep=colon,
      header=false,
      string type,
	  display columns/0/.style={column name=Age of intervention,
		assign cell content/.code={
\pgfkeyssetvalue{/pgfplots/table/@cell content}
{\multirow{10}{*}{##1}}}},
      display columns/1/.style={column name=Outcome, column type={l}, text indicator=", column type/.add={}{|}},
      display columns/2/.style={column name=Control, column type={r}, column type/.add={}{|}},
      display columns/3/.style={column name=\specialcell{\noindent Low/moderate \\ intensity statins}, column type={r}, column type/.add={}{|}},
      display columns/4/.style={column name=High intensity statins, column type={r}, column type/.add={}{|}},
      display columns/5/.style={column name=\specialcell{\noindent Low/moderate intensity \\ statins and ezetimibe}, column type={r}, column type/.add={}{|}},
      display columns/6/.style={column name=Inclisiran, column type={r}, column type/.add={}{|}},
      every head row/.style={
        before row={\toprule
					& & \multicolumn{1}{c}{Absolute value} & \multicolumn{4}{c}{Difference to control}\\
					},
        after row={\midrule}
            },
        every nth row={10}{before row=\midrule},
        every last row/.style={after row=\bottomrule},
    ]{CSV/Res_HOF_PSA.csv}
  \end{center}
\end{table}

\begin{table}[h!]
  \begin{center}
    \caption{PSA results -- Summary of all interventions -- All females.}
    \label{Microsim5PSA00}
	\hspace*{-3.00cm}
     \fontsize{1pt}{1.5pt}\selectfont\pgfplotstabletypeset[
      multicolumn names,
      col sep=colon,
      header=false,
      string type,
	  display columns/0/.style={column name=Age of intervention,
		assign cell content/.code={
\pgfkeyssetvalue{/pgfplots/table/@cell content}
{\multirow{10}{*}{##1}}}},
      display columns/1/.style={column name=Outcome, column type={l}, text indicator=", column type/.add={}{|}},
      display columns/2/.style={column name=Control, column type={r}, column type/.add={}{|}},
      display columns/3/.style={column name=\specialcell{\noindent Low/moderate \\ intensity statins}, column type={r}, column type/.add={}{|}},
      display columns/4/.style={column name=High intensity statins, column type={r}, column type/.add={}{|}},
      display columns/5/.style={column name=\specialcell{\noindent Low/moderate intensity \\ statins and ezetimibe}, column type={r}, column type/.add={}{|}},
      display columns/6/.style={column name=Inclisiran, column type={r}, column type/.add={}{|}},
      every head row/.style={
        before row={\toprule
					& & \multicolumn{1}{c}{Absolute value} & \multicolumn{4}{c}{Difference to control}\\
					},
        after row={\midrule}
            },
        every nth row={10}{before row=\midrule},
        every last row/.style={after row=\bottomrule},
    ]{CSV/Res_HOF_PSA_sex_0_ldl_0.csv}
  \end{center}
\end{table}

\begin{table}[h!]
  \begin{center}
    \caption{PSA results -- Summary of all interventions -- Females with LDL-C $\geq$3.0 mmol/L.}
    \label{Microsim5PSA03}
	\hspace*{-3.00cm}
     \fontsize{1pt}{1.5pt}\selectfont\pgfplotstabletypeset[
      multicolumn names,
      col sep=colon,
      header=false,
      string type,
	  display columns/0/.style={column name=Age of intervention,
		assign cell content/.code={
\pgfkeyssetvalue{/pgfplots/table/@cell content}
{\multirow{10}{*}{##1}}}},
      display columns/1/.style={column name=Outcome, column type={l}, text indicator=", column type/.add={}{|}},
      display columns/2/.style={column name=Control, column type={r}, column type/.add={}{|}},
      display columns/3/.style={column name=\specialcell{\noindent Low/moderate \\ intensity statins}, column type={r}, column type/.add={}{|}},
      display columns/4/.style={column name=High intensity statins, column type={r}, column type/.add={}{|}},
      display columns/5/.style={column name=\specialcell{\noindent Low/moderate intensity \\ statins and ezetimibe}, column type={r}, column type/.add={}{|}},
      display columns/6/.style={column name=Inclisiran, column type={r}, column type/.add={}{|}},
      every head row/.style={
        before row={\toprule
					& & \multicolumn{1}{c}{Absolute value} & \multicolumn{4}{c}{Difference to control}\\
					},
        after row={\midrule}
            },
        every nth row={10}{before row=\midrule},
        every last row/.style={after row=\bottomrule},
    ]{CSV/Res_HOF_PSA_sex_0_ldl_3.csv}
  \end{center}
\end{table}

\begin{table}[h!]
  \begin{center}
    \caption{PSA results -- Summary of all interventions -- Females with LDL-C $\geq$4.0 mmol/L.}
    \label{Microsim5PSA04}
	\hspace*{-3.00cm}
     \fontsize{1pt}{1.5pt}\selectfont\pgfplotstabletypeset[
      multicolumn names,
      col sep=colon,
      header=false,
      string type,
	  display columns/0/.style={column name=Age of intervention,
		assign cell content/.code={
\pgfkeyssetvalue{/pgfplots/table/@cell content}
{\multirow{10}{*}{##1}}}},
      display columns/1/.style={column name=Outcome, column type={l}, text indicator=", column type/.add={}{|}},
      display columns/2/.style={column name=Control, column type={r}, column type/.add={}{|}},
      display columns/3/.style={column name=\specialcell{\noindent Low/moderate \\ intensity statins}, column type={r}, column type/.add={}{|}},
      display columns/4/.style={column name=High intensity statins, column type={r}, column type/.add={}{|}},
      display columns/5/.style={column name=\specialcell{\noindent Low/moderate intensity \\ statins and ezetimibe}, column type={r}, column type/.add={}{|}},
      display columns/6/.style={column name=Inclisiran, column type={r}, column type/.add={}{|}},
      every head row/.style={
        before row={\toprule
					& & \multicolumn{1}{c}{Absolute value} & \multicolumn{4}{c}{Difference to control}\\
					},
        after row={\midrule}
            },
        every nth row={10}{before row=\midrule},
        every last row/.style={after row=\bottomrule},
    ]{CSV/Res_HOF_PSA_sex_0_ldl_4.csv}
  \end{center}
\end{table}

\begin{table}[h!]
  \begin{center}
    \caption{PSA results -- Summary of all interventions -- Females with LDL-C $\geq$5.0 mmol/L.}
    \label{Microsim5PSA05}
	\hspace*{-3.00cm}
     \fontsize{1pt}{1.5pt}\selectfont\pgfplotstabletypeset[
      multicolumn names,
      col sep=colon,
      header=false,
      string type,
	  display columns/0/.style={column name=Age of intervention,
		assign cell content/.code={
\pgfkeyssetvalue{/pgfplots/table/@cell content}
{\multirow{10}{*}{##1}}}},
      display columns/1/.style={column name=Outcome, column type={l}, text indicator=", column type/.add={}{|}},
      display columns/2/.style={column name=Control, column type={r}, column type/.add={}{|}},
      display columns/3/.style={column name=\specialcell{\noindent Low/moderate \\ intensity statins}, column type={r}, column type/.add={}{|}},
      display columns/4/.style={column name=High intensity statins, column type={r}, column type/.add={}{|}},
      display columns/5/.style={column name=\specialcell{\noindent Low/moderate intensity \\ statins and ezetimibe}, column type={r}, column type/.add={}{|}},
      display columns/6/.style={column name=Inclisiran, column type={r}, column type/.add={}{|}},
      every head row/.style={
        before row={\toprule
					& & \multicolumn{1}{c}{Absolute value} & \multicolumn{4}{c}{Difference to control}\\
					},
        after row={\midrule}
            },
        every nth row={10}{before row=\midrule},
        every last row/.style={after row=\bottomrule},
    ]{CSV/Res_HOF_PSA_sex_0_ldl_5.csv}
  \end{center}
\end{table}


\begin{table}[h!]
  \begin{center}
    \caption{PSA results -- Summary of all interventions -- All males.}
    \label{Microsim5PSA10}
	\hspace*{-3.00cm}
     \fontsize{1pt}{1.5pt}\selectfont\pgfplotstabletypeset[
      multicolumn names,
      col sep=colon,
      header=false,
      string type,
	  display columns/0/.style={column name=Age of intervention,
		assign cell content/.code={
\pgfkeyssetvalue{/pgfplots/table/@cell content}
{\multirow{10}{*}{##1}}}},
      display columns/1/.style={column name=Outcome, column type={l}, text indicator=", column type/.add={}{|}},
      display columns/2/.style={column name=Control, column type={r}, column type/.add={}{|}},
      display columns/3/.style={column name=\specialcell{\noindent Low/moderate \\ intensity statins}, column type={r}, column type/.add={}{|}},
      display columns/4/.style={column name=High intensity statins, column type={r}, column type/.add={}{|}},
      display columns/5/.style={column name=\specialcell{\noindent Low/moderate intensity \\ statins and ezetimibe}, column type={r}, column type/.add={}{|}},
      display columns/6/.style={column name=Inclisiran, column type={r}, column type/.add={}{|}},
      every head row/.style={
        before row={\toprule
					& & \multicolumn{1}{c}{Absolute value} & \multicolumn{4}{c}{Difference to control}\\
					},
        after row={\midrule}
            },
        every nth row={10}{before row=\midrule},
        every last row/.style={after row=\bottomrule},
    ]{CSV/Res_HOF_PSA_sex_1_ldl_0.csv}
  \end{center}
\end{table}

\begin{table}[h!]
  \begin{center}
    \caption{PSA results -- Summary of all interventions -- Males with LDL-C $\geq$3.0 mmol/L.}
    \label{Microsim5PSA13}
	\hspace*{-3.00cm}
     \fontsize{1pt}{1.5pt}\selectfont\pgfplotstabletypeset[
      multicolumn names,
      col sep=colon,
      header=false,
      string type,
	  display columns/0/.style={column name=Age of intervention,
		assign cell content/.code={
\pgfkeyssetvalue{/pgfplots/table/@cell content}
{\multirow{10}{*}{##1}}}},
      display columns/1/.style={column name=Outcome, column type={l}, text indicator=", column type/.add={}{|}},
      display columns/2/.style={column name=Control, column type={r}, column type/.add={}{|}},
      display columns/3/.style={column name=\specialcell{\noindent Low/moderate \\ intensity statins}, column type={r}, column type/.add={}{|}},
      display columns/4/.style={column name=High intensity statins, column type={r}, column type/.add={}{|}},
      display columns/5/.style={column name=\specialcell{\noindent Low/moderate intensity \\ statins and ezetimibe}, column type={r}, column type/.add={}{|}},
      display columns/6/.style={column name=Inclisiran, column type={r}, column type/.add={}{|}},
      every head row/.style={
        before row={\toprule
					& & \multicolumn{1}{c}{Absolute value} & \multicolumn{4}{c}{Difference to control}\\
					},
        after row={\midrule}
            },
        every nth row={10}{before row=\midrule},
        every last row/.style={after row=\bottomrule},
    ]{CSV/Res_HOF_PSA_sex_1_ldl_3.csv}
  \end{center}
\end{table}

\begin{table}[h!]
  \begin{center}
    \caption{PSA results -- Summary of all interventions -- Males with LDL-C $\geq$4.0 mmol/L.}
    \label{Microsim5PSA14}
	\hspace*{-3.00cm}
     \fontsize{1pt}{1.5pt}\selectfont\pgfplotstabletypeset[
      multicolumn names,
      col sep=colon,
      header=false,
      string type,
	  display columns/0/.style={column name=Age of intervention,
		assign cell content/.code={
\pgfkeyssetvalue{/pgfplots/table/@cell content}
{\multirow{10}{*}{##1}}}},
      display columns/1/.style={column name=Outcome, column type={l}, text indicator=", column type/.add={}{|}},
      display columns/2/.style={column name=Control, column type={r}, column type/.add={}{|}},
      display columns/3/.style={column name=\specialcell{\noindent Low/moderate \\ intensity statins}, column type={r}, column type/.add={}{|}},
      display columns/4/.style={column name=High intensity statins, column type={r}, column type/.add={}{|}},
      display columns/5/.style={column name=\specialcell{\noindent Low/moderate intensity \\ statins and ezetimibe}, column type={r}, column type/.add={}{|}},
      display columns/6/.style={column name=Inclisiran, column type={r}, column type/.add={}{|}},
      every head row/.style={
        before row={\toprule
					& & \multicolumn{1}{c}{Absolute value} & \multicolumn{4}{c}{Difference to control}\\
					},
        after row={\midrule}
            },
        every nth row={10}{before row=\midrule},
        every last row/.style={after row=\bottomrule},
    ]{CSV/Res_HOF_PSA_sex_1_ldl_4.csv}
  \end{center}
\end{table}

\begin{table}[h!]
  \begin{center}
    \caption{PSA results -- Summary of all interventions -- Males with LDL-C $\geq$5.0 mmol/L.}
    \label{Microsim5PSA15}
	\hspace*{-3.00cm}
     \fontsize{1pt}{1.5pt}\selectfont\pgfplotstabletypeset[
      multicolumn names,
      col sep=colon,
      header=false,
      string type,
	  display columns/0/.style={column name=Age of intervention,
		assign cell content/.code={
\pgfkeyssetvalue{/pgfplots/table/@cell content}
{\multirow{10}{*}{##1}}}},
      display columns/1/.style={column name=Outcome, column type={l}, text indicator=", column type/.add={}{|}},
      display columns/2/.style={column name=Control, column type={r}, column type/.add={}{|}},
      display columns/3/.style={column name=\specialcell{\noindent Low/moderate \\ intensity statins}, column type={r}, column type/.add={}{|}},
      display columns/4/.style={column name=High intensity statins, column type={r}, column type/.add={}{|}},
      display columns/5/.style={column name=\specialcell{\noindent Low/moderate intensity \\ statins and ezetimibe}, column type={r}, column type/.add={}{|}},
      display columns/6/.style={column name=Inclisiran, column type={r}, column type/.add={}{|}},
      every head row/.style={
        before row={\toprule
					& & \multicolumn{1}{c}{Absolute value} & \multicolumn{4}{c}{Difference to control}\\
					},
        after row={\midrule}
            },
        every nth row={10}{before row=\midrule},
        every last row/.style={after row=\bottomrule},
    ]{CSV/Res_HOF_PSA_sex_1_ldl_5.csv}
  \end{center}
\end{table}

\end{landscape}


\clearpage

I think it would also be nice to have another concise summary tables to present in the main body
of the paper to demonstrate the changes in outcomes by sex and LDL-C. Thus, I will make a table containing
only the incremental QALYs and costs (per person), by intervention, sex, and LDL-C. 


\color{Blue4}
\begin{stlog}\input{log/108.log.tex}\end{stlog}
\color{black}

\begin{landscape}

\begin{table}[h!]
  \begin{center}
    \caption{Summary of all interventions by LDL-C -- Females.}
    \label{PSAsum0}
	\hspace*{-2cm}
     \fontsize{5pt}{6.5pt}\selectfont\pgfplotstabletypeset[
      multicolumn names,
      col sep=colon,
      header=false,
      string type,
	  display columns/0/.style={column name=Intervention,
		assign cell content/.code={
\pgfkeyssetvalue{/pgfplots/table/@cell content}
{\multirow{12}{*}{##1}}}},
	  display columns/1/.style={column name=\specialcell{Age of \\ intervention},
		assign cell content/.code={
\pgfkeyssetvalue{/pgfplots/table/@cell content}
{\multirow{3}{*}{##1}}}},
      display columns/2/.style={column name=Outcome, column type={l}, text indicator=", column type/.add={}{|}},
      display columns/3/.style={column name=All, column type={r}, column type/.add={}{}},
      display columns/4/.style={column name=$\geq$3.0 mmol/L, column type={r}, column type/.add={}{}},
      display columns/5/.style={column name=$\geq$4.0 mmol/L, column type={r}, column type/.add={}{}},
      display columns/6/.style={column name=$\geq$5.0 mmol/L, column type={r}, column type/.add={}{}},
      every head row/.style={
        before row={\toprule
					& & & \multicolumn{4}{c}{LDL-C}\\
					},
        after row={\midrule}
            },
        every nth row={12}{before row=\midrule},
        every last row/.style={after row=\bottomrule},
    ]{CSV/Res_HOF_PSA_sum0.csv}
  \end{center}
\end{table}

\begin{table}[h!]
  \begin{center}
    \caption{Summary of all interventions by LDL-C -- Males.}
    \label{PSAsum1}
	\hspace*{-2cm}
     \fontsize{5pt}{6.5pt}\selectfont\pgfplotstabletypeset[
      multicolumn names,
      col sep=colon,
      header=false,
      string type,
	  display columns/0/.style={column name=Intervention,
		assign cell content/.code={
\pgfkeyssetvalue{/pgfplots/table/@cell content}
{\multirow{12}{*}{##1}}}},
	  display columns/1/.style={column name=\specialcell{Age of \\ intervention},
		assign cell content/.code={
\pgfkeyssetvalue{/pgfplots/table/@cell content}
{\multirow{3}{*}{##1}}}},
      display columns/2/.style={column name=Outcome, column type={l}, text indicator=", column type/.add={}{|}},
      display columns/3/.style={column name=All, column type={r}, column type/.add={}{}},
      display columns/4/.style={column name=$\geq$3.0 mmol/L, column type={r}, column type/.add={}{}},
      display columns/5/.style={column name=$\geq$4.0 mmol/L, column type={r}, column type/.add={}{}},
      display columns/6/.style={column name=$\geq$5.0 mmol/L, column type={r}, column type/.add={}{}},
      every head row/.style={
        before row={\toprule
					& & & \multicolumn{4}{c}{LDL-C}\\
					},
        after row={\midrule}
            },
        every nth row={12}{before row=\midrule},
        every last row/.style={after row=\bottomrule},
    ]{CSV/Res_HOF_PSA_sum1.csv}
  \end{center}
\end{table}

\end{landscape}


\subsection{Results: Simulations in a CE plane}

Finally, as the last result from the PSA, the results of the simulations can be presented in a common CE plane.

\color{Blue4}
\begin{stlog}\input{log/109.log.tex}\end{stlog}
\begin{figure}
    \centering
    \includegraphics[width=0.5\textwidth]{log/110.pdf}
    \caption{PSA simulations presented in a common cost-effectiveness plane, by age of intervention.}
    \label{Scatter0}
\end{figure}
\begin{stlog}\input{log/110.log.tex}\end{stlog}
\color{black}

Right, so I think it's worth repeating that figure (figure~\ref{Scatter0})
without Inclisiran, so the other interventions can be seen. 

\color{Blue4}
\begin{stlog}\input{log/111.log.tex}\end{stlog}
\begin{figure}
    \centering
    \includegraphics[width=0.5\textwidth]{log/112.pdf}
    \caption{PSA simulations presented in a common cost-effectiveness plane, by age of intervention, excluding Inclisiran. Solid line: \textsterling 20,000 per QALY willingness-to-pay threshold; dashed line: \textsterling 30,000 per QALY willingness-to-pay threshold}
    \label{Scatter1}
\end{figure}
\begin{stlog}\input{log/112.log.tex}\end{stlog}
\color{black}

Much better (figure~\ref{Scatter1}). 
These three interventions are cost-effective at both thresholds at all ages, with 
the exception of low/moderate intensity statins and ezetimibe at age 60, 
where 100\% of simulations are cost-effective at the \textsterling 30,000 
per QALY willingness-to-pay threshold, but only 50\% meet the \textsterling
20,000 threshold. Additionally, in the total population, none of the interventions
are cost-saving. Inclisiran is not cost-effective in any simulation. 

As usual, let's now stratify by sex.

\color{Blue4}
\begin{stlog}\input{log/113.log.tex}\end{stlog}
\begin{figure}
    \centering
    \includegraphics[width=0.8\textwidth]{log/114.pdf}
    \caption{PSA simulations presented in a common cost-effectiveness plane, by age of intervention and sex.}
    \label{Scatter0sex}
\end{figure}
\begin{figure}
    \centering
    \includegraphics[width=0.8\textwidth]{log/114_1.pdf}
    \caption{PSA simulations presented in a common cost-effectiveness plane, by age of intervention and sex, excluding Inclisiran. Solid line: \textsterling 20,000 per QALY willingness-to-pay threshold; dashed line: \textsterling 30,000 per QALY willingness-to-pay threshold}
    \label{Scatter1sex}
\end{figure}
\begin{stlog}\input{log/114.log.tex}\end{stlog}
\color{black}

This is much more interesting (figure~\ref{Scatter1sex}) -- 
all simulations are cost-effective for the first three interventions in males at all ages,
whereas only the first have most simulations under both willingness-to-pay thresholds.
By LDL-C: 

\color{Blue4}
\begin{stlog}\input{log/115.log.tex}\end{stlog}
\begin{figure}
    \centering
    \includegraphics[width=0.8\textwidth]{log/116.pdf}
    \caption{PSA simulations presented in a common cost-effectiveness plane, by age of intervention and LDL-C. Females.}
    \label{Scatter0sexldl0}
\end{figure}
\begin{figure}
    \centering
    \includegraphics[width=0.8\textwidth]{log/116_1.pdf}
    \caption{PSA simulations presented in a common cost-effectiveness plane, by age of intervention and LDL-C. Males.}
    \label{Scatter0sexldl1}
\end{figure}
\begin{figure}
    \centering
    \includegraphics[width=0.8\textwidth]{log/116_2.pdf}
    \caption{PSA simulations presented in a common cost-effectiveness plane, by age of intervention and LDL-C, excluding Inclisiran. Females. Solid line: \textsterling 20,000 per QALY willingness-to-pay threshold; dashed line: \textsterling 30,000 per QALY willingness-to-pay threshold}
    \label{Scatter0sexldl0}
\end{figure}
\begin{figure}
    \centering
    \includegraphics[width=0.8\textwidth]{log/116_3.pdf}
    \caption{PSA simulations presented in a common cost-effectiveness plane, by age of intervention and LDL-C, excluding Inclisiran. Males. Solid line: \textsterling 20,000 per QALY willingness-to-pay threshold; dashed line: \textsterling 30,000 per QALY willingness-to-pay threshold}
    \label{Scatter0sexldl1}
\end{figure}
\begin{stlog}\input{log/116.log.tex}\end{stlog}
\color{black}

\clearpage
\pagebreak
\section{Scenario analyses}
\label{Sceansec}

It's also of interest to check some scenarios for this analysis. 
The discounting rate is particularly interesting in this analysis, 
as the analysis timespan is different from different ages, and a
steep discounting rate could have major implications across 
such a dramatic age span. Additionally, statins notoriously have poor
adherence, so it's worth looking at what happens when adherence to statins
drops in two scenarios. The first scenario will assume the worst case -- people still get their
prescriptions (and so incur a cost) but the benefit fades. Inclisiran
doesn't suffer from the same issue -- if people aren't adherent, 
it's because they didn't show up to the doctor's appointment, not because they
got Inclisiran dispensed and didn't take it, so both effect and cost would be
removed from the analysis, making little/no difference to the ICER. 
However, Inclisiran is a new drug, and the long-term efficacy on lowering LDL-C
is unclear, so this first scenario could also shed some light on the expected benefits/costs
of Inclisiran if its' efficacy decreases over time. Second, the more likely
non-adherence case: people stop taking their prescription, but don't get it dispensed, 
so do not accrue benefit or cost. 

Thus, the following scenario analyses will be conducted:

\begin{itemize}
\item Discounting at 0\%
\item Discounting at 1.5\%
\item The interventions decrease in effectiveness on LDL-C by 1\% per year, 
and for statin-based interventions only,
a random sample of 20\% stop taking them immediately (but still incur costs). 
\item 40\% of people are immediately non-adherent to therapy, and do not incur costs. 
\end{itemize}

The first two only need changes to the utility and cost matrices, 
but the third and fourth require re-simulation.

\subsection{code}

\color{Blue4}
\begin{stlog}\input{log/117.log.tex}\end{stlog}
\color{black}

\clearpage

\subsection{Results}

\begin{table}[h!]
  \begin{center}
    \caption{Microsimulation results -- Low/moderate intensity statins. Scenario 1: discounting rate set at 0\%}
    \label{Microsim1Sce1}
     \fontsize{6pt}{8pt}\selectfont\pgfplotstabletypeset[
      multicolumn names,
      col sep=colon,
      header=false,
      string type,
	  display columns/0/.style={column name=,
		assign cell content/.code={
\pgfkeyssetvalue{/pgfplots/table/@cell content}
{\multirow{10}{*}{##1}}}},
      display columns/1/.style={column name=, column type={l}, text indicator="},
      display columns/2/.style={column name=, column type={r}, column type/.add={|}{|}},
      display columns/3/.style={column name=Absolute value, column type={r}},
      display columns/4/.style={column name=Difference to control, column type={r}, column type/.add={}{|}},
      every head row/.style={
        before row={\toprule
					\multicolumn{1}{c}{Age of intervention} & \multicolumn{1}{c}{Outcome} & \multicolumn{1}{c}{Control} & \multicolumn{2}{c}{Intervention}\\
					},
        after row={\midrule}
            },
        every nth row={10}{before row=\midrule},
        every last row/.style={after row=\bottomrule},
    ]{CSV/Res_HOF_1_SCE1.csv}
  \end{center}
\end{table}

\begin{table}[h!]
  \begin{center}
    \caption{Microsimulation results -- High intensity statins. Scenario 1: discounting rate set at 0\%}
    \label{Microsim2Sce1}
     \fontsize{6pt}{8pt}\selectfont\pgfplotstabletypeset[
      multicolumn names,
      col sep=colon,
      header=false,
      string type,
	  display columns/0/.style={column name=,
		assign cell content/.code={
\pgfkeyssetvalue{/pgfplots/table/@cell content}
{\multirow{10}{*}{##1}}}},
      display columns/1/.style={column name=, column type={l}, text indicator="},
      display columns/2/.style={column name=, column type={r}, column type/.add={|}{|}},
      display columns/3/.style={column name=Absolute value, column type={r}},
      display columns/4/.style={column name=Difference to control, column type={r}, column type/.add={}{|}},
      every head row/.style={
        before row={\toprule
					\multicolumn{1}{c}{Age of intervention} & \multicolumn{1}{c}{Outcome} & \multicolumn{1}{c}{Control} & \multicolumn{2}{c}{Intervention}\\
					},
        after row={\midrule}
            },
        every nth row={10}{before row=\midrule},
        every last row/.style={after row=\bottomrule},
    ]{CSV/Res_HOF_2_SCE1.csv}
  \end{center}
\end{table}


\begin{table}[h!]
  \begin{center}
    \caption{Microsimulation results -- Low/moderate intensity statins and ezetimibe. Scenario 1: discounting rate set at 0\%}
    \label{Microsim3Sce1}
     \fontsize{6pt}{8pt}\selectfont\pgfplotstabletypeset[
      multicolumn names,
      col sep=colon,
      header=false,
      string type,
	  display columns/0/.style={column name=,
		assign cell content/.code={
\pgfkeyssetvalue{/pgfplots/table/@cell content}
{\multirow{10}{*}{##1}}}},
      display columns/1/.style={column name=, column type={l}, text indicator="},
      display columns/2/.style={column name=, column type={r}, column type/.add={|}{|}},
      display columns/3/.style={column name=Absolute value, column type={r}},
      display columns/4/.style={column name=Difference to control, column type={r}, column type/.add={}{|}},
      every head row/.style={
        before row={\toprule
					\multicolumn{1}{c}{Age of intervention} & \multicolumn{1}{c}{Outcome} & \multicolumn{1}{c}{Control} & \multicolumn{2}{c}{Intervention}\\
					},
        after row={\midrule}
            },
        every nth row={10}{before row=\midrule},
        every last row/.style={after row=\bottomrule},
    ]{CSV/Res_HOF_3_SCE1.csv}
  \end{center}
\end{table}

\begin{table}[h!]
  \begin{center}
    \caption{Microsimulation results -- Inclisiran. Scenario 1: discounting rate set at 0\%}
    \label{Microsim4Sce1}
     \fontsize{6pt}{8pt}\selectfont\pgfplotstabletypeset[
      multicolumn names,
      col sep=colon,
      header=false,
      string type,
	  display columns/0/.style={column name=,
		assign cell content/.code={
\pgfkeyssetvalue{/pgfplots/table/@cell content}
{\multirow{10}{*}{##1}}}},
      display columns/1/.style={column name=, column type={l}, text indicator="},
      display columns/2/.style={column name=, column type={r}, column type/.add={|}{|}},
      display columns/3/.style={column name=Absolute value, column type={r}},
      display columns/4/.style={column name=Difference to control, column type={r}, column type/.add={}{|}},
      every head row/.style={
        before row={\toprule
					\multicolumn{1}{c}{Age of intervention} & \multicolumn{1}{c}{Outcome} & \multicolumn{1}{c}{Control} & \multicolumn{2}{c}{Intervention}\\
					},
        after row={\midrule}
            },
        every nth row={10}{before row=\midrule},
        every last row/.style={after row=\bottomrule},
    ]{CSV/Res_HOF_4_SCE1.csv}
  \end{center}
\end{table}


\begin{table}[h!]
  \begin{center}
    \caption{Microsimulation results -- Low/moderate intensity statins. Scenario 2: discounting rate set at 1.5\%}
    \label{Microsim1Sce2}
     \fontsize{6pt}{8pt}\selectfont\pgfplotstabletypeset[
      multicolumn names,
      col sep=colon,
      header=false,
      string type,
	  display columns/0/.style={column name=,
		assign cell content/.code={
\pgfkeyssetvalue{/pgfplots/table/@cell content}
{\multirow{10}{*}{##1}}}},
      display columns/1/.style={column name=, column type={l}, text indicator="},
      display columns/2/.style={column name=, column type={r}, column type/.add={|}{|}},
      display columns/3/.style={column name=Absolute value, column type={r}},
      display columns/4/.style={column name=Difference to control, column type={r}, column type/.add={}{|}},
      every head row/.style={
        before row={\toprule
					\multicolumn{1}{c}{Age of intervention} & \multicolumn{1}{c}{Outcome} & \multicolumn{1}{c}{Control} & \multicolumn{2}{c}{Intervention}\\
					},
        after row={\midrule}
            },
        every nth row={10}{before row=\midrule},
        every last row/.style={after row=\bottomrule},
    ]{CSV/Res_HOF_1_SCE2.csv}
  \end{center}
\end{table}

\begin{table}[h!]
  \begin{center}
    \caption{Microsimulation results -- High intensity statins. Scenario 2: discounting rate set at 1.5\%}
    \label{Microsim2Sce2}
     \fontsize{6pt}{8pt}\selectfont\pgfplotstabletypeset[
      multicolumn names,
      col sep=colon,
      header=false,
      string type,
	  display columns/0/.style={column name=,
		assign cell content/.code={
\pgfkeyssetvalue{/pgfplots/table/@cell content}
{\multirow{10}{*}{##1}}}},
      display columns/1/.style={column name=, column type={l}, text indicator="},
      display columns/2/.style={column name=, column type={r}, column type/.add={|}{|}},
      display columns/3/.style={column name=Absolute value, column type={r}},
      display columns/4/.style={column name=Difference to control, column type={r}, column type/.add={}{|}},
      every head row/.style={
        before row={\toprule
					\multicolumn{1}{c}{Age of intervention} & \multicolumn{1}{c}{Outcome} & \multicolumn{1}{c}{Control} & \multicolumn{2}{c}{Intervention}\\
					},
        after row={\midrule}
            },
        every nth row={10}{before row=\midrule},
        every last row/.style={after row=\bottomrule},
    ]{CSV/Res_HOF_2_SCE2.csv}
  \end{center}
\end{table}


\begin{table}[h!]
  \begin{center}
    \caption{Microsimulation results -- Low/moderate intensity statins and ezetimibe. Scenario 2: discounting rate set at 1.5\%}
    \label{Microsim3Sce2}
     \fontsize{6pt}{8pt}\selectfont\pgfplotstabletypeset[
      multicolumn names,
      col sep=colon,
      header=false,
      string type,
	  display columns/0/.style={column name=,
		assign cell content/.code={
\pgfkeyssetvalue{/pgfplots/table/@cell content}
{\multirow{10}{*}{##1}}}},
      display columns/1/.style={column name=, column type={l}, text indicator="},
      display columns/2/.style={column name=, column type={r}, column type/.add={|}{|}},
      display columns/3/.style={column name=Absolute value, column type={r}},
      display columns/4/.style={column name=Difference to control, column type={r}, column type/.add={}{|}},
      every head row/.style={
        before row={\toprule
					\multicolumn{1}{c}{Age of intervention} & \multicolumn{1}{c}{Outcome} & \multicolumn{1}{c}{Control} & \multicolumn{2}{c}{Intervention}\\
					},
        after row={\midrule}
            },
        every nth row={10}{before row=\midrule},
        every last row/.style={after row=\bottomrule},
    ]{CSV/Res_HOF_3_SCE2.csv}
  \end{center}
\end{table}

\begin{table}[h!]
  \begin{center}
    \caption{Microsimulation results -- Inclisiran. Scenario 2: discounting rate set at 1.5\%}
    \label{Microsim4Sce2}
     \fontsize{6pt}{8pt}\selectfont\pgfplotstabletypeset[
      multicolumn names,
      col sep=colon,
      header=false,
      string type,
	  display columns/0/.style={column name=,
		assign cell content/.code={
\pgfkeyssetvalue{/pgfplots/table/@cell content}
{\multirow{10}{*}{##1}}}},
      display columns/1/.style={column name=, column type={l}, text indicator="},
      display columns/2/.style={column name=, column type={r}, column type/.add={|}{|}},
      display columns/3/.style={column name=Absolute value, column type={r}},
      display columns/4/.style={column name=Difference to control, column type={r}, column type/.add={}{|}},
      every head row/.style={
        before row={\toprule
					\multicolumn{1}{c}{Age of intervention} & \multicolumn{1}{c}{Outcome} & \multicolumn{1}{c}{Control} & \multicolumn{2}{c}{Intervention}\\
					},
        after row={\midrule}
            },
        every nth row={10}{before row=\midrule},
        every last row/.style={after row=\bottomrule},
    ]{CSV/Res_HOF_4_SCE2.csv}
  \end{center}
\end{table}



\begin{table}[h!]
  \begin{center}
    \caption{Microsimulation results -- Low/moderate intensity statins. Scenario 3: Interventions decrease in efficacy at 1\% per year.}
    \label{Microsim1Sce3}
     \fontsize{6pt}{8pt}\selectfont\pgfplotstabletypeset[
      multicolumn names,
      col sep=colon,
      header=false,
      string type,
	  display columns/0/.style={column name=,
		assign cell content/.code={
\pgfkeyssetvalue{/pgfplots/table/@cell content}
{\multirow{10}{*}{##1}}}},
      display columns/1/.style={column name=, column type={l}, text indicator="},
      display columns/2/.style={column name=, column type={r}, column type/.add={|}{|}},
      display columns/3/.style={column name=Absolute value, column type={r}},
      display columns/4/.style={column name=Difference to control, column type={r}, column type/.add={}{|}},
      every head row/.style={
        before row={\toprule
					\multicolumn{1}{c}{Age of intervention} & \multicolumn{1}{c}{Outcome} & \multicolumn{1}{c}{Control} & \multicolumn{2}{c}{Intervention}\\
					},
        after row={\midrule}
            },
        every nth row={10}{before row=\midrule},
        every last row/.style={after row=\bottomrule},
    ]{CSV/Res_HOF_1_SCE3.csv}
  \end{center}
\end{table}

\begin{table}[h!]
  \begin{center}
    \caption{Microsimulation results -- High intensity statins. Scenario 3: Interventions decrease in efficacy at 1\% per year.}
    \label{Microsim2Sce3}
     \fontsize{6pt}{8pt}\selectfont\pgfplotstabletypeset[
      multicolumn names,
      col sep=colon,
      header=false,
      string type,
	  display columns/0/.style={column name=,
		assign cell content/.code={
\pgfkeyssetvalue{/pgfplots/table/@cell content}
{\multirow{10}{*}{##1}}}},
      display columns/1/.style={column name=, column type={l}, text indicator="},
      display columns/2/.style={column name=, column type={r}, column type/.add={|}{|}},
      display columns/3/.style={column name=Absolute value, column type={r}},
      display columns/4/.style={column name=Difference to control, column type={r}, column type/.add={}{|}},
      every head row/.style={
        before row={\toprule
					\multicolumn{1}{c}{Age of intervention} & \multicolumn{1}{c}{Outcome} & \multicolumn{1}{c}{Control} & \multicolumn{2}{c}{Intervention}\\
					},
        after row={\midrule}
            },
        every nth row={10}{before row=\midrule},
        every last row/.style={after row=\bottomrule},
    ]{CSV/Res_HOF_2_SCE3.csv}
  \end{center}
\end{table}


\begin{table}[h!]
  \begin{center}
    \caption{Microsimulation results -- Low/moderate intensity statins and ezetimibe. Scenario 3: Interventions decrease in efficacy at 1\% per year.}
    \label{Microsim3Sce3}
     \fontsize{6pt}{8pt}\selectfont\pgfplotstabletypeset[
      multicolumn names,
      col sep=colon,
      header=false,
      string type,
	  display columns/0/.style={column name=,
		assign cell content/.code={
\pgfkeyssetvalue{/pgfplots/table/@cell content}
{\multirow{10}{*}{##1}}}},
      display columns/1/.style={column name=, column type={l}, text indicator="},
      display columns/2/.style={column name=, column type={r}, column type/.add={|}{|}},
      display columns/3/.style={column name=Absolute value, column type={r}},
      display columns/4/.style={column name=Difference to control, column type={r}, column type/.add={}{|}},
      every head row/.style={
        before row={\toprule
					\multicolumn{1}{c}{Age of intervention} & \multicolumn{1}{c}{Outcome} & \multicolumn{1}{c}{Control} & \multicolumn{2}{c}{Intervention}\\
					},
        after row={\midrule}
            },
        every nth row={10}{before row=\midrule},
        every last row/.style={after row=\bottomrule},
    ]{CSV/Res_HOF_3_SCE3.csv}
  \end{center}
\end{table}

\begin{table}[h!]
  \begin{center}
    \caption{Microsimulation results -- Inclisiran. Scenario 3: Interventions decrease in efficacy at 1\% per year.}
    \label{Microsim4Sce3}
     \fontsize{6pt}{8pt}\selectfont\pgfplotstabletypeset[
      multicolumn names,
      col sep=colon,
      header=false,
      string type,
	  display columns/0/.style={column name=,
		assign cell content/.code={
\pgfkeyssetvalue{/pgfplots/table/@cell content}
{\multirow{10}{*}{##1}}}},
      display columns/1/.style={column name=, column type={l}, text indicator="},
      display columns/2/.style={column name=, column type={r}, column type/.add={|}{|}},
      display columns/3/.style={column name=Absolute value, column type={r}},
      display columns/4/.style={column name=Difference to control, column type={r}, column type/.add={}{|}},
      every head row/.style={
        before row={\toprule
					\multicolumn{1}{c}{Age of intervention} & \multicolumn{1}{c}{Outcome} & \multicolumn{1}{c}{Control} & \multicolumn{2}{c}{Intervention}\\
					},
        after row={\midrule}
            },
        every nth row={10}{before row=\midrule},
        every last row/.style={after row=\bottomrule},
    ]{CSV/Res_HOF_4_SCE3.csv}
  \end{center}
\end{table}

\begin{table}[h!]
  \begin{center}
    \caption{Microsimulation results -- Low/moderate intensity statins. 
Scenario 4: 40\% of people stop taking therapy immediately.}
    \label{Microsim1Sce4}
     \fontsize{6pt}{8pt}\selectfont\pgfplotstabletypeset[
      multicolumn names,
      col sep=colon,
      header=false,
      string type,
	  display columns/0/.style={column name=,
		assign cell content/.code={
\pgfkeyssetvalue{/pgfplots/table/@cell content}
{\multirow{10}{*}{##1}}}},
      display columns/1/.style={column name=, column type={l}, text indicator="},
      display columns/2/.style={column name=, column type={r}, column type/.add={|}{|}},
      display columns/3/.style={column name=Absolute value, column type={r}},
      display columns/4/.style={column name=Difference to control, column type={r}, column type/.add={}{|}},
      every head row/.style={
        before row={\toprule
					\multicolumn{1}{c}{Age of intervention} & \multicolumn{1}{c}{Outcome} & \multicolumn{1}{c}{Control} & \multicolumn{2}{c}{Intervention}\\
					},
        after row={\midrule}
            },
        every nth row={10}{before row=\midrule},
        every last row/.style={after row=\bottomrule},
    ]{CSV/Res_HOF_1_SCE4.csv}
  \end{center}
\end{table}

\begin{table}[h!]
  \begin{center}
    \caption{Microsimulation results -- High intensity statins. 
Scenario 4: 40\% of people stop taking therapy immediately.}
    \label{Microsim2Sce4}
     \fontsize{6pt}{8pt}\selectfont\pgfplotstabletypeset[
      multicolumn names,
      col sep=colon,
      header=false,
      string type,
	  display columns/0/.style={column name=,
		assign cell content/.code={
\pgfkeyssetvalue{/pgfplots/table/@cell content}
{\multirow{10}{*}{##1}}}},
      display columns/1/.style={column name=, column type={l}, text indicator="},
      display columns/2/.style={column name=, column type={r}, column type/.add={|}{|}},
      display columns/3/.style={column name=Absolute value, column type={r}},
      display columns/4/.style={column name=Difference to control, column type={r}, column type/.add={}{|}},
      every head row/.style={
        before row={\toprule
					\multicolumn{1}{c}{Age of intervention} & \multicolumn{1}{c}{Outcome} & \multicolumn{1}{c}{Control} & \multicolumn{2}{c}{Intervention}\\
					},
        after row={\midrule}
            },
        every nth row={10}{before row=\midrule},
        every last row/.style={after row=\bottomrule},
    ]{CSV/Res_HOF_2_SCE4.csv}
  \end{center}
\end{table}


\begin{table}[h!]
  \begin{center}
    \caption{Microsimulation results -- Low/moderate intensity statins and ezetimibe. 
Scenario 4: 40\% of people stop taking therapy immediately.}
    \label{Microsim3Sce4}
     \fontsize{6pt}{8pt}\selectfont\pgfplotstabletypeset[
      multicolumn names,
      col sep=colon,
      header=false,
      string type,
	  display columns/0/.style={column name=,
		assign cell content/.code={
\pgfkeyssetvalue{/pgfplots/table/@cell content}
{\multirow{10}{*}{##1}}}},
      display columns/1/.style={column name=, column type={l}, text indicator="},
      display columns/2/.style={column name=, column type={r}, column type/.add={|}{|}},
      display columns/3/.style={column name=Absolute value, column type={r}},
      display columns/4/.style={column name=Difference to control, column type={r}, column type/.add={}{|}},
      every head row/.style={
        before row={\toprule
					\multicolumn{1}{c}{Age of intervention} & \multicolumn{1}{c}{Outcome} & \multicolumn{1}{c}{Control} & \multicolumn{2}{c}{Intervention}\\
					},
        after row={\midrule}
            },
        every nth row={10}{before row=\midrule},
        every last row/.style={after row=\bottomrule},
    ]{CSV/Res_HOF_3_SCE4.csv}
  \end{center}
\end{table}

\begin{table}[h!]
  \begin{center}
    \caption{Microsimulation results -- Inclisiran. 
Scenario 4: 40\% of people stop taking therapy immediately.}
    \label{Microsim4Sce4}
     \fontsize{6pt}{8pt}\selectfont\pgfplotstabletypeset[
      multicolumn names,
      col sep=colon,
      header=false,
      string type,
	  display columns/0/.style={column name=,
		assign cell content/.code={
\pgfkeyssetvalue{/pgfplots/table/@cell content}
{\multirow{10}{*}{##1}}}},
      display columns/1/.style={column name=, column type={l}, text indicator="},
      display columns/2/.style={column name=, column type={r}, column type/.add={|}{|}},
      display columns/3/.style={column name=Absolute value, column type={r}},
      display columns/4/.style={column name=Difference to control, column type={r}, column type/.add={}{|}},
      every head row/.style={
        before row={\toprule
					\multicolumn{1}{c}{Age of intervention} & \multicolumn{1}{c}{Outcome} & \multicolumn{1}{c}{Control} & \multicolumn{2}{c}{Intervention}\\
					},
        after row={\midrule}
            },
        every nth row={10}{before row=\midrule},
        every last row/.style={after row=\bottomrule},
    ]{CSV/Res_HOF_4_SCE4.csv}
  \end{center}
\end{table}

\begin{table}[h!]
  \begin{center}
    \caption{Microsimulation results -- Summary of all interventions. Scenario 1: discounting rate set at 0\%. All results shown are the difference between the intervention and control.}
    \label{Microsim5Sce1}
     \fontsize{6pt}{8pt}\selectfont\pgfplotstabletypeset[
      multicolumn names,
      col sep=colon,
      header=false,
      string type,
	  display columns/0/.style={column name=Age of intervention,
		assign cell content/.code={
\pgfkeyssetvalue{/pgfplots/table/@cell content}
{\multirow{4}{*}{##1}}}},
      display columns/1/.style={column name=Outcome, column type={l}, text indicator=", column type/.add={}{|}},
      display columns/2/.style={column name= \specialcell{\noindent Low/moderate \\ intensity statins}, column type={r}},
      display columns/3/.style={column name=High intensity statins, column type={r}, column type/.add={}{}},
      display columns/4/.style={column name=\specialcell{\noindent Low/moderate intensity \\ statins and ezetimibe}, column type={r}},
      display columns/5/.style={column name=Inclisiran, column type={r}, column type/.add={}{}},
      every head row/.style={
        before row={\toprule
					},
        after row={\midrule}
            },
        every nth row={4}{before row=\midrule},
        every last row/.style={after row=\bottomrule},
    ]{CSV/Res_HOF_SCE1.csv}
  \end{center}
\end{table}



\begin{table}[h!]
  \begin{center}
    \caption{Microsimulation results -- Summary of all interventions. Scenario 2: discounting rate set at 1.5\%. All results shown are the difference between the intervention and control.}
    \label{Microsim5Sce2}
     \fontsize{6pt}{8pt}\selectfont\pgfplotstabletypeset[
      multicolumn names,
      col sep=colon,
      header=false,
      string type,
	  display columns/0/.style={column name=Age of intervention,
		assign cell content/.code={
\pgfkeyssetvalue{/pgfplots/table/@cell content}
{\multirow{4}{*}{##1}}}},
      display columns/1/.style={column name=Outcome, column type={l}, text indicator=", column type/.add={}{|}},
      display columns/2/.style={column name= \specialcell{\noindent Low/moderate \\ intensity statins}, column type={r}},
      display columns/3/.style={column name=High intensity statins, column type={r}, column type/.add={}{}},
      display columns/4/.style={column name=\specialcell{\noindent Low/moderate intensity \\ statins and ezetimibe}, column type={r}},
      display columns/5/.style={column name=Inclisiran, column type={r}, column type/.add={}{}},
      every head row/.style={
        before row={\toprule
					},
        after row={\midrule}
            },
        every nth row={4}{before row=\midrule},
        every last row/.style={after row=\bottomrule},
    ]{CSV/Res_HOF_SCE2.csv}
  \end{center}
\end{table}



\begin{table}[h!]
  \begin{center}
    \caption{Microsimulation results -- Summary of all interventions. Scenario 3: Interventions decrease in efficacy at 1\% per year. All results shown are the difference between the intervention and control.}
    \label{Microsim5Sce3}
     \fontsize{6pt}{8pt}\selectfont\pgfplotstabletypeset[
      multicolumn names,
      col sep=colon,
      header=false,
      string type,
	  display columns/0/.style={column name=Age of intervention,
		assign cell content/.code={
\pgfkeyssetvalue{/pgfplots/table/@cell content}
{\multirow{4}{*}{##1}}}},
      display columns/1/.style={column name=Outcome, column type={l}, text indicator=", column type/.add={}{|}},
      display columns/2/.style={column name= \specialcell{\noindent Low/moderate \\ intensity statins}, column type={r}},
      display columns/3/.style={column name=High intensity statins, column type={r}, column type/.add={}{}},
      display columns/4/.style={column name=\specialcell{\noindent Low/moderate intensity \\ statins and ezetimibe}, column type={r}},
      display columns/5/.style={column name=Inclisiran, column type={r}, column type/.add={}{}},
      every head row/.style={
        before row={\toprule
					},
        after row={\midrule}
            },
        every nth row={4}{before row=\midrule},
        every last row/.style={after row=\bottomrule},
    ]{CSV/Res_HOF_SCE3.csv}
  \end{center}
\end{table}



\begin{table}[h!]
  \begin{center}
    \caption{Microsimulation results -- Summary of all interventions. 
Scenario 4: 40\% of people stop taking therapy immediately. 
All results shown are the difference between the intervention and control.}
    \label{Microsim5Sce4}
     \fontsize{6pt}{8pt}\selectfont\pgfplotstabletypeset[
      multicolumn names,
      col sep=colon,
      header=false,
      string type,
	  display columns/0/.style={column name=Age of intervention,
		assign cell content/.code={
\pgfkeyssetvalue{/pgfplots/table/@cell content}
{\multirow{4}{*}{##1}}}},
      display columns/1/.style={column name=Outcome, column type={l}, text indicator=", column type/.add={}{|}},
      display columns/2/.style={column name= \specialcell{\noindent Low/moderate \\ intensity statins}, column type={r}},
      display columns/3/.style={column name=High intensity statins, column type={r}, column type/.add={}{}},
      display columns/4/.style={column name=\specialcell{\noindent Low/moderate intensity \\ statins and ezetimibe}, column type={r}},
      display columns/5/.style={column name=Inclisiran, column type={r}, column type/.add={}{}},
      every head row/.style={
        before row={\toprule
					},
        after row={\midrule}
            },
        every nth row={4}{before row=\midrule},
        every last row/.style={after row=\bottomrule},
    ]{CSV/Res_HOF_SCE4.csv}
  \end{center}
\end{table}

\clearpage
\pagebreak
\section{Threshold analysis}
\label{Thresh}

Finally, given that Inclisiran was not even close to cost-effective 
in any analysis so far, it seems necessary to perform a threshold
analysis to estimate the maximum annual cost at which Inclisiran would
be cost-effective. Recall that the willingness-to-pay threshold
in the UK is a range, from \textsterling 20,000 to \textsterling 30,000; 
thus, I will conduct the threshold analysis at both thresholds. 
The analysis is exactly the same as before, yet instead of using a fixed
cost for Inclisiran, I will estimate the ICER at all costs from 
\textsterling 10 to \textsterling 1,000 (in increments of \textsterling 1), 
and present the maximum annual cost at which the ICER is under each 
threshold. 
Additionally, because the 0\% discounting results above were so different from
the primary analysis (and out of personal interest/to be as generous as possible),
I will estimate the maximum cost using 0\% discounting and the \textsterling 30,000
willingness-to-pay threshold. 

\color{Blue4}
\begin{stlog}\input{log/118.log.tex}\end{stlog}
\begin{stlog}\input{log/119.log.tex}\end{stlog}
\color{black}

So the maximum annual costs, in the overall population, that Inclisiran
would be cost-effective at the \textsterling 20,000 and \textsterling 30,000
willingness-to-pay thresholds when intervening from different ages are: 
\begin{itemize}
\item Age 30: \textsterling 63 at \textsterling 20,000 and \textsterling 88 at \textsterling 30,000
\item Age 40: \textsterling 74 at \textsterling 20,000 and \textsterling 104 at \textsterling 30,000
\item Age 50: \textsterling 65 at \textsterling 20,000 and \textsterling 90 at \textsterling 30,000
\item Age 60: \textsterling 45 at \textsterling 20,000 and \textsterling 60 at \textsterling 30,000
\end{itemize}

And for 0\% discounting:
\begin{itemize}
\item Age 30: \textsterling 123 at \textsterling 20,000 and \textsterling 173 at \textsterling 30,000
\item Age 40: \textsterling 122 at \textsterling 20,000 and \textsterling 172 at \textsterling 30,000
\item Age 50: \textsterling 93 at \textsterling 20,000 and \textsterling 130 at \textsterling 30,000
\item Age 60: \textsterling 56 at \textsterling 20,000 and \textsterling 76 at \textsterling 30,000
\end{itemize}

As usual, it's a good idea to stratify by sex and LDL-C.

\color{Blue4}
\begin{stlog}\input{log/120.log.tex}\end{stlog}
\color{black}

\begin{table}[h!]
  \begin{center}
    \caption{Maximum annual cost of Inclisiran (\textsterling) at which the ICER is under the UK willingness-to-pay (WTP) threshold, by discounting rate, WTP threshold, sex, and LDL-C.}
    \label{Tresh}
     \fontsize{7pt}{9pt}\selectfont\pgfplotstabletypeset[
      multicolumn names,
      col sep=colon,
      header=false,
      string type,
	  display columns/0/.style={column name=Discounting rate,
		assign cell content/.code={
\pgfkeyssetvalue{/pgfplots/table/@cell content}
{\multirow{16}{*}{##1}}}},
	  display columns/1/.style={column name=WTP threshold (\textsterling),
		assign cell content/.code={
\pgfkeyssetvalue{/pgfplots/table/@cell content}
{\multirow{8}{*}{##1}}}},
	  display columns/2/.style={column name=Sex,
		assign cell content/.code={
\pgfkeyssetvalue{/pgfplots/table/@cell content}
{\multirow{4}{*}{##1}}}},
      display columns/3/.style={column name=Age of intervention, column type={l}},
      display columns/4/.style={column name=Overall, column type={r}, column type/.add={|}{}},
      display columns/5/.style={column name=$\geq$3.0 mmol/L, column type={r}},
      display columns/6/.style={column name=$\geq$4.0 mmol/L, column type={r}},
      display columns/7/.style={column name=$\geq$5.0 mmol/L, column type={r}, column type/.add={}{|}},
      every head row/.style={
        before row={\toprule
					& & & & & \multicolumn{3}{c}{LDL-C}\\
					},
        after row={\midrule}
            },
        every last row/.style={after row=\bottomrule},
    ]{CSV/Treshres.csv}
  \end{center}
\end{table}

So, even in the highest risk groups, and under the most generous scenario, the maximum cost-effective price
of Inclisiran doesn't approach the current price. Moreover, given that all other interventions in this study
were cost-effective, the comparator here is probably wrong (i.e.,  it should probably be one of the interventions),
which would make Inclisiran even less cost-effective. 


\clearpage
\bibliography{/Users/jed/Documents/Library.bib}
\end{document}

